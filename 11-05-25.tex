\item[2. Existenz von Lösungen der stationären Navier-Stokes-Gleichung]
Es sei $\Omega\subset \R^3$ offen, zusammenhängend und beschränkt, und es sei $K:\Omega\ra
\R^3$ eine gegeben Funktion. Wir suchen eine Lösung $v:\Omega\ra\R^3$ zum Problem
\begin{align}\label{xyz}
\left.
\begin{array}{rcll}
\eta \Delta v - v\cdot \nabla v - \nabla \rho + K&=&0 & \text{ in } \Omega\\
   \div v&=&0 & \text{ in } \Omega\\
    v &=&0 & \text{ auf } \partial \Omega
   \end{array}\right\}
\end{align}
   mit hinreichend glatten $v,\rho$.

   Eine sehr schöne Herleitung der Gleichung findet man in einem Vortrag auf der Website des Clay
   Mathematics Institute $\ra $ Link sehe Vorlesungssite.
   \end{description}

   \subsection*{Einschub: Sobolev-Räume}

   Wir betrachten die Menge $C^1(\Omega, \R)$ und das darauf definierte Skalarprodukt
   \[
   (u,v)_{H^1}=\int_\Omega u(x)v(x)\d x+ \int_\Omega\nabla u(x) \nabla v(x) \d x.
   \]
   Der normierte Vektorraum $(C^1(\Omega,\R),\sqrt{(\cdot,\cdot)_{H^1}})$ ist nicht vollstöndig. Die
   Vervollständigung bezeichnen wir mit $H^1(\Omega,\R)$. Die Vervollständigung von $C^1(\Omega,\R)$
   bezüglich der $H^1$-Norm bezeichnen wir mit $H_0^1(\Omega,\R)$. Die Räume $H^1$, $H^1_0$ heißen
   Sobolev-Räume und sind (separable) Hilberträume.

   Wir benötigen den folgenden wichtigen Satz aus der linearen Funktionalanalysis:

   \begin{theorem}[Rellich]\label{001}
   Die Einbettung
    \[
        H_0^1(\Omega,\R)\hookrightarrow L^2(\Omega,\R)
    \]
    ist kompakt.
    \end{theorem}

    Weiter gilt die Abschätzung

    \begin{lem}[Poincaré-Friedrichs] \label{002}
    Sei $u\in H_0^1(\Omega,\R)$. Es gilt
    \[
    \int_{\Omega}u^2\leq (d_j)^2\int_\Omega(\partial_ju)^2,
    \]
    wobei $d_j=\sup\{|x_j-y_j|\, | \, (x_1,…,x_n),(y_1,…y_n)\in \Omega\}$, bzw. gilt
    \[
    \|u\|_{L^2}\leq \frac{\diam \,  \Omega}{\sqrt{n}}\|x\|_{H^1}
    \]
    \end{lem}

    \begin{proof}
    Übungsaufgabe. \[ \]
    \end{proof}

    Wir verfeinern nun das Einbettungresultat von Rellich. Dazu benötugen wir das folgende

    \begin{lem}[Ladyzhenshaya]
    Sei $\Omega\subset\R^3$. Für $u\in H_0^1(\Omega,\R)$ gilt
    \[
    \|u\|_{L^4}\leq \sqrt[4]{8} \|u\|_{L^2}^{\nicefrac14}\|\nabla u \|_{L^2}^{\nicefrac34}
    \]    
    \end{lem}

    \begin{proof}
    Wir betrachten zuerst den Fall, dass
    \[
    u^2(x_1,x_2,x_3)=\int_{-\infty}^{x_1} \partial_1 u^2(\xi,x_1,x_3)\d \xi\leq 2\int_{-\infty}
    ^{\infty}|u(\xi,x_2,x_3)\partial_1u(\xi,x_2,x_3)|\d\xi.
    \]
    Es folgt
    \[
    \max_{x_1\in\Omega}u^2(x_1,x_2,x_3)\leq2\int_{-\infty}^\infty|u\cdot\partial_1u| \d x_1
    \]
    Nun lassen wir $x_3$ fest und integrieren über $x_1$ und $x_2$:
    \begin{align*}
    \iint u^4(x_1,x_2,x_3) &\leq \int \max_{x_1} u^2(x_1,x_2,x_3) \d x_2 \cdot \int\max_{x_2}
    u^2(x_1,x_2,x_3)\d x_1\\
        &\leq 4 \iint |u\partial_1 u|  \d x_1 \d x_2 \cdot \iint|u \partial _2 u| 
        \d x_1 \d x_2\\
        &\leq 4\left( \iint u^2 \d x_1 \d x_2\right)^{\nicefrac22}\cdot\left(\iint(\partial_1u)^2
                \d x_1\d x_2\right)^{\nicefrac 12}\cdot\left( \iint(\partial_2 u)^2 \d x_1 \d x_2
                \right)^{\nicefrac 12} \qq \text{(nach Cauchy-Schwarz)}
    \end{align*}
   Jetzt integrieren wir über $x_3$ und bekommen
   \begin{align*}
   \iiint u^4\d x_1\d x_2 \d x_3 &\leq 4 \int \d x_3 \left(\iint u ^2 \d x_1 \d x_2\right) 
    \left( \iint(\partial_1u)^2 + (\partial_2 u)^2 \d x_1 \d x_2 \right)\\
        &\leq 4\left( \iint\max_{x_3} u^2(x_1,x_2,x_3) \d x_1 \d x_2 \right)\left( \iiint 
        (\partial_1 u )^2 + (\partial _2 u)^2 \d x_1 \d x_2 \d x_3\right)\\
        &\leq 8 \iiint | u \partial_3 u | \d x_1 \d x_2 \d x_3 \cdot\left(\iiint (\partial_1u)^2+
        (\partial _2 u)^2 \d x_1 \d x_2 \d x_3\right)\\
        & \leq 8\cdot \left( \iiint u^2 \right)^{\nicefrac 12}\cdot \left( \iiint (\partial_3u)^2
        \right)^{\nicefrac12}\cdot \left( \iiint(\partial_1 u)^2 + (\partial_2u)^2 \right)^{\nicefrac 22}\\
        &\leq 8 \|u\|_{L^2}\cdot\|\nabla u \|_{L^2}^3
    \end{align*}
    Sei nun  $u\in H_0^1(\Omega,\R)$. Wir wählen $(u_j)_{j=1}^\infty$ eine Folge in $C_0^1(\Omega,\R)$,
    so dass
    \[
    u_j\ra u \begin{cases} \text{ in } H_0^1\\ \text{ in } L^2  \end{cases}.
    \]
    Dies ist wegen der Ladyzhenskaya-Ungleichung eine Cauchy-Folge in $L^4$ und konvergiert somit
    gegen $v\in L^4$. Mit Hölder gilt $u=v$ und wir können den Limes in der Ungleichung bilden.
    \[ \]
    \end{proof}

    Mit \textit{Poincaré-Friedrichs} und \textit{Ladyzhenshaya} folgt sofort, dass
    \[
    \|u\|_{L^4}\leq \left( \frac{8\diam\, \Omega}{\sqrt{n}} \right)^{\nicefrac14} \cdot\|u\|_{H^1}
    \]
    für alle $u\in H_0^1(\Omega,\R)$ und damit auch

    \begin{cor}\label{004}
    Die Einbettung
    \[
        H_0^1(\Omega,\R) \hookrightarrow L^4(\Omega,\R)
    \]
    ist kompakt.
    \end{cor}

    \noindent Zurück zu Navier-Stokes: Es sei

    \[
        X=\{v\in C^2(\ol\Omega,\R^3) \, | \, \div \, v=0, v\big|_{\partial\Omega}=0\}
    \]
    und es sei
    \[
        \ms H:=\ol{X}^{H^1(\Omega)} = \{v\in H_0^1(\Omega,\R^3) \, | \, \div \, v =0\}
    \]
    Wir betrachten $\ms H$ als Hilbertraum mit dem Skalarprodukt
    \[
        (u,v)_{\ms H}= \int_\Omega\nabla u \cdot \nabla v \d x
        =\int_\Omega\sum_{k,j=1}^3\partial_ju_k\partial_kv_j
    \]
Dieses ist mittels \textit{Poincaré} äquivalent zum üblichen Skalarprodukt auf $H^1$:

Wir multiplizieren nun die Navier-Stokes-Gleichung (\ref{xyz}) mit $w\in X$ und erhalten

\begin{align*}
    0&=\int_\Omega (\eta\Delta v - v\nabla v + K)w = \int_\Omega \nabla \rho  \cdot w  \\
    \LRa \, 0&=\int_\Omega (\eta \underbrace{\nabla v \cdot \nabla w}_{=\sum_{k,j=1}^3 
    \partial_jv_k\partial_k w_j} -\underbrace{v\cdot v \cdot \nabla w}_{=\sum_{k,j=1}^3
    v_kv_j(\partial_kw_j)}- K\cdot w)
\end{align*}
mit
\[
    a(u,v,w):= \int_\Omega u\cdot(v\cdot \nabla w)
\]
folgt $u$ Lösung von Navier-Stokes
\begin{align}\label{13}
    \Ra \, \eta\cdot (v,w)_{\ms H} - a(v,v,w)- \int_\Omega Kw =0 \qq \forall x \in \ms H
\end{align}

Lösungen von (\ref{13}) heißen schwache Lösungen der Navier-Stokes-Gleichung.

\begin{remark}
    Es gilt
    \begin{align*}
        a(v,v,v)&=\int_\Omega v(v\cdot \nabla v)=\frac12 \int_\Omega\sum_{k,j} v_k\partial_k(v_j,v_j)
        =-\frac12 \int_\Omega\sum(v_j,v_j)\partial_kv_k=0 \qq \text{für } v\in \ms H 
    \end{align*}
\end{remark}
Für $K\in L^2 (\Omega,\R^3)$ ist $\int_\Omega K \cdot$ ein stetiges Funktional auf $\ms H$, somit
existiert nach \textit{Riesz} ein $\tilde K \in \ms H$, so dass
\[
    \int K\cdot w= (\tilde K, w)  \qq \forall w\in \ms H
\]
Dies gilt auch für $a(u,v,\cdot)$ mit $u,v\in \ms H$, somit existiert $B(u,v) \in \ms H$ mit
\[
    a(u,v,w)=(B(u,v),w)_{\ms H}.
\]
Es folgt, dass
\begin{align*}
    (\ref{13}) \, &\LRa \, (\eta v - B(v,v)- \tilde K, w)_{\ms H}=0 \qq \forall w\in \ms H\\
        &\LRa \, \eta v - B(u,v)=\tilde K.
\end{align*}
Es sei nun $Y=L^4(\Omega,\R^3)$. Dank \textit{Ladyzhenshaya-Ungleichung} ist die Einbettung
\[
    \ms H \hookrightarrow Y
\]
kompakt und es gilt mit zweimaliger Anwendung der \textit{Cauchy-Schwarz-Ungleichung}:
\[
    |a(u,v,w)|\leq\|u\|_{L^4}\|v\|_{L^4}\|w\|_{H^1}
\]


\begin{theorem}[Existenz schwacher Lösungen]\label{005}
Sei $\ms H$ ein Hilbertraum, $Y$ ein Banachraum und es sei die Einbettung $\ms H \hookrightarrow Y $
kompakt. Insbesondere sei
\[
    \|u\|_Y\leq \beta \|u\|_{\ms H} \qq \forall u \in \ms H
\]
Es sei $a: \ms H^3 \ra \R$ eine Multilinearform, so dass
\begin{align}\label{14}
    |a(u,v,w)|\leq \|u\|_Y\|v\|_Y\|w\|_{\ms H}
\end{align}
und $a(v,v,v)=0$ für alle $v\in \ms H$. Es sei $\tilde K\in \ms H$, $\eta >0$. Dann existiert
$v\in \ms H$ mit
\[
    \eta(v,w)- a(v,v,w)= (\tilde K, w)_{\ms H} \forall w \in \ms H
\]
\end{theorem}

\begin{proof}
    o.E. sei $\eta=1$. Wir suchen eine Lösung zu $v-B(v,v)+\tilde K=0$. Es gilt
    \begin{align*}
        \|B(v,v)\|_{\ms H}& \leq \alpha \|u\|_Y\|v\|_Y\leq \alpha \beta^2 \|u\|_{\ms H}\|v\|_{\ms H}
    \end{align*}
    Es sei $F(v)=B(v,v)$. $F$ ist lokal Lipschitz, denn für $\|u\|_Y< \rho,$ $\|v\|_Y<\rho$ gilt
    \begin{align*}
        \|F(u)-F(v)\|_{\ms H} &= \|B(u-v,u)-B(v,u-v)\|_{\ms H}\leq 2\alpha \rho^2 \|u-v\|_Y\\
            &\leq 2\alpha\beta \rho^2 \|u-v\|_{\ms H}.
    \end{align*}
    Sei $(v_n)_{n\in \N}$ beschränkt in $\ms H$. Nach Wahl einer Teilfolge ist $v_n$ eine Cauchyfolge
    in $Y$. $\Ra$ $F(v_n)$ ist eine Cauchyfolge in $\ms H$, da gilt
    \begin{align*}
        \|F(u)-F(v)\|_{\ms H}\leq 2\alpha \rho\|u-v\|_Y
    \end{align*}
    Damit ist $F\in \ms C(\ms H,\ms H)$. 

    Nehmen wir nun an, dass $v$ eine Lösung ist von
    \[
        v=tF(v)-t\tilde K \qq \text{für } t\in [0,1]
    \]
    Damit gilt
    \[
        (v,v)_{\ms H} = t\underbrace{a(v,v,v)}_{=0} + t(\tilde K, v)_{\ms H} \, \Ra \,
        \|v\|_{\ms H}\leq \|\tilde K\|_{\ms H}.
    \]

    Die beiden Voraussetzungen des \textit{Leray-Schauder-Prinzips} sind damit erfüllt,
    somit existiert ein Fixpunkt $v$, der $ v=F(v)-\tilde K$ erfüllt. \[ \]
\end{proof}
