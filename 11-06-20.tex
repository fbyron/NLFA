\begin{defi}[Bochner-messbar, Bochner-integrierbar]\label{4.22}
    Eine Funktion $f:S\ra X$ heißt \textit{Bochner-messbar}, falls eine Folge $(f_n)_{n\in\N}$
    von Treppenfunktionen $f_n:S\ra X$ existiert, so dass für fast alle $s\in S$ gilt
    \[
        \lim_{n\ra \infty} \|f_n(s)-f(s)\| _X=0.
    \]
    Falls $f$ in eine solche Folge zudem gilt
    \[
        \lim_{n\ra \infty} \int_S\|f_n(s)-f(s)\|_X \d s =0,
    \]
    so heißt $f$ \textit{Bochner-integrierbar} und schreiben
    \begin{align}\label{26}
        \int_S f(s)\d s=\lim_{n\ra\infty} \int_S f_n(s)\d s
    \end{align}
\end{defi}

\begin{prop}\label{4.23}
    \begin{description}
    \item{1)}
    Falls $f:S\ra X$ Bochner-messbar ist, so ist die Funktion $\|f(\cdot)\|:S\ra \R$ Lebesgue-messbar.
    \item{2)}
    Der Grenzwert in (\ref{26}) existiert für Bochner-integrable Funktionen $f$.
    \item{3)}
    Der Grenzwert in (\ref{26}) ist unabhängig von der Wahl der Folge $(f_n)_{n\in\N}$. 
    \end{description}
\end{prop}

\begin{proof}
    \begin{description}
        \item{1)}
        Es sei $f$ Bochner-messbar, $(f_n)_{n\in \N}$ eine Folge von Treppenfunktionen, so dass
        $\lim_{n\ra \infty}\|f_n(s)-f(s)\|_X=0$ f.f.a. $s\in S$. Dann ist $\|f_n(\cdot)\|:S\ra \R$ eine
        (reelle) Treppenfunktion. Es gilt
        \[
            \lim_{n\ra\infty} \|f_n(s)\| =\|f(s)\| \qq \text{ffa } s\in S 
        \]
        Damit folgt, dass $\|f(\cdot)\|$ Lebesgue-messbar ist als punktweiser Limes messbarer Funktionen.
        \item{2)}
        Es gilt für $n,k\in\N$, dass
        \begin{align*}
            \left\|\int_Sf_n\d s - \int_Sf_k(s)\d s\right\|_X &=\left\| \int_S f_n(s)-f_k(s)\d s
                \right\| _X\leq \int_S\|f_n (s)- f_k(s)\|_X\d s\\ 
            &\leq \int_S\|f_n-f\|_X + \|f_k-f\|_X \d s \ra 0 \qq (n,k\ra \infty)
        \end{align*}
        Damit ist $\int_S f_n(s)\d s$ eine Cauchy-Folge und besitzt einen Grenzwert im Banachraum $X$.
        \item{3)}
        Es sei $(\tilde f _n)_{n\in\N}$ eine andere Folge von Treppenfunktionen, für die gilt
        \[
            \lim_{n\ra \infty} \int_S\|\tilde f_n(s)-f(s)\|_Y \d s=0
        \]
        Wir wählen
        \[
            \hat f_n= \begin{cases} \tilde f_{\nicefrac n2} & \text{$n$ gerade.}\\
                      f_{\nicefrac{(n+1)}{2}} & \text{$n$ ungerade.} \end{cases}
        \]
        Es gilt $\lim_{n\ra \infty}\int_S \hat f_n(s)$ existiert wegen 2) somit auch für Teilfolgen.
        Somit ist gezeigt dass das Bochner-Integral für Bochner-integrierbare Funktionen definiert ist.

    \end{description}
    \[ \]
\end{proof}

\begin{theorem}[Pettis]\label{4.24}
    Es sei $X$ ein separabler Banachraum. Dann ist $f:S\ra X$ genau dann Bochner-messbar, wenn für alle
    $\vp \in X'$ die Funktion
    \[
        \lal \vp,f(\cdot)\ral_X:S\ra \R
    \]
    Lebesgue-messbar ist.
\end{theorem}

\begin{proof}
    \begin{description}
        \item{a)}
        Sei $f$ Bochner-messbar, $(f_n)_{n\in\N}$ die entsprechende Folge von Treppenfunktionen, sei
        $\vp \in X'$. Es folgt f.f.a. $s\in S$, dass
        \[
            \underbrace{\lal \vp, f_n(s)\ral}_{\text{Treppenfunktionen in $\R$}} \ra \lal \vp , f(s)\ral
        \]
        Somit ist $\lal \vp, f(\cdot)\ral$ als punktweiser Limes von reellwertiger Treppenfunktionen 
        Lebesgue-messbar.
        \\[.5cm]
        Wir zeigen zunächst dass $\|f(\cdot)\|_X: S\ra \R$ Lebesgue-messbar ist.

        \begin{lem}\label{27}
            Sei $X$ ein separabler Banachraum. Dann existiert $(\vp_n)_{n\in \N}$ eine Folge in $X'$ mit
            $\|\vp _n\|_{X'}\leq 1$, so dass $\forall \vp_0\in X'$, $\|\vp_0\|\leq 1$ eine Teilfolge
            $(\vp_{n_k})_{k\in\N}$ existiert, mit $\lim_{n\ra \infty}\vp_{n_k}(f)=\vp_0(f)$ für alle 
            $f\in X$.
        \end{lem}
        (Lemma nötig, weil $X'$ nicht unbedingt separabel ist nicht nötig für reflexiven Fall)

        \noindent Es sei $a\in \R$ und definieren
        \[
            A:=\{s: \, \|f(s)\|_X\leq a\}, 
        \]
        sowie
        \[
            A_\vp := \{s: \, |\vp(f(s))|\leq a  \} \qq \text{für } \vp \in X'
        \]
        Falls gilt $A=\bigcap_{j=1}^\infty A_{\vp_j}$, wobei $\vp _j$ die Folge aus dem Lemma oben ist,
        so ist $A$ als abzählbare Schnittmenge messbarer Mengen messbar und a) bewiesen. Es gilt
        natürlich, dass
        \[
            A\subset\bigcap_{\|\vp\|_{X'}\leq 1} A_{\vp},
        \]
        aber wir wissen auch, dass für festes $s$ ein $\vp_0\in X'$ existiert mit $\|\vp_0\|_{X'}=1$ und
        $\vp_0(f(s))=\|f(s)\|_X$ nach \textit{Hahn-Banach}. Damit gilt auch
        \[
            A\supset \bigcap_{\|\vp\|_{X'}\leq 1} A_\vp,
        \]
        somit folgt
        \[
            A=\bigcap _{\|\vp\|_{X'}=1} A_\vp
        \]
        Mit dem Lemma oben folgt aber dass gilt
        \[
            \bigcap _{\|\vp\|_{X'}\leq 1}A_\vp = \bigcap _{j=1}^\infty A_{\vp_j} \qq \text{($\vp_j$ aus
                    dem Lemma)}
        \]
        somit folgt $A=\bigcap_{j=1}^\infty A_{\vp_j}$ und a) ist bewiesen.
        \item{b)}
        Nachdem $X$ separabel ist, können wir für alle $n\in \N$ die Menge $X$ mit eine abzählbaren Menge
        offener Kugeln $\{S_{j,n}\}_{j\in\N}$,
        \[
            S_{j,n} = B_{\nicefrac1n} (f_{j,n})
        \]
        mit Radius $\frac1n$ überdecken. Nach Teil a) gilt, dass
        \[
            \|f(\cdot)- f_{j,n}\|:S\ra \R
        \]
        Lebesgue-Messbar ist für alle $j,n\in\N$. Damit ist aber auch die Menge
        \[
            B_{j,n}:=\{s\in S: \, f(s)\in S_{j,n}\}
        \]
        Lebesgue-messbar. Es gilt $S=\bigcup_{j=1}^\infty B_{j,n}$. Wir setzen 
        \[
            f_n(s)= f_{j,n}\qq \text{für } s\in B_{j,n}':= B_{j,n}\setminus \bigcup_{i=1}^{j-1} B_{i,n}.
        \]
        Es folgt, dass $\|f(s)- f_n(s)\|_X\leq \frac1n$ für alle $s\in S$. Nachdem $B_{j,n}'$
        Lebesgue-messbar ist für alle $j,n\in\N$ ist $f_n(\cdot)$ aber Bochner-messbar, denn auf dem
        endlichen Maßraum kann man $f_n$ durch endliche Treppenfunktionen approximieren, und damit ist
        auch $f$ Bochner-messbar.
    \end{description}
    \[\]
\end{proof}

\begin{remark}
    Zur Approximation $f_n$ abzählbarer Treppenfunktionen durch endliche Treppenfunktionen:
    Wir nehmen für festes $n\in \N$
    \[
        (f_{n,k})_{k\in\N} := \left(\sum_{j=1}^k f_{j,n} \cdot \chi_{B_{j,n}'}\right)
    \]
    und für alle $s\in S$ gilt:
    \[
        f_{n,k}(s) \ra f_n(s) \qq (k\ra \infty)
    \]
    Damit ist $f_n$ Bochner-messbar.
\end{remark}

Nun beweisen wir das Lemma \ref{27}:
\begin{proof}
    es sei $\{f_n\}_{n\in\N}$ dicht in $X$. Wir betrachten
    \[
        \eta_n\cdot \vp \ra \eta_n(\vp)= \begin{pmatrix} \vp(f_1)& \vp(f_2)& \cdots & \vp(f_n)\end{pmatrix}.
    \]
    für $\vp \in S'=\{\vp\in X': \, \|\vp\|_{X'}=1\}$. Der Zielraum ist endlichdimensional,
    also separabel. Somit existiert für festes $n$ eine Folge $\{\vp_{n,k}\}_{k\in\N}$ in $S'$, so dass
    $\{\eta_n(\vp_{n,k})\}_{k\in \N}$ dicht liegt in im Bild $\eta_n(S')$ von $S'$. Es folgt nun,
    dass für jedes $\vp _0\in S'$ eine Teilfolge $(\vp_{n,m_n})_{n\in\N}$ existiert, so dass
    \[
        |\vp_{n,m_n}(f_i)-\vp_0(f_i)|< \frac1n \qq \text{für alle } i\in \{1,…,n\}.
    \]
    damit gilt aber, dass
    \[
        \lim_{n\ra \infty} \vp_{n,m_n}(f_i) = \vp_0(f_i) \qq \forall i \in \N.
    \]
    Wegen der Dichtheit der $\{f_i\}_{i\in\N}$ und der gleichmäßigen Beschränktheit der $\vp_{n,k}$ gilt
    \[
        \lim_{n\ra \infty} \vp_{n,m_n}(f)= \vp _0(f) \qq \forall f\in X.
    \]
\end{proof}

\begin{cor}\label{4.25}
    Sei $X$ ein separabler Banachraum, $f:S\ra X$. Es seien $f_n: S\ra \N$, $n\in\N$ Bochner-messbare
    Funktionen, so dass f.f.a. $s\in S$ gilt
    \[
        f_n(s)\rightharpoonup f(s)
    \]
    Dann ist $f$ Bochner-messbar.
\end{cor}

\begin{proof}
    Sei $F\in X'$. Es gilt $\lal F,f_n(\cdot)\ral$ ist Lebesgue messbar, wegen Theorem \ref{4.24}
    ($\Ra$). und es gilt, dass
    \[
        \lal F,f_n (\cdot)\ral \ra \lal F,f(\cdot)\ral \qq \text{f.ü. auf } S
    \]
    Damit ist $\lal F,f(\cdot)\ral$ Lebesgue-messbar, und wegen Theorem \ref{4.24} ($\Ra$) ist $f$
    Bochner-messbar.
    \[ \]
\end{proof}
