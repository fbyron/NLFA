\newpage

\chapter{Der Leray-Schauder Grad und der Schauder'sche Fixpunktsatz}

Das Ziel ist es, den Fixpunktsatz von Brouwer auf $\infty$-dimensionale Räume zu verallgemeinern.
So einfach ist das nicht.

\begin{theorem}[Kakutani]\label{3.1}
    Sei $H$ ein $\infty$-dimensionaler separabler Hilbertraum. Dann existiert eine stetige Abbildung
    \[
        f:H\ra H,
    \]
    welche die Einheitenkugel in sich selbst abbildet und die keinen Fixpunkt besitzt.
\end{theorem}

\begin{proof}
    Sei $(e_n)_{n\in \Z}$ eine Orhonormalbasis von $H$. Für $x=\sum_{j=-\infty}^\infty\alpha_je_j$
    definieren wir
    \[
        U(x):=\sum_{j=-\infty}^\infty \alpha _je_{j+1};\qq f(x):=\frac12 (1-\|x\|)e_0+U(x)
    \]
    Zu zeigen:
    \begin{description}
    \item{-}
    $U$ ist stetig.
    \item{-}
    $\|U(x)\|<1$ für $\|x\|<1$
    \item{-}
    $f$ besitzt keinen Fixpunkt in $\ol{B_1(0)}$
    \begin{itemize}
        \item bei $x=0$.
        \item auf dem Rand.
        \item im Innern.
    \end{itemize}
    \end{description}
    \[  \]
\end{proof}

\section{Der Abbildungsgrad auf endlichen Banachräumen}

\begin{defi}\label{3.2}
    Sei $X$ ein (reeller) Banachraum von Dimension $n$, und sei $\Phi$ ein Isomorphismus von $X$ nach 
    $\R^n$. Für $f\in D_y(\ol \Omega,X)$, $\Omega \subset X$ offen, beschränkt, $y\in X$ definieren wir
    \[
        \deg(f,\Omega,y)=\deg(\Phi\circ f \circ \Phi^{-1},\Phi(\Omega),\Phi(y))
    \]
\end{defi}

\begin{prop}\label{3.3}
    Der in Definition \ref{3.2} definierte Abbildungsgrad ist unabhängig von der Wahl von $\Phi$.
\end{prop}

\begin{proof}
    Sei $\Psi$ ein zweiter Isomorphismus. Es gilt $A=\Psi\circ\Phi^{-1}\in \Gl(n)$. Wir schreiben
    \[
        f^\star:=\Phi \circ f \circ \Phi^{-1}, \qq y^\star\in \RV(\tilde f^\star)
    \]
    und wählen wie üblich $\tilde f^\star\in D^1_y(\Phi(\ol\Omega),\R^n)$ aus derselben Komponente von
    $D_y(\Phi(\ol \Omega),\R^n)$ wie $f^\star$, so dass $y^\star\in \RV(\tilde f^\star)$. Dann ist auch
    \[
        A\circ \tilde f^\star \circ A^{-1}\in D^1_y(\Psi (\ol \Omega),\R^n)
    \]
    in derselben Komponente von $D_y(\Psi(\ol \Omega),\R^n)$ wie $A\circ f^\star\circ A^{-1}=\Psi\circ
    f\circ \Psi^{-1}$.
    Nach der Kettenregel gilt
    \begin{align*}
        J_{A\circ \tilde f^\star\circ A^{-1}}(Ay^\star)&= \det(A) J_{\tilde f^\star}(y^\star)
        \det(A^{-1})=J_{\tilde f^\star} (y^\star).
    \end{align*}
    Und es folgt
    \[ 
        \deg(\Psi\circ f \circ\Psi^{-1},\Psi(\Omega),\Psi(y))
        = \deg(\Phi\circ f \circ\Phi^{-1},\Phi(\Omega),\Phi(y))
    \]
\end{proof}

Die Eigenschaften des Abbildungsgrades im $\R^n$ übertragen sich mit Definition \ref{3.2} auf den
Abbildungsgrad in endlichdimensionalen Banachräumen. Das gilt ebenfalls für die Reduktionseigenschaft aus
Satz \ref{2.14}, nachdem $\Phi:X\ra\R^n$ so gewählt werden kann, dass $\Phi(X_1)=\R^n$, wobei $X_1$ ein
$m$-dimensionaler Untervektorraum von $X$ ist.

\section{Kompakte Operatoren}

Es seien $X$ und $Y$ Banachräume, $\Omega\subset X$ (nicht unbedingt offen).

\begin{defi}
    Ein stetiger, beschränkter Operator $F:\Omega\ra Y$ heißt
    \begin{description}
        \item{•} \textit{endlichdimenisional}, falls ein endlichdimensionaler Untervektorraum $Y_1\subset Y$
        existiert, sodass $F(\Omega)\subset Y_1$.
        \item{•} \textit{kompakt}, falls $F$ beschränkte Teilmenge von $\Omega$ auf relativ kompakte
        Teilmengen von $Y$ abbildet.
    \end{description}
    Wir bezeichnen die Menge der endlichdimensionalen Operatoren von $\Omega$ nach $Y$ mit
    \[
        \ms F(\Omega,Y)
    \]
    und die kompakten mit
    \[
        \ms C(\Omega,Y)
    \]
\end{defi}

\begin{remark}
    \begin{description}
        \item{1)}
        Es gilt
        \[
            \ms F(\Omega,Y)\subset \ms C(\Omega,Y)\subset C(\Omega,Y)
        \]
        \item{2)}
        Falls $\Omega$ kompakt ist, so gilt
        \[
            \ms C(\Omega,Y)\subset C(\Omega,Y)
        \]
        \item{3)}
        Falls $\dim(Y)<\infty$, so gilt
        \[
            \ms F(\Omega,Y)=\ms C(\Omega,Y)
        \]
        \item{4)}
        Falls $\Omega\subset\R^n$ beschränkt, so gilt
        \[
            \ms F (\ol \Omega,\R^n)=\ms C(\ol \Omega,\R^n)=C(\ol\Omega,\R^n)
        \]
    \end{description}
\end{remark}
