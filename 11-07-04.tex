\begin{prop}\label{4.36}
    Der Raum $C^1(\ol I; V)$ liegt dicht in $W$.
\end{prop}

\begin{proof}
    Folgt wie globale Approximation von Sobolev-Funktionen. Es sei 
    \[
        \tilde u(t)=\begin{cases} u(t) & t\in I \\ u(-t) & t\in(-\tau,0) \\ u(T-t) & t\in (T,T+\tau) \\
        0 & \text{sonst} \end{cases}
    \]
    und es sei $\eta_n(t)$ eine Standard-Diracfolge. Wir betrachten
    \[
        u_n(t)=\int_\R \eta_n(t-s) \tilde u(s) \d s.
    \]
    Es gilt $u_n(t)\in C^1(\ol I;V)$ und es gilt
    \[
        \frac{\d}{\d t} \tilde u_n = \int_\R \eta'_n(t-s) \tilde u(s) \d s
        =\int_\R \eta_n(t-s) \frac{\d}{\d t} u(s) \d s \qq \forall t\in I
    \]
    Die Konvergenz von $u_n\ra u$ in $L^p(I;V)$ und von $\frac{\d}{\d t}u_n:=v_n\ra v$ in $L^{p'}(I;V')$
    folgt wie im skalarwertigen Fall. Mit Benutzung von Proposition \ref{4.35} folgt $v=\frac{\d}{\d t}u$
    und somit die Behauptung. \[ \]
\end{proof}

\begin{prop}\label{4.37}
    Es gilt
    \[
        W\hookrightarrow C(\ol I;V') 
    \]
    stetig.
\end{prop}

\begin{proof}
    Es sei $u\in W$. Damit ist $\frac{\d}{\d t}u \in L^{p'}(I;V')$ eine lokal integrierbare Funktion.
    Wir definieren
    \[
        v(t)=\int_0^t\frac{\d}{\d t} u(\tau)\d \tau,
    \]
    mit $\|v(t)-v(s)\|_{V'}\leq \int_0^t \|\frac{\d}{\d t} u(\tau)\|_{V'}$.
    Somit ist $v(t)\in C(\ol I; V')$. Wie im skalaren Fall folgt, dass
    \[
        v(t)=u(t)+c \qq \text{mit} \, c\in V' \, \text{f.f.a.} \, t\in I.
    \]
    Damit ist auch $u$ stetig. Mit Hölder folgt
    \begin{align*}
        \|v(t)\|&\leq T^{\nicefrac1p}\left\|\frac{\d}{\d t}u\right\|_{L^{p'}(I;V')} \\
        \|c\|_{V'}&\leq T^{-\nicefrac{1}{p}}\|u-v\|_{L^p(I;V)}
    \end{align*}
    Damit gilt aber
    \[
        \max \|u(t)\|_{V'}\leq C\cdot \left( \|u'\|_{L^{p'}(I;V)} + \|u-v\|_{L^{p'}(I;V)}\right)
        \leq \tilde C \|u\|_W
    \]
\end{proof}

Es folgt das wichtige Einbettungslemma

\begin{lem}\label{4.38}
    Es sei $(V,H,V')$ ein Gelfand-Tripel, $W$ der Funktionenraum wie oben. Dann gilt
    \[
        W\hookrightarrow C(\ol C; H)
    \]
    und für alle $u,v\in W$, $s.t \in \ol I$ gilt
    \begin{align}\label{33}
        \int_s^t \lal  \frac{\d u}{\d t}(\tau), v(t)\ral_V+ \lal \frac{\d v}{\d t} (\tau) , u(\tau)\ral_V
        \d\tau =(u(t),v(t))_H - (u(s),v(s))_H
    \end{align}
\end{lem}

\begin{proof}
    \begin{description}
    \item{1)}
    Es seies zunächst $u,v \in C(\ol I; V)$. Es gilt
    \begin{align*}
        (u(t),v(t))_H'&=(u'(t),v(t))_H + (u(t),v'(t))_H\\
        \Ra \, (u(t),v(t))_H-(u(s),v(s))_H &= \int_s^t (u',v)_H + (u,v')_H \overset{\text{Gelfand-Tri.}}
        {=} \int_s^t\lal \frac{\d}{\d t} u, v\ral_V+\lal u,\frac{\d}{\d t} v \ral _V
    \end{align*}
    \item{2)}
    Es sei nun $\vp\in C^1(\ol I, \R)$, mit $\vp(a)=0$, $\vp(b)=1$ mit $0<a<b<T$ und $u\in
    C^1(\ol I;V)$. Wir setzen $v=\vp u$, $w=u-\vp u$. Es gilt
    \begin{align*}
        v'&= \vp'u + \vp u' \\
        w'&=u'-\vp'u-\vp u'
    \end{align*}
    Nachdem \ref{33} bereits für $u,v,w\in C^1(\ol I; V)$ gilt, (Schritt 1) folgt
    \begin{align*}
        (v(t),u(t))_H&= \int_a^t\left\{ \vp'(s)(u(s), u(s))_H+ 2\vp\lal u'(s),u(s)\ral_V\right\} \d s\\
        (-w(t),u(t))_H &= \int_t^b\{ -\vp (s)(u(s),u(s))_H + 2 (1-\vp(s))\lal u'(s),u(s)\ral\} \d s
    \end{align*}
    Durch Subtraktion der beiden Gleichungen erhalten wir
    \[
        \|u(t)\|_H^2=\int_a^b \{\vp'(s)(u(s),u(s))+ 2 \vp(s)\lal u'(s),u(s)\ral_V\}\d s
        =2\int_a^b\lal u'(s),u(s)\ral _V\d s.
    \]
    Somit folgt
    \[
        \|u(t)\|_H^2\leq K\left( \|u\|_{C(\ol I; V)} \cdot \|u\|_{L^p(I;V)} + \left\| \frac{\d}{\d t}
        u\right\|_{L^{p'}(I;V')}\cdot\|u\|_{L^p(I;V)}\right)  \leq K\|u\|^2_W,
    \]
    wegen stetiger Einbettung von $W$ in $C(\ol I;V')$.
    \item{3)}
    Nun sei $u_n$ eine Cauchy-Folge in $C(I,V)$, diegegen $u\in W$ konvergiert. Es gilt
    \[
        \|u_n-u_k\|_{C(\ol I;H)} \leq K\|u_n-u_k\|_W \qq \text{(Schritt 2)}
    \]
    Damit konvergiert die Folge auch in $C(\ol I; H)$ gegen $u$. Die Normabschätzung bleibt im Limes
    erhalten, ebenso die Formel für die prtielle Integration \ref{33}.
    \end{description}
    \[ \]
\end{proof}

\begin{remark}
    \begin{description}
    \item{i)}
    \ref{33} ist analog zur reellwertigen partiellen Integrartion, $u,v:I\ra \R$,
    \[
        \int_s^tu'(\tau)v(\tau) + u(\tau)v'(\tau) \d \tau = u(t)v(t)-u(s)v(s)
    \]
    \item{ii)}
    Für $u=v\in W$ folgt
    \[
        \int_s^t \lal \frac{\d u}{\d t}(\tau), u(\tau)\ral_V \d\tau= \frac12 \|u(t)\|_H^2-\frac12\|u(s)\|
        _H^2
    \]
    \end{description}
\end{remark}

\subsubsection*{Das Anfangswertproblem, Existenz}

Wir betrachten das Problem
\begin{align*}
    \frac{\d u}{\d t} + Au &= b\\
    u(0)&= u_0
\end{align*}
und wollen eine Variante des \textit{Satzes von Brézis} beweisen.

\noindent Es sei also $A:V\ra V'$ ein Operator auf einen separablen, reflexiven Banachraum $V$. Es sei
$u\in L^p(I;V)$, $1<p<\infty$, $I=(0,T)$, $T<\infty$. Wir betrachten den induzierten Operator
\[
    (\tilde Au)(t) := A(u(t)), \qq t\in I.
\]
Unter bestimmeten Voraussetzungen gilt für den induzierten Operator, dass
\[
    \tilde A: L^p(I;V) \ra L^{p'}(I;V') = \left(L^p(I;V))\right)'    
\]
ist. In diesem Fall identifizieren wir $A$ mit $\tilde A$. Im Folgenden sei nun $X:= L^p(I;V)$,
$I=(0,T)$, $T<\infty$, $1<p<\infty$, $W'=L^{p'}(I;V)=(L^p(I;V))'$, $(V,H,V')$ sei ein Gelfand-Tripel.

\begin{theorem}\label{4.39} [Existenz von Lösungen für das Cauchy-Problem]
    Es sei $A:V\ra V'$ ein Operator, so dass der induzierte Operator $A:X\ra X'$ pseudomonoton und
    beschränkt ist, sowie der Koerzitivitätsbedingung
    \[
        \lal Au,u\ral _X\geq C_0\cdot \|u\|_x^p
    \]
    genügt. Dann existiert für alle $u_0\in H$, $b\in X'$ eine Lösung $u\in W$ von
    \[
        \frac{\d u}{\d t}+ Au=b
    \]
\end{theorem}

\begin{remark}
    Die Anfangswertbedingug ergibt wegen der Einbettung
    \[
        W\hookrightarrow C(\ol I;H)
    \]
    Sinn.
\end{remark}

\begin{proof}
    Dieser Beweis verläuft ähnlich dem Existenzbeweis für schwache Lösungen der Wärmeleitungsgleichung.
    Siehe z.B. \textit{Evans: Partial Differential Equations, Chapter 7}.

    \noindent Wir benutzen wieder das Galerkin-Verfahren. Es sei also $\{x_i\}_{i\in \N}\subset V$, so
    dass $\{w_i\}_{i=1}^n$ linear unabhängig sind und $\mr{span}\{w_i\}_{i\in\N}$ dicht liegt in $V$. Wir
    setzen $V_n=\mr{span}\{w_i\}_{i=1}^n$ und suchen eine Lösung $u_n$ der Form
    \[
        u_n(t)=\sum_{k=1}^n C^n_k(t) w_k,
    \]
    die für alle $t\in [0,T]$ das Galerkin-System.
    \begin{align}\label{34}
        \left.\begin{array}{rl} \Lal \frac{\d u_n}{\d t}(t), w_j\Ral _V + \lal Au_n(t),w_j\ral _V &= \lal b_n(t), w_j\ral\\
        u_n(0)&= u_0^n \end{array} \right\}
    \end{align}
    für alle $j=1,…,n$ löst. Hier ist
    \[
        \begin{cases} b_n&\in C(\ol I; V') \\ b_n\ra b \qq &\text{in} \, X'\end{cases}
    \]
    Eine solche Folge exisiert wegen Dichtheit von $C_c^\infty(\ol I;V')$ und
    \[
        u_0^n=\sum_{k=1}^n C_{0k}^nw_i \in V_n, 
    \]
    so dass $u_0^n\ra u_0$ in $H$.
    \begin{description}
    \item{1)}
        Lösbarkeit des Galerkin-Systems. Aufgrund der linearen Unabhängigkeit der $\{ w_i \}_{i=1}^n$ ist
        die Matrix $D=(d_{ij})_{i,j=1}^n=((w_i,w_j)_H)_{i,j=1}^n$ invertierbar. Somit können wir \ref{34}
        schreiben als System gewöhnlicher Differentialgleichungen für die Funktion
        \begin{align*}
            t\mapsto C^n(t)&=\begin{pmatrix} C_1^n(t) & \cdots & C_n^n(t)\end{pmatrix}\\
            \frac{\d C^n(t)}{\d t} &= f^n(t,C^n(t))\\
            C^n(0)&= C_0^n
        \end{align*}
        Wobei
        \[
            f_j^n(t,C)=\sum_{k=1}^n d_{jk}^{-1} (\lal b_n(t), w_k\ral_V-\lal A(\sum_{l=1}^nC_l^nw_l), w_k
            \ral_V)
        \]
        für $j=1,…,n$ und $C_0^n=(C_{01}^n, … , C_{0n}^n)$
