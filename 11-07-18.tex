\begin{beispiel}[Nichtlineare Elastizitätstheorie]
    \[
        F:=\nabla y \in\mathbb M^{3\times 3} \qq \text{Deformationsgradient}
    \]
    Annahme: Lagrangian hängt ab von $F$ (evtl. auch auf $x$, $y$)
    \[
        I(y)=\int_\Omega L(\nabla y(x), y(x), x) \d x
    \]
\end{beispiel}
\subsubsection*{Rahmeninvarianz}

Starkörperbewegung sollten Energieänderung bewirken
\[
    \tilde y(x)= a+Qy(x), \qq a\in\R^3
\]
Wir haben
\begin{align*}
    \nabla \tilde y &= Q\nabla y.\\
    \Ra \, L(QF,\cdot, \cdot) &= L(F,\cdot, \cdot) \qq \forall Q\in \mr{SO}(3), \, F\in 
    \mathbb{M}^{3\times 3}
\end{align*}
Diese Invarianz kollidiert aber leider mit der Konvexitätsannahme. Normieren:
\begin{align*}
    L(\Id) &= 0 \qq\min(L)\\
    L(Q)&= 0, \qq\text{(Rahmeninvarianz)}
\end{align*}
    \[
        \Ra \, L\begin{pmatrix} 1&&\\&1&\\&&1\end{pmatrix}
        = L\begin{pmatrix} -1&&\\&-1&\\&&1\end{pmatrix}
        =L\begin{pmatrix} -1&&\\&1&\\&&-1\end{pmatrix}
        = L\begin{pmatrix}1&&\\&-1&\\&&-1\end{pmatrix}=0
    \]
Aus der Konvexität von $L$ würde nun folgen, dass
\[
    L\left(\frac14 \left[ \begin{pmatrix} 1&&\\&1&\\&&1\end{pmatrix} 
    + \begin{pmatrix} -1&&\\&-1&\\&&1\end{pmatrix}+
        \begin{pmatrix} -1&&\\&1&\\&&-1\end{pmatrix} +
        \begin{pmatrix} 1&&\\&-1&\\&&-1\end{pmatrix}\right]\right) 
    = L\left(\begin{pmatrix}0&0&0\\0&0&0\\0&0&0\end{pmatrix}\right)=0
\]
Das ist aber die Deformation, die den Körper auf einen Punkt schrumpfen lässt diese sollte hohe Energie
besitzen. Konvexitätsannahme und Rahmeninvarianz ergben physökalisch unsinnige Energien. Viel besser wäre
eine Energie der Form
\begin{align}\label{40}
\left.\begin{array}{rlcl}
    L(P,r,z,x)&=L(F,\det F, z ,x)&&\text{$\det F$ Volumenänderung der Deformation.}\\
    (P,r)&\mapsto L(P,r,z,x) &&\text{konvex für alle $z$, $x$.}
\end{array}
\right\}
\end{align}
\[
    \leadsto \, y:\Omega\subset\R^3\ra \R^3 ; \qq I(y)=\int\Omega L(\nabla y(x),\det \nabla y(x),y(x),x)
    \d x.
\]
\begin{remark}
    \[
        \det QF = \det F, \qq Q\in \mr{SO}(3)
    \]
    Eine Energie der Form (\ref{40}), beispielsweise
    \[
        L(F,\det F, z ,x) = \tilde L(F) + (\det F -1)^2
    \]
    mit $\tilde L(F)=\tilde L(QF)$ konvex.
    Existieren Minimierer von Funktionalen der Form (\ref{40})? Unsere „üblichen“ Sätze sind nicht
    anwendbar.
\end{remark}

Funktionale der Form (\ref{40}) nennt man polykonvexe Funktionale. Wir zeigen im Folgenden die Existenz
von Minimierern polykonvexer (und koerziver) Funktionale. Dazu betrachten wir zunächst bestimmte
Lagrangians, deren Euler-Lagrange-Gleichungen von \textit{jeder} glatten Funktion erfüllt wird.

\begin{defi}[Null-Lagrangefunktion]\label{5.5}
    Eine Funktion $L:\R^{n\times m} \times \R^m\times \ol \Omega$ heißt \textit{Null-Lagrangefunktion},
    falls das System der (starken) Euler-Lagrange-Gleichungen
    \[
        -\sum_{i=1}^n \partial_{x_i} (L_{p_{ki}}(\nabla u , u , x) + L_{z_i}(\nabla u , u, x) ) = 0 
        \qq (k=1,…,n)
    \]
    \textit{automatisch} von jeder glatten Funktion $u:\Omega\ra \R^m$ erfüllt wird.
\end{defi}

\begin{remark}
    Für $m=1$ sind die Null-Lagrangians langweilig, sie sind nur für die Funktionen, die affin in
    der Ableitung von $u$ sind (i.e. $L_p(u')=\mr{const}$ $\Ra$ $\partial_xL_p(u')=0$)
\end{remark}

\begin{theorem}[Null-Lagrangian und Randwerte]\label{5.6}
    Es sei $L$ ein Null-Lagrangian, und es seien $u,\tilde u\in C^2(\ol \Omega,\R^m)$ mit $u=\tilde u$
    auf $\partial \Omega$. Dann gilt $I(u)=I(\tilde u)$
    \[
        I(v)=\int_\Omega L(\nabla v(x),v(x), x )\d x.
    \]
\end{theorem}

\begin{proof}
    Wir definieren $i(\tau):= I(\tau u+(1-\tau)\tilde u)$ und leiten ab:
    \begin{align*}
        i'(\tau) &= \int_\Omega \left[\sum_{i=1}^n\sum_{k=1}^m L_{p_{k_i}}(\tau\nabla u + 
        (1-\tau)\nabla\tilde u,\tau u + (1-\tau)\tilde u, x)\cdot(\partial_{x_i} u_k - 
        \partial_{x_i}\tilde u_k) + \sum_{k=1} L^{z_k}(-"-,-"-,-"-)\cdot (u_k-\tilde u_k)\right] \d x \\
        &\overset{\text{part. int}}{=} \sum_{k=1}^m \int_\Omega\left[ - \sum_{i=1} \partial_{x_i}
        L_{p_{k_i}}(-"-,-"-,-"-)+ L_{z_k}(-"-,-"-,-"-) \right] (u_k-\tilde u_k) \d x =0, 
    \end{align*}
    da $\tau u + (1-\tau)\tilde u \in L^2(\ol\Omega)$ die Euler-Lagrange-Gl. erfüllt. Die Randterme
    fallen weg, da $u_k=\tilde u_k$ auf $\partial \Omega$. Die Behauptung folgt. \[ \]
\end{proof}

\noindent \textbf{Notation:} Es sei $A$ eine $n\times n$-Matrix. Mit $\mr{cof}\,A$ bezeichnen wir die 
Kofaktor-Matrix von $A$, d.h.
\[
    (\mr{cof}\, A)_{ki}= (-1)^{i+k} d(A)_{ki}
\]
mit $d(A)_{ki}=$ Determinante der $(n-1)\times(n-1)$-Matrix, die aus $A$ durch Löschen der $k$-ten
Reihe und der $i$-ten Spalte entsteht.
\begin{lem}[Divergenzfreiheit der Reihen der Kofaktormatrix]\label{5.7}
    Es sei $u:\R^n\ra \R^n$ glatt. Dann gilt
    \[
        \sum_{i=1}^n \partial_{x_i} (\mr{cof}\, \nabla u (x))_{ki}=0
    \]
\end{lem}

\begin{proof}
    Das war Übungsaufgabe 8. Im Wesentlichen:
    \[
        (\det P)\cdot \Id = P^T(\mr{cof} \, P)
    \]
    Das leiten wir ab. \[ \]
\end{proof}

\begin{theorem}[Determinanten sind Null-Lagrange-Funktionen]\label{5.8}
    Die Determinantenfunktion
    \[
        L(P)=\det \, P
    \]
    ist ein Null-Lagrangian.
\end{theorem}

\begin{proof}
    Es sei zu zeigen, dass für jede Funktion $u:\Omega\ra \R^n$ gilt
    \[
        \sum_{i=1}^n\partial _{x_i} (L_{p_{k_i}}(\nabla u (x)))=0
    \]
    Es gilt $L^{p_{ki}}=(\mr{cof}\, P)_{ki}$, $i,k=1,…,n$, also nach Lemma \ref{5.7}:
    \[
        \sum_{i=1}^n\partial _{x_i}(L_{p_{ki}}(\nabla u(x)))=\sum_{i=1}^n\partial_{x_i}(\mr{cof}\, \nabla
         u)_{ki} =0
    \]
\end{proof}

\begin{remark}
    Eine Anwendung des Satzes \ref{5.8} ist ein einfacher Beweis des \textit{Brouwer'schen Fixpunktsatzes
    }, siehe \textit{Evans: PDE, THEOREM 8.3}
\end{remark}

\begin{theorem}[Determinanten sind schwach stetige Funktionen]\label{5.9}
    Es sei $n<q<\infty$ und  $u_k\rightharpoonup u$ in $W^{1,q}(\Omega;\R^n)$, $\Omega\Subset \R^n$
    offen. Dann gilt
    \[
        \det \, \nabla u_k \rightharpoonup \det\, \nabla u \qq \text{schwach in }L^{\nicefrac q1}(\Omega)
    \]
\end{theorem}

\begin{proof}
    \begin{description}
        \item{1)}
        Wieder benutzen wir die Matirx-Identität
        \[
            (\det\, P)\Id = P(\mr{cof} \, P)^T
        \]
        bzw. Laplace'scher Entwicklungssatz!
        \[
            \Ra \, \det \, P=\sum_{k=1}^nP_{kj}(\mr{cof}\, P)_{kj}
        \]
        Sei nun $w\in C^\infty (\Omega;\R^n)$. Dann gilt
        \begin{align*}
            \det\, \nabla w &= \sum_{k=1}^n(\partial_{x_k w_j})(\mr{cof}\,\nabla u)_{kj} \qq (j=1,…,n)\\
                &=\underbrace{\sum_{k=1}^n -w_j\cdot (\partial_{x_k}(\mr{cof}\, \nabla w)_{kj})}_{=0, 
                \text{ wegen Lemma \ref{5.7}}} + 
                \sum_{k=1}^n\partial_{x_k}(w_j\cdot(\mr{cof}\, \nabla w)_{kj})
        \end{align*}
        Somit ist die Determinante der Jakobimatrix eine Divergenz.
        (Das haben wir bereits einmal gesehen, siehe Proposition \ref{2.8}) Es sei nun $v\in C_c^\infty
        (\Omega)$, damit folgt
        \[
            \int_\Omega v \det \, \nabla w = -\sum_{k=1}^n\int_{\Omega} (\partial_{x_n}v)\cdot
            w_j(\mr{cof}\, \nabla w)_{ik}
        \]
        Mittels einer Standardapproximation folgt auch
        \[
            \int_\Omega v \det\, \nabla u_l = - \sum_{k=1}^n \int_\Omega(\partial_{x_k}v)(u_l)_j
            (\mr{cof}\, \nabla u_l)_{jk}
        \]
        Für $n<q<\infty$ und $u_k\rightharpoonup u$ in $W^{1,q(\Omega; \R^n)}$ folgt mit dem
        \textit{Morrey'schen Einbettungssatz} (und \textit{Arzelá-Ascoli}), dass $u_k\ra u$ gleichmäßig
        in $\Omega$.\\
        \noindent \textbf{Behauptung:} Es gilt
        \[
            \lim_{l\ra \infty}\int_\Omega \vp (\mr{cof}\, \nabla u_l)_{jk} = \int_\Omega \vp(\mr{cof}\,
            \nabla u)_jk\qq \forall \vp \in C_c^\infty, \, (j,k=1,…,n)
        \]


    \end{description}
\end{proof}
