\documentclass[14pt]{article} 
\usepackage[utf8]{inputenc}
\usepackage{epsfig}
\usepackage[ngerman]{babel}
\usepackage{mathtools,extarrows}
\usepackage[thmmarks]{ntheorem}
\usepackage{amsmath,theorem}
\usepackage{amsfonts}
\usepackage{amssymb}
\usepackage{txfonts}
\usepackage{marvosym}
\usepackage{MnSymbol}
\usepackage{mathrsfs}
\usepackage{parallel}
\usepackage{tikz}
\usepackage{geometry}
\usepackage{titletoc}
\usepackage{url}
\usepackage{nicefrac}
\usepackage[pdftex,bookmarks=true,bookmarksnumbered=true,bookmarksopen=true,colorlinks=true,
		filecolor=black,linkcolor=blue,urlcolor=blue,plainpages=false,pdfpagelabels,
		citecolor=black,setpagesize=false]{hyperref}


\makeatletter
\renewcommand\l@section{\@dottedtocline{1}{0em}{2.5em}}
\renewcommand\l@subsection{\@dottedtocline{2}{2.5em}{2em}}
\makeatother

\titlecontents{chapter}[1.5em]{\addvspace{1em}\bfseries}
	{\contentslabel{1.5em}}{\hspace^{-1.5em}}
	{\hfill\contentspage}
\dottedcontents{section}[3.8em]{}{1.3em}{1pc}
\dottedcontents{subsection}[6.0em]{}{2.2em}{1pc}
\dottedcontents{subsubsection}[9.2em]{}{3.1em}{1pc}

%%%%theoremenvironments
\theoremstyle{plain} 

\theoremheaderfont{\bfseries}

\theorembodyfont{\it}
\newtheorem{theorem}{Theorem}[section]
\newtheorem{lem}[theorem]{Lemma}
\newtheorem{cor}[theorem]{Corollar}
\newtheorem{prop}[theorem]{Proposition}

\theorembodyfont{\rm}
%\newtheorem{remark}[theorem]{Bemerkung}
\newtheorem{defi}[theorem]{Definition}
\newtheorem{rul}[theorem]{Regel}

\theoremstyle{nonumberplain}

\theoremheaderfont{\bfseries}
\theorembodyfont{\normalfont}
\newtheorem{remark}{Bemerkung:}

\theoremheaderfont{%
    \normalfont\scshape}
\theorembodyfont{\normalfont}
\theoremsymbol{\square}
\newtheorem{proof}{Beweis:}
\newtheorem{idea}{Beweisidee:}

\topmargin=-2mm \textwidth=155mm \textheight=230mm
\oddsidemargin=0mm \evensidemargin=0mm
\thispagestyle{empty}
\newcommand{\mb}{\mathbb}
\newcommand{\mc}{\mathcal}
\newcommand{\mf}{\mathfrak}
\newcommand{\mr}{\mathrm}
\newcommand{\ms}{\mathscr}
\newcommand{\K}{\mb{K}}
\newcommand{\C}{\mb{C}}
\newcommand{\R}{\mb{R}}
\newcommand{\Q}{\mb{Q}}
\newcommand{\Z}{\mb{Z}}
\newcommand{\N}{\mb{N}}
\newcommand{\ra}{\rightarrow}
\newcommand{\LRa}{\Leftrightarrow}
\newcommand{\Ra}{\Rightarrow}
\newcommand{\La}{\Leftarrow}
\newcommand{\eps}{\varepsilon}
\newcommand{\vt}{\vartheta}
\newcommand{\vp}{\varphi}
\newcommand{\qq}{\hspace{.25cm}}
\newcommand{\ol}{\overline}
\newcommand{\ul}{\underline}
\renewcommand{\d}{ \, \mr{d}}
\newcommand{\Res}{\mr{Res}}
\newcommand{\Const}{\mr{Const} \, }
\newcommand{\Id}{\mr{Id}}
\renewcommand{\star}{\filledstar}
\newcommand{\Gl}{\mr{Gl}}
\newcommand{\sign}{\mr{sign}\,}
\newcommand{\dist}{\mr{dist}}
\newcommand{\diag}{\mr{diag}}
\newcommand{\CP}{\mr{CP}}
\newcommand{\CV}{\mr{CV}}
\newcommand{\supp}{\mr{supp}\,}
\newcommand{\RV}{\mr{RV}}
\renewcommand{\div}{\mr{div}\,}
\newcommand{\conv}{\mr{conv}}
\newcommand{\diam}{\mr{diam}}
\newcommand{\CS}{\mr{CS}}
\newcommand{\Bild}{\mr{Bild}}
%\renewcommand{\span}{\mr{span}}
\newcommand{\lal}{\langle}
\newcommand{\ral}{\rangle}

\geometry{hmargin=3cm,top=2cm,bottom=4cm}

\newcommand{\makelicense}{Dieser Mitschrieb steht unter der freien \texttt{CC-BY-SA-DE 3.0} 
Lizenz.
\begin{center}\includegraphics[height=31pt,width=88pt]{license.pdf}\end{center}
Für weitere Informationen besuchen Sie
\begin{center}\url{http://creativecommons.org/licenses/by-sa/3.0/deed.de}\end{center}}


\input xy
\xyoption{all}

\title{\sc{Nichtlineare Funktionalanalysis}}
\author{Luis Felipe Müller}
%\date{Sommersemester 2011}

\begin{document}


\maketitle

\hrule \hspace*{1cm}\\[3mm]
{\sc Kompletter Mitschrieb zur gleichnamigen Vorlesung bei Herrn Dondl\\
\hspace*{1cm}\hfill (Sommersemester 2011, Uni Heidelberg)}\\
\hrule
\vspace{1cm}
\noindent\makelicense

%\vfill
%\listoffigures
%\vfill
%\listoftables

\newpage

\pagenumbering{roman}

\section*{Organisatorisches}

\subsection*{Termine}
\begin{itemize}
    \item Vorlesung: Mo./Mi. 9-11ct. -104 Ang. Math.
    \item Übungsaufgaben: Mi. - Mi. vor der Vorlesung. Kasten in der Angew.Math
    \item Übungsgruppe: Fr. 16-18 Angew. Math. -101
    \item Website zur Vorlesung: \url{http://dondl.org/wiki/Sommersemester_11}
    \item Literatur: \begin{enumerate}
        \item Růžička M: Nichtlineare Funktionalanalysis, Eine Einführung
        \item Aubin-Ekeland: Applied nonlinear Analysis
        \item Deimling: Nonlinear Functional Analysis
        \item Schwartz: Nonlinear Functional Analysis
        \item Zeidler: Nonlinear Functional Analysis and its applications 
        \end{enumerate}
    \item Prüfungen: Zulassung mit etwa 50\% der Übungsaufgaben-Punkte. Prüfung ist mündlich, 
            beispielsweise am 29. Juli (Fr.) 
    \item Dozent: Patrick Dondl, Sprechstunde Mo./Mi. 11-12 in Raum 130 (Angew. Math.)
    \item Tutor: Julian Scheuer
\end{itemize}

\newpage

\tableofcontents

\newpage

\pagenumbering{arabic}

\chapter{Einleitung}

\section{Thema der Vorlesung}

In der linearen Funktionalanalysis haben wir eine Vielzahl von Methoden kennengelernt um
Ergebnisse aus der endlichdimensionalen linearen Algebra auf den unendlichdimensionalen Fall 
zu verallgemeinern. Ein Hauptaufgabe war dabei, die Lösbarkeit von Gleichungen der Form
\[
    Ax=y
\]
für lineare Operatoren $A$ auf $\infty$-dimensionalen Banchräumen zu zeigen.
\begin{description}
    \item{Ein Beispiel:}
    
    Sei $\Omega\subset\R^3$, gefüllt mit einer inkompressiblen, riskosen Flüssigkeit.
    $v_j(x)$ sei die Geschwindigkeit der Flüssigkeit an der Stelle $x\in \Omega$. 
    $p(x)$ ist der Druck an der Stelle $x$.
    \begin{description}
    \item{Randbedingungen:}
    $v_j(x)=0 \qq x\in \partial\Omega$
    \item{Inkompressibilität:}
    $\partial_jv_j(x)=0 \qq x\in \Omega$
    \item{Bewegungsungleichung:}
    Wir betrachten die Kräfte, die auf einen kleinen Würfel, eingeschlossen durch
    $(x_1,x_2,x_3)$, $(x_1+\Delta x_1, x_2+\Delta x_2, x_3+\Delta x_3)$. Druck auf eine Oberfläche
    des Würfels mit Normale $x$, ist $f_j^p=p\cdot \Delta x_2 \Delta x_3 \cdot \delta_{ij} $ auf die
    gegenüberliegende Seite wirkt
    \[
        -(p+\partial _i p \Delta x_i) \Delta x_2 \Delta x_3 \delta_{ij}
    \]
    \end{description}
Zusammen ergibt sich 
\[
    f=(\partial_j p)\Delta V
\]

Kraft durch Viskosität auf eine Oberfläche mit Normale $x_1$ ist 
\[
    f_j^{V,x_i}= -2\eta \Delta x_2\Delta x_3 \partial_1v_j
\]
mit einer Konstante $\eta$. Der gleiche Trick wie oben ergibt für die gegenüberliegende
Oberfläche
\[
    \eta \Delta x_2 \Delta x_3 \partial (v_j+\partial v_j\Delta x_1)
\]
Zusammen ergibt sich
\begin{align*}
    f_j^V&= \eta\Delta V \cdot \partial_i\partial_iv_j\\
    \text{Newton:} \qq \rho \Delta V \frac{\d}{\d t} v_j (t,x(t))
        &= \eta \Delta V  \partial _i\partial_iv_j- \Delta V (\partial _j p) + \Delta 
        V\kappa_j 
\end{align*}
mit einer externen Konstante $\kappa_j$, bspw. Gravitation.
Teilen durch $\Delta V$ und die Kettenregel ergibt
\begin{align*}
    \rho \partial_t v_j&= \eta \partial_i\partial_i v_j - \rho (v_i\partial_i) - \partial_j p + K_j 
        \qq \text{(Newton)}\\
    d_jv_j&=0
\end{align*}
Navier-Stokes. Frage: Existiert eine eindeutige Lösung zur sationären Navier-Stokes-Gleichung:
\begin{align}\label{1}
    \eta \partial_i\partial_iv_j-(v_i\partial_i)v_j+\partial_jp+K_j=0
\end{align}
\end{description}
Wir können die Gleichung etwas umschreiben. Sei $H$ ein Hilbertraum
\[
    H:=\ol{\left\{ u\in C^\infty_c (\Omega,\R^3), \text{ so dass } \partial_jv_j=0 \right\}}
        ^{W^{1,2}(\Omega, \R^3)}
\]
Ein Skalarprodukt auf $H$ ist gegeben durch
\begin{align*}
    (u,v)_H:= \int_{\Omega} \nabla u\cdot \nabla v \d x = \int_\Omega \sum_{i,j=1}^3 
    (\partial_iu_j)(\partial_iv_j)
\end{align*}
$\Omega$ beschränkt $\Ra$ $(\cdot, \cdot)_H$ ist äquivalent zum üblichen Skalarprodukt 
(mittels Poincaré).
Wir multiplizieren (\ref{1}) mit $\omega \in H $, integrieren und erhalten
\begin{align*}
    \int_\Omega (\eta \partial_i\partial_i v_j- (v,\partial_i) v_j + K_j) \cdot \omega_j 
    &= \int_\Omega (\partial _jp)\omega_j=0 \text{, da $\omega$ divergenzfrei.}\\
    (\ref{1}) \Ra \eta (v,\omega)_H-a(v,v,w)-\int_\Omega K\omega&=0
\end{align*}
Ebenso für $a$:
\begin{align*}
    a(u,v,w):= (\underbrace{B(u,v)}_{\text{bilinear.}},w)_H
\end{align*}
Also
\begin{align*}
    (\ref{1}) \Ra \, (\eta v - B(u,v)) - \tilde K , \omega)_H=0 \qq \forall \omega \in H
\end{align*}
somit
\[
    \eta v - B(v,v) = \tilde K
\]
Das ist eine Gleichung der Form
\begin{align}\label{2}
    Fv=\tilde K,\qq\text{mit $F$ einem Nichtlinearen Operator}
\end{align}
Im ersten Teil der Vorlesung beschäftigen wir uns mit der eindeutigen Lösbarkeit von
Gleichungen der Form
\[
    Fx=y, \qq F:X\ra Y, \qq X,Y\qq \text{Banachräume}
\]
und zum Abschluß zeigen wir mit Hilfe des \textit{Schauder'schen Fixpunktsatzes} die Existenz
und finden eine Lösung von (\ref{2}), also der schwachen Form der stationären 
Navier-Stokes-Gleichung.

Im zweiten Teil der Vorlesung beschäftigen wir uns mit Variationsrechnung (d.h. dem Finden
von Minimierern nichtlinearer Funktionalen)
\[
    W:X\ra \R \qq\text{mit} \qq X \qq \text{ein Banachraum}
\]
Finde 
\[
    x_0\in X: \, W(x_0)= \inf_{y\in X}W(y)
\]
Insbesonder treffen wir dort auf Probleme in der Elastizitätstheorie.

\section*{Aufbau der Vorlesung}
\begin{itemize}
    \item Abbildungsgrad $\ra$ Existenz von Lösungen von $Fx=y$
    \item Monotone Operatoren $\ra$ Eindeutigkeit von Lösungen von $Fx=y$; 
    	zeitabhängige Probleme.
    \item Variationsrechnung $\ra$ $\inf_{y\in X} W(y)$
\end{itemize}

\section{Vorarbeiten}

\subsection{Ableitung in Banachräume und implizite Funktionen}

Es seien $X$ und $Y$ Banachräume, $\Omega\subset X$ offen, $F:\Omega \ra Y$, $x_0\in \Omega$

\begin{defi}[Gâteaux-Ableitung]
    Die Gâteaux-Ableitung $\d F(x_0,\psi)$ des Operators $F$ im Punkt $x_0$ in Richtung $\psi\in X$ 
    ist gegeben durch 
    \[
        \d F (x_0,\psi)=\lim_{s\ra 0} \frac{F(x_0+s\cdot \psi)-F(x_0)}{s}= \left. \frac{\d}{\d s} 
        F(x_0+s\psi)\right|_{s=0}
    \]
    falls der Limes existiert. Der Operator $F$ heißt in diesem Fall in $x_0$ Richtung $\psi$ 
    Gâteaux-differenzierbar.
\end{defi}
\begin{defi}[Fréchet-Ableitung]
    Der Operator $F$ heißt Fréchet-differenzierbar in $x_0\in \Omega$, falls ein beschränkter,
    linearer Operator 
    \[
        F'(x_0):X\ra Y 
    \]
    existiert, so dass 
    \begin{align}\label{3}
        \lim_{\|h\|\ra 0} \frac{\| F(x_0+h) - F(x_0) -F'(x_0)\cdot h \|  }{\|h\|} = 0
    \end{align}
    $F'(x_0)$ heißt dann Fréchet-Ableitung von $F$ in $x_0$.
\end{defi}

\begin{theorem}
    \begin{description}
        \item{i)} $F'(x_0)$ ist durch (\ref{3}) eindeutig bestimmt.
        \item{ii)} Falls $F$ stetig ist in $x_0$, so ist jeder lineare Operator, der (\ref{3}) erfüllt,
                ebenfalls stetig.
        \item{iii)} Ist $L:X\ra Y$ linear, so gilt
                \[
                    L'(x)=L \qq \forall x\in X
                \]
    \end{description}
\end{theorem}

\begin{proof}

\begin{description}

\item{i)} 
Es gelte (\ref{3}) auch für L. Dann haben wir
\begin{align*}
    \|Lh-F'(x_0)h\| \leq \eps \|h\| \qq \text{falls} \qq \|h\| < \delta = \delta(\eps, x_0)
\end{align*}
Für beliebiges $h$ folgt aber
\begin{align*}
    \|(L-F'(x_0)) (\delta\|h\|^{-1}\cdot h)\|&\leq \delta\eps\\
    \Ra \, \|(L-F'(x_0))h  \|&\leq \eps \|h\| \qq \forall \, h\in X, \, \forall \, \eps>0 \\
    \Ra \, \|L-F'(x_0)\|_{\ms L(X,Y)} &= 0
\end{align*}

\item{ii)}
(\ref{3}) wird umgeformt zu
\[
    \|F'(x_0) h \| \leq \eps \|h\| + \|F(x_0+h)-F(x_0)\|
\]
Mit $h\ra 0$ folgt die Stetigkeit (für $\|h\|\leq \delta$) von $F'(x_0)$ an der Stelle 0. 
Wegen Linearität von $F'(x_0)$ ist $F'(x_0)$ somit stetig. 

\item{iii)} (\ref{3}) gilt offensichtlich für $L'(x_0)=L$, mit i) folgt Eindeutigkeit.
\end{description}
\[ \]
\end{proof}


\begin{prop}
    Jeder Fréchet-differenzierbare Operator $F$ ist Gâteaux-differenziebar $\forall \, \psi\in X$ 
    und es gilt
    \[
        F'(x_0)\psi=\d F(x_0,\psi)
    \]
\end{prop}
\begin{proof}
    Übungsaufgabe \[\]
\end{proof}

\begin{defi}
    $F$ heißt (Fréchet-)differenzierbar auf $\Omega$, falls $\forall\, x\in X$ ein $F'(x)$ existiert,
    sodass $F'(x)$ stetig ist  und (\ref{3}) erfüllt. $F$ heißt stetig (Fréchet-)differenzierbar in
    $\Omega$, falls die Abbildung
    \[
        F': \Omega\ra \ms L(X,Y)
    \]
    stetig ist.
\end{defi}

\begin{prop}
    Existiert die Gâteaux-Ableitung $\d F(x,\psi) \, \forall x\in \Omega$, und ist sie linear und
    stetig in $\psi$ $\forall\, x\in \Omega$, so ist $F$ Fréchet-differenziebar auf $\Omega$ und
    es gilt 
    \[
        F'(x)\psi=\d F(x,\psi)
    \]
\end{prop}

\begin{proof} 
    Übungsaufgabe \[\]
\end{proof}

\begin{defi}
    Sei $F$ auf $\Omega$ stetig differenziebar, $x_0\in \Omega$. Falls ein stetiger linearer Operator
    \[
        F''(x_0):X\ra \ms L(X,Y)
    \]
    existiert mit
    \[
        \lim_{\|h\|\ra 0}\frac{\|F'(x_0+h)-F'(x_0)-F''(x_0)h\|_{\ms L(X,Y)}}{\|h\|}=0
    \]
    dann heißt $F$ in $x_0$ zweimal (Fréchet-)differenzierbar und $F''(x_0)$ heißt zweite Ableitung
    von $F$ in $x_0$. Höhere Ableitungen entsprechend.
\end{defi}

\begin{remark}
    Es gilt die Kettenregel:
    Seien $X$, $Y$, $Z$ Banachräume, $\Omega_X\subset X$ offen, $x_0\in \Omega_X$,
    \[
        F:\Omega_X\ra Y, \qq F(x_0)=y_0\in \Omega_Y\subset Y \qq \text{offen}
    \]
    \[
        G:\Omega_Y\ra Z
    \]
    Falls $F'(x_0)$ und $G'(y_0)$ existiert, so ist 
    \[
    (G\circ F)'(x_0)=G'(y_0)\circ F'(x_0)
    \]
\end{remark}

\begin{defi}[Partielle Ableitung]
    Seien $X,$ $Y$, $Z$ Banachräume, $\Omega_X\subset X$ offen, $x_0\in \Omega_X$, $\Omega_Y\subset Y$ 
    offen, $y_0\in \Omega_Y$. Der Operator
    \[
        F: \Omega_X \times \Omega_Y \ra Z 
    \]
    heißt partiell in $(x_0,y_0)$ nach dem zweiten Argument (nach $y$) differenzierbar, falls die 
    Abbildung 
    \[
        F(x_0, \cdot) : \Omega_Y \ra Z 
    \]
    differenziebar ist. Wir nennen den linearen Operator $F_Y(x_0,y_0): Y\ra Z$, der
    \[
        \lim_{\|h\|\ra 0} \frac{\| F(x_0,y_0+h) - F(x_0,y_0) - F_Y( x_0,y_0)h \|}{\|h\|} = 0 
    \]
    erfüllt, die partielle Ableitung von $F$ in $(x_0,y_0)$ nach dem zweiten Argument.
\end{defi}

\begin{prop}\label{1.10}
    Seien $X,Y$ Banachräume, $\Omega \subset X $ offen und konvex mit $x_0, x_1\in \Omega$. 
    $F: \Omega \ra Y$ sei stetig Fréchet-differenzierbar auf $\Omega$. Dann gilt
    \[
        F(x_1)-F(x_0)=\int_{0}^1 F'(x_0+t(x_1-x_0))(x_1-x_0)\d t
    \]
    Das Integral ist als Limes der entsprechenden Riemannsumme zu verstehen und dieser existiert.
\end{prop}

\begin{proof}
    Übungsaufgabe \[\]
\end{proof}

Ähnlich dem endlichdimensionalen Fall geben uns die Ableitungen im Banachraum
hinreichende Bedingungen um Operatoren implizit zu definieren. Die Fragestellung ist die folgende:
Seien $X,\, Y,\, Z$ Banachräume, $U$ eine Umgebung von $x_0\in X$. $V$ eine Umgebung von $y_0\in Y$.
Wir suchen zu $F:U\times V \ra Z$ einen Operator
\[
    T:U_0\subset U \ra V, 
\]
sodass gilt
\[
    F(x,Tx)=F(x_0,y_0) \qq \forall \, x\in U_0.
\]
Durch eine einfache Verschiebung ist es ausreichend, den Fall
\[
    F(x_0,y_0)=0
\]
zu untersuchen.

\begin{prop}\label{1.11}
    Sei $X$ ein Banachraum, $\Id: X\ra X$, $x\mapsto x$ die Identität auf $X$. Es sei
    \[
        R:B_r(0)\subset X \ra X, 
    \]
    eine $k$-Kontraktion mit $k<1$, d.h. $\|R(x)-R(y)\|\leq k \|x-y\|$, und es gelte
    \[
        \|R(0)\| < r(1-k)
    \]
    Dann existiert genau ein $x\in B_r(0)$ mit
    \[
        (\Id + R)x = 0
    \]
\end{prop}

\begin{proof}
    Sei $S=-R$, wir suchen also einen Fixpunkt von $S$.
    \begin{description}
        \item{1. Eindeutigkeit:} 
        Seien $Sx=x$ und $Sx'=x'$, damit gilt
        \[
            \|x-x'\|=\|Sx-Sx'\| \leq k \|x-x'\|.
        \]
        Mit $k<1$ folgt $x=x'$.
        
        \item{2. Existenz:}
        Sei $x\in B_r(0)$, es gilt
        \[
            \|Sx\|\leq \|Sx-S(0)\|+\|S(0)\|\leq k\|x\|+\|S(0)\|< kr + r(1-k) = r
        \]
        Sei $x_p=Sx_{p-1}$, $x_0=0$. Es gilt (wie auch im Banach'schen Fixpunktsatz, siehe ÜB 1), dass
        \[
            \|x_{n+p}-x_n\| \leq k^n(1-k)^{-1}\|x\|,
        \]
        damit ist $(x_p)_{p\in \N}$ eine Cauchy-Folge und konvergiert gegen $x\in X$.
        Wir haben weiter, dass
        \[
            \|x\| \leq \underbrace{\|x-x_{p+1}\|}_{\ra 0} + \|x_{p+1}\| \qq \text{mit} 
            \qq \|x_{p+1}\|< (1-k) \|S(0)\|=r\\
            \Ra \, \|x\|<r
        \]
        Wegen $x_{p+1}=Sx_p$ gilt dass $x=Sx$, somit ist $x$ der gesuchte Fixpunkt. 
    \end{description}
    \[  \]
\end{proof}

\begin{theorem}[Satz über implizite Funktion]\label{1.12}
    Seien $X$, $Y$, $Z$ Banachräume, $U\subset X$ Umgebung von $x_0\in X$, $V\in Y$ Umgebung von 
    $y_0 \in Y$. Sei weiter
    \[F:U\times V\ra Z\]
    stetig und stetig differenzierbar nach der zweiten Variablen. $F_Y(x_0,y_0)$ sei eine Bijektion
    von $Y$ nach $Z$ und es gelte
    \[
        F(x_0,y_0)=0
    \]
    Dann existiert $B_\delta(x_0)\subset U$, $B_r(y_0)\subset V$ und genau ein Operator 
    $T:B_\delta(x_0)\ra B_r(y_0)$, so dass $T(x_0)=y_0$ und $F(x,Tx)=0$ $\forall \, x\in B_\delta(x_0)$.
    $T$ ist stetig.
\end{theorem}

\begin{proof}
    Ohne Einschränkung sei $x_0=y_0=0$. Sei $L:= F_Y(0,0)$, $\Id: Y\ra Y$ die Identität auf $Y$. Es 
    sei $S(x,y):= L^{-1}F(x,y)-y$ für $(x,y)\in U\times V$. Somit gilt
    \[
        F(x,y)=0 \qq \LRa \qq y+S(x,y)=0.
    \]
    $S$ ist stetig differenzierbar nach dem zweiten Argument mit
    \[
        S_Y=L^{-1} F_Y(x,y)-\Id
    \]
    Damit gilt
    \[
        S_Y(0,0)=0
    \]
    Sei $k\in (0,1)$. Wegen Stetigkeit von $S_Y$ existiert $r>0$ mit 
    \[
        \|S_Y(x,y)\|\leq k \qq \forall (x,y)\in B_r(0)\times B_r(0)
    \]
    Sei nun $x\in B_r(0)$, $y, \tilde y \in B_r(0)$
    Es gilt nach Proposition \ref{1.10}, dass
    \[
        \|S(x,y)-S(x,\tilde y)\|=\left\|\int_0^1 S_Y(x,\tilde y +t(y-\tilde y)) (y-\tilde y) \d t 
        \right\|\leq k\cdot\|y-\tilde y\|
    \]
    Wegen $S(0,0)$ und Stetigkeit von $S$ existiert $\delta \leq r$, so dass
    \[
        \|S(x,0)\|\leq r(1-k) \qq \forall \, x \in B_\delta (0)
    \]
    Sei also $x\in B_\delta(0)$. Nach Proposition \ref{1.11} existiert genau ein $y\in B_r(0)$ mit
    $y+S(x,y)=0$. Wir setzen
    \[
        Tx=y, \qq T:B_\delta(0)\ra B_r(0)
    \]
    Es gilt $T(0)=0$ wegen $0+S(0,0)=T(0)+S(0,T(0))=0$ und der Eindeutigkeit von $T$. Es bleibt die
    Stetigkeit von $T$ zu zeigen: Seien $x,x'\in B_\delta (0)$, damit gilt
    \[
        0=Tx+S(x,Tx)=T(x'+S(x,Tx'))
    \]
    also
    \begin{align*}
        \|Tx-Tx'\|&\leq \|S(x',Tx')-S(x,Tx')\|+\|S(x,Tx)-S(x,Tx)\|\\
            &\leq \|S(x',Tx') - S(x,Tx')\|+k\|Tx-Tx'\|\\
            &= (1-k)\|Tx-Tx'\|\\
            &\leq \|S(x',Tx')-S(x,Tx')\| \ra 0 \qq \text{für $x\ra x'$} 
    \end{align*}
    Somit ist $T$ stetig.
    \[\]
\end{proof}

\begin{remark}
    Ist $F$ $r$-mal stetig differenziebrar, so gilt das auch für $T$.
\end{remark}

\begin{proof}
    Übungsaufgabe \[ \]
\end{proof}

\begin{theorem}\label{1.13}
    Seien $X$, $Y$ Banachräume, $U\subset X$ eine Umgebung von $x_0$. Es sei $F:U\ra Y$ stetig 
    differenzierbar und $F'(x_0)$ sei eine lineare Bijektion von $X$ nach $Y$.
    Dann existiert eine Umgebung $U_0\subset U$ von $x_0$, so dass
    \[
        F|_{U_0}: U_0\ra F(U_0)\ni y_0 -F(x_0)
    \]
    ein Homöomorphismus (bistetige Abbildung) ist.
\end{theorem}

\begin{proof}
    Wir wenden Satz $\ref{1.12}$ auf
    \[
        \tilde F(x,y):= F(x)-y
    \]
    an.
    \[  \]
\end{proof}

\begin{remark}
    Ist $F$ $r$-mal stetig differenzierbar, so gilt das auch für $F^{-1}$ ($F$ ist ein 
    $r$-Diffeomorphismus).
\end{remark}

\begin{proof}
    Übungsaufgabe\[ \]
\end{proof}

\begin{defi}[Zusammenhände Mengen]\label{1.14}
    \begin{description}
    \item{-}
    Sei $X$ ein (topologischer metrischer, normierter) Raum. Eine Menge $\Omega \subset X$ heißt
    zusammenhängend, falls es keine zwei abgeschlossenen (offenen) $\Omega_1$, $\Omega_2$ gibt
    mit
    \[
        \Omega\subset\Omega_1\cup\Omega_2, \qq \Omega\cap \Omega_1 \cap \Omega_2 = \varnothing, \qq
        \Omega\cap\Omega_{1,2}\neq \varnothing
    \]
    \item{-}
    Eine Menge $\Omega\subset X$ heißt wegzusammenhängend, falls sich je zwei Punkte in $\Omega$ durch
    eine stetige, in $\Omega$ verlaufende Kurve verbinden lassen.
    \item{-}
    Eine Menge $\ol\Omega \subset \Omega$ heißt Zusammenhangskomponente von $\Omega$, falls
    $\ol \Omega \subset \Omega$ maximal, zusammenhängend.
    \end{description}
\end{defi}

\begin{remark}
    Wegzusammenhängend $\Ra$ Zusammenhängend.

    \noindent Offen, zusammenhängend $\Ra$ Wegzusamenhängend
\end{remark}

\begin{theorem}[Mittelwert]\label{1.15}
    Seien $X$, $Y$ Banachräume, $F:X\ra Y$ stetig differenzierbar.
    \begin{description}
    \item{i)}
    Falls $\Omega$ konvex ist, so gilt
    \[
        \|F(x)-F(y)\|\leq M\|x-y\|,
    \]
    wobei
    \[
        M=\max_{0\leq t \leq 1}\| F'((1-t)x + ty) \|
    \]
    \item{ii)}
    Umgekehrt gilt: Falls
    \[
        \| F(x)-F(y) \|\leq M\|x-y\| \qq \forall \, x,y\in \Omega
    \]
    Dann gilt
    \[
        \sup_{x\in \Omega}\|F'(x)\|\leq M
    \]
    \end{description}
\end{theorem}

\begin{proof}
    Sei $f(t):= F((1-t)x+ty)$, $0\leq t\leq1$. Nach Kettenregel gilt
    \[
        f'(t)=F'((1-t)x+ty)(x-y)  
    \]
    \[
        \Ra \, \|f'(x)\|\leq\tilde M:= M\|x-y\|  
    \]
    \begin{description}
    \item{i)}
    Sei $\phi(t):=\|f(t)\|$ für $\delta>0$. Wir wollen zeigen, dass $\phi(t)\leq0$ $\forall \, 
    \delta>0$, $0\leq t\leq1$. Sei also (zum Widerspruch)
    \[
        t_0:=\max\{t\in [0,1]\, | \, \phi(s)\leq 0 \, \forall \, s\leq t\}.
    \]
    Dann gilt
    \begin{align*}
        \phi(t_0+\eps)&=\|f(t_0+\eps)-f(t_0)+f(t_0)-f(0)\|-(\tilde M + \delta)t\\
        &\leq \|f(t_0+\eps)-f(t_0)\|-(\tilde M +\delta) -\phi(t_0)\\
        &\leq \|f'(t_0)\eps+ \mc o (1)\|-(\tilde M +\delta)\eps\\
        &\leq (-\delta + \mc o (1))\eps 
    \end{align*}
    \item{ii)}
    Angenommen, es existiert $x_0$ mit $\|F'(x_0)\|\geq M+2\delta$, $\delta>0$. Dann existiert
    $e\in X$, $\|e\|=1$, $\| F'(x_0)e \|\geq M+\delta$. Somit gilt
    \begin{align*}
        M\eps&\geq \|F(x_0+\eps e) - F(x_0)\|= \|F'(x_0)(\eps e)+\mc o (\eps)\|\\
         &\geq (M+\delta)\eps- \mc o(\eps) > M\eps
    \end{align*}
    Das ist ein Widerspruch.
    \end{description}
    \[\]
\end{proof}

\begin{cor}\label{1.16}
    Sei $\Omega \subset X$ offen, (weg-)zusammenhängend, $F$ stetig differenzierbar auf $\Omega$.
    Es gilt
    \[
        F= \Const \qq \LRa \qq F'=0
    \]
\end{cor}
\begin{remark}
    Wir schreiben wie im endl. dim. $C(\Omega)= C^0(\Omega), \, C^1(\Omega)\, …$
\end{remark}

\noindent \textbf{Anwendungen}: Lokale Existenz und Eindeutigkeit Banachraum-wertiger
Differentialgleichungen. Sei $X$ Banachraum, $\Omega\subset X$ offen, $I\subset \R$
kompaktes Intervall. Es sei $C_b(I,\Omega)$ der Banachraum der beschränkten, stetigen Abbildungen
von $I$ nach $\Omega$, versehen mit der $\sup$-Norm.

\begin{lem}
    Sei $Y$ ein Banachraum, $f\in C(\Omega,Y)$ und sei die Funktion
    \[
        f_\star: C_b(I,\Omega)\ra C_b(I,Y)
    \]
    definiert als
    \[
        (f_\star x)(t)= f(x(t))
    \]
    Es gilt $f_\star\in C^r$
\end{lem}

\begin{proof}
    \begin{description}
    \item{$r=0$:}
    Sei $x_0\in C_b(I,\Omega)$, $\eps>0$ $\forall \, t \in I$ existiert $\delta(t)>0$, so dass
    \[
        \|f(\xi)-f(x_0(t))\|\leq \frac{\eps}{2} \qq \forall \, \xi\in B_{2\delta(t)}(x_0(t))
    \]
    Die offenen Kugeln 
    \[
        \{ B_{\delta(t)}(x_0(t)) \}_{t\in I}
    \]
    sind eine offene Überdeckung vom $\{ x_0(t) \}_{t\in I}$. Diese Menge ist als stetiges Bild einer
    kompakten Menge kompakt, und somit existiert endliche Teilüberdeckung
    \[
        \{ B_{\delta(t_j)}(x_0(t_j)) \}_{1\leq j \leq N}
    \]
    Sei nun $x\in C_b(I,\Omega)$ mit
    \[
        \|x-x_0\|\leq \delta := \min_{1\leq j \leq N}\delta(t_j)
    \]
    Somit existiert $\forall \, t\in I$ ein $t_j$, so dass $\|x_0(t)-x_0(t_j)\|< \delta(t_j)$,
    und deshalb gilt
    \[
        \|f(x(t))- f(x_0(t))\|\leq \underbrace{\|f(x(t))-f(x_0(t_j))\|}_{\leq 2\delta}
        +\underbrace{\|f(x_0(t_j))-f(x_0(t))\|}_{\leq\delta},
    \]
    denn
    \[
        \|x(t)-x_0(t_j)\|\leq \|x(t)-x_0(t)\|+\|x_0(t)-x_0(t_j)\|\leq 2\delta(t_j).
    \]
    Somit folgt die Steigkeit……
    \item{$r=1$:}
    Wir müssen zeigen, dass
    \[
        \sup_{t\in I} \|f(x_0(t)+x(t))- f(x_0(t)) - f'(x_0(t))x(t)\|\leq \eps \sup_{t\in I}\|x(t)\|
    \]
    denn
    \[
        (f'_\star(x_0)x)(t)= f'(x_0(t))x(t)
    \]
    Übungsaufgabe. Folgt wie Stetigkeit durch Kompaktheit von $I$.
    \end{description}
    \[  \]
\end{proof}

\section{Der Broawer'sche Abbildungsgrad}

\subsection*{Motivation}
\begin{description}
    \item[Ziel:]
    $f(x)=0$ zu lösen für $f:U\subset X \ra X$, $X$ Banachraum.
    \item[Frage:]
    Existenz/Anzahl der Lösungen
    \item[Rückblick auf Funktionentheorie:]
    Sei $z_0\in \C$
    \[
        n(\gamma,z_0) = \frac{1}{2\pi i} \int_\gamma \frac1{z-z_0} \d z
    \]
\end{description}
$\leadsto$ Verallg.: $f\in \mc H(\C)$. $0\nin f(\gamma)$
\[
    n(f(\gamma),0)=\frac1{2\pi i} \int_\gamma \frac{f'}{f} \d z = \sum_k n(\gamma,z_k) \alpha_k
\]
wobei $f(z_n)=0$, $\alpha_k$ Vielfachheiten.

\noindent Ziel: Verallg. des Begriffs „Umlaufzahl“ für Abb. $f:U\subset \R^n\ra \R^n$

\subsection{Die Determinantenformel}

\subsubsection{Notation}

$U\subset\subset R^n$ offen,
\begin{align*}
    J_f(x)&=\det \partial f(x)\\
    RV(f)&=\{ y\in \R^n \,| \, \forall \, x\in f^{-1}(y), \, J_f(x)\neq 0 \}\\
    CV(f)&= \R^n \setminus RV(f)\\
    D_y^r(\ol U, \R^n)&:=\{f\in C^k(\ol U, \R^n)\,|\,y\nin f(\partial U)\}\\
    D_y(\ol u, \R^n)&:=D_y^0(\ol U, \R^n)
\end{align*}
$\tau(R^n)$ bezeichne die Topologie auf $\R^n$.
\begin{defi}
    Eine Abbildung
    \[
        \deg: \bigcup_{U\in \tau(\R^n),\, y\in \R^n} (D_y(\ol U, \R^n)\times \{U\}\times \{y\}\ra \R,
    \]
    d.h.
    \[
        \deg=\deg(f,U,y) 
    \]
    heißt Gradabbildung, falls
    \begin{description}
        \item[D1]
        $\deg(f,U,y)=\deg(f-y,U,0)$
        \item[D2]
        $\deg(\Id,U,y)=1 \qq \forall y\in U$
        \item[D3]
        Seien $U_1,U_2\subset U$ offen und disjunkt, sodass $y\nin f(\ol U\,|\,(U_1\cup U_2))$, dann
        gelte
        \[
            \deg(f,U,y)=\deg(f,U_1,y)+\deg(f,U_2,y)
        \]
        \item[D4]
        $H(t)=(1-t)f+tg \in D_y(\ol U,\R^n) \qq \forall \, t\in [0,1]\, \Ra \, \deg(f,U,y)=\deg(g,U,y)$
        (Homotopieinvarianz)
\end{description}
\end{defi}

\begin{theorem}
    Sei $\deg$ eine Gradabbildung. Dann gilt
    \begin{description}
        \item{i)}
        $\deg(f,\varnothing,y)=0$ und
        \[
            \deg(f,U,y)=\sum_{i=1}^N \deg(f,U_i,y)
        \]
        falls $y\nin f(\ol U\setminus\bigcup_{i=1}^NU_i)$, $U_i\subset U$ offen und disjunkt.
        \item{ii)}
        $y\nin f(U) \, \Ra \, \deg(f,U,y)=0$
        \item{iii)}
        $|f(x)-g(x)|<\dist(y,f(\partial U)) \qq \forall \, x\in \partial U \qq \ra \qq \deg(f,u,y)
        =\deg(g,U,y)$
    \end{description}
\end{theorem}

\begin{proof}
    \begin{description}
    \item{i)}
    Sei $U_1=U$, $U_2=\varnothing$, einsetzen in (\textbf{D3})
    \[
        \Ra \, \deg(f,\varnothing,y)=0
    \]
    $i=1$: $U_2=\varnothing$
    \[
        \Ra \, \deg(f,U,y)=\deg(f,U_1,y)
    \]
    $i>1:$ Induktion mittels (\textbf{D3})
    \item{ii)}
    \begin{align*}
        y\nin f(U) \ra y \nin &=(\ol U)\, \Ra \, y\nin f(\ol U\setminus \varnothing)\\
        \overset{(i)}{\Ra } \, \deg (f,U,y)&=0 \qq (i=1, U_1=\varnothing)
    \end{align*}
    \item{iii)}
    Sei $H(t,x):=(1-t)f(x)+tg(x)$ und sei $x\in \partial U$
    \begin{align*}
        \Ra \, |H(t,x)-y|&= |f(x)-y+t(g(x)-f(x))|\\
            &\geq |f(x)-y| - |g(x)-f(x)|\\
            &\geq \dist(y,f(\partial U))- |g(x)-f(x)| >0\\
        \Ra \, y&\nin H(t,\partial U) \qq \forall \, t\,\Ra \,H(t)\in D_y(\ol U,\R^n)\\
        &\overset{(\textbf{D4})}{\Ra} \text{ Behauptung.}
    \end{align*}
    \end{description}
\end{proof}

\begin{theorem}
    \begin{description}
        \item{i)} $\deg(\cdot,U,y)$ ist lokal konstant in $D_y(\ol U,\R^n)$
        \item{ii)} $\deg(f,U,\cdot)$ ist lokal konstant in $\R^n\setminus f(\partial U)$
        \item{iii)} Seien $H:[0,1]\times \ol U\ra \R^n$ und $y:[0,1]\ra \R^n$ stetig (d.h. $H$ ist eine
            Homotopie zwischen $H(0)=H(0,\cdot)$ und $H(1)$), so gilt
        \[
            \deg(H(0),U,y(0))=\deg(H(1),U,y(1)),
        \]
        falls $H(t)\in D_{y(t)}(\ol U, \R^n)$ $\forall t \in [0,1]$
    \end{description}
\end{theorem}

\begin{proof}
    Beachte: $D_y(\ol U, \R^n)$ ist offen in $C^0(\ol U, \R^n)$
    \begin{description}
    \item{i)}
        $\|f-g\|_{C^0,\ol U} <\eps \, \Ra \, |f(x)-g(x)|<\eps \, \forall \, x,\partial U$ mit
        \[
            \eps:= \dist(y,f(\partial U)) \, \Ra \, \deg (f,U,y)=\deg (g,U,y)
        \]
    \item{ii)}
        Sei $y_0\nin f(\partial U)$ und $y\in B_{\dist(y_0,f(\partial U))}(y_0\subset \R^n\setminus
                f(\partial U))$
        \begin{eqnarray*}
            \Ra \, \|(f-y)-f\|&<\dist(y_0,f(\partial U))\\
                \overset{i)}{\Ra} \, \deg(f-y,U,y_0) &= \deg(f,U,y_0)\\
            \overset{\textbf{D1}}{\Ra } \, \deg (f,U,y_0+y)&= \deg(f,U,y_0)
        \end{eqnarray*}
    \item{iii)}
        $H$ ist gleichmäßig stetig
        \[
            \Ra \, H:[0,1]\ra C^0(\ol U,\R^n)\qq t\mapsto H(t,\cdot)
        \]
        ist auch stetig. $H$ ist ein stetiger Weg in $D_y(\ol U,\R^n)$. Sei $y$ fest
        \[
            \Ra \, \deg(H(t,U,y))=\Const
        \]
        weil $\deg(\cdot,U,y)$ konst. auf Zsh-Komponenten ist. Für $y=y(t):$
        \begin{align*}
            \deg(H(0),U,y(0))&=\deg(H(0)-y(0),U,0)=\deg(H(t)-y(t), U, 0) \qq \forall t\\
            &=
        \end{align*}
    \end{description}
\end{proof}

\begin{lem}\label{1.3.5}
    Zwei Matrizen $M_1,M_2\in \Gl(n)$ sind genau dann homotop in $\Gl(n)$, falls
    \[
        \sign \det M_1 = \sign\det M_2
    \]
\end{lem}

\begin{proof}
    \begin{description}
    \item{„$\Ra$“} Sei $M\in \Gl(n)$. Wegen der Linearität von $\det $ in Zeilen können elementare
    Zeilenumformungen mit Hilfe stetiger Deformationen in $\Gl(n)$ erzeugt werden.
    \[
        M \, \leadsto\diag(m_1,…,m_m), \qq \text{mit} \qq |m_i|=1
    \]
    weil $\sign \det M_1=\sign \det M_2$.
    \[
        H(t):= \begin{pmatrix} \pm \cos (\pi t)& \mp \sin(\pi t)\\ \sin(\pi t) & \cos(\pi t)\end{pmatrix}
    \]
    ist eine Homotopie in $\Gl(n)$ von $\diag(\pm 1,1)$ nach $\diag(\mp 1,-1)$. $i=n$ transformiere
    $\diag(\mp 1,-1)$.

    \noindent $i=n$ transformiere $\diag(m_i,…,m_i)$ nach $\diag(\pm1,1)$

    \[
        \leadsto \begin{pmatrix} \sign\det M& 0& \cdots&0\\ 0& 1 &\cdots&0\\\vdots&&\ddots&0\\ 0&\cdots&
        \cdots& 1\end{pmatrix}
    \]

    \end{description}
\end{proof}

\begin{theorem}
    Sei $f\in D_y^1(\ol U, \R^n)$, $y\nin CV(f)$ und $\deg$ eine Gradabbildung. Dann gilt
    \[
        \deg(f,U,y)=\sum_{x\in f^{-1}(y)} \sign J_f(x),
    \]
    wobei die Summe endlich ist.
\end{theorem}

\begin{proof}
    O.B.d.A. $y_0$ (\textbf{D1}). Alle $x\in f^{-1} (0)$ sind isolierte Punkte in $U$ (Homöomorphiesatz).
    $f^{-1}(y)$ hat höchstens am Rand einen Häufungspunkt, aber $0\nin f(\partial U)$.
    \[
        \Ra \, f^{-1}(0)= \{ x^i \}_{i=1}^N
    \]
    Wähle $\delta >0$ so klein, dass $B_\delta(x^i)$ paarw. disjunkt.
    \[
        \deg(f,U,0)=\sum_{i=1}^N \deg(f,B_\delta(x^i),0)
    \]
    beachte $0\nin f(\ol U, \setminus \bigcup_{i=1}^N B_\delta(x^i))$.
    \begin{align*}
        f(x)&= \partial f(x)(x-x') + |x-x^i| r(x-x') \qq \text{mit} \qq r\in C^0(B_\delta(x^i),\R^n), 
        \qq r(0)=0)\\
            &=
    \end{align*}
    Zeige $0\nin H(t,\partial B_\delta(x^i))$.
    \[
        J_f(x^i)\neq 0 \, \Ra \, \exists\, \lambda >0:\, |\partial f(x^i)(x-x^i)|\geq \lambda|x-x^i|
    \]
    O.B.d.A. sei $\delta$ so klein, dass $|r(x-x^i)|<\lambda$ in $B_\delta (x^i)$.
    \[
        \Ra\, | H(t,x) |>|\partial f(x^i)(x-x^i)|-(1-t)(x-x^i)r(x-x^i)\geq \lambda \delta - \delta|r|>0
        \qq \forall \, x\in \partial B_\delta (x^i)
    \]
    \[
        \overset{(\textbf{D4})}{\Ra} \, \deg(f,U,0)=\sum_{i=1}^N\deg (\partial f (x^i)(\cdot-x^i),
                B_\delta(x^i),0
    \]
    Lemma \ref{1.3.5}
    \[
        \Ra \, \deg(\partial f (x^i)(\cdot- x^i),B_\delta (x^i),0)= \deg(\diag(\sign J_f(x^i),1,…,1),
            B_\delta(x^i),0)
    \]
    Falls $J_f(x^i)>0\, \overset{\textbf{D2}}{\Ra}\,\deg(I(\cdot-x^i),B_\delta (x^i),0)=1$. Es genüngt
    also, $\deg(M(\cdot-x^i),B_1(x^i),0)$. Zu berechnen, wobei $M=\diag(-1,1,…,1)$ ist.
    \begin{align*}
        U_1&:= B_1 (x^i) = \{ \max_{1\leq k\leq n} |x_k-x_k^i|<1 \}\\
        U_2&:= U_1+ (2,0,\cdot,0)\\
        g(r)&=2-|r-1|, \, h(r)=1-r^2\\
        f_1(x)&:= (1-g(x_1-x_1^i)h(x_2-x_2^(i))…h(x_n-x_n^{(i)}),…, 1)\\
        f_2(x)&:= (1,x_2-x_2^{(i)}, … , x_n-x_n^{(i)})\\
        f_1^{-1}(0)&= \{ y,z \} \qq y=x^i, z = x^i+(2,0,…,0)\\
        f_1|_{\partial U}&= f_2|_{\partial U} \qq \Ra \qq \deg(f_1,U_2,0)=0\\
        \Ra \, \deg (f_1,U,0)&= \deg(f_1,U_1,0)+ \deg(f_2,U_2,0) \qq (\star)\\
        \Ra \, \deg(M,B_1(x^i),0)&=\deg(\partial f_1(y),B_1(x^i), 0)\\
        &=\deg(f_1, U_1,0) \overset{(\star)}{=} -\deg (f_1,U_2,0)\\
        &= -\deg(\partial f_1(z),U_2,0)=\deg(\Id, U_2,0)=1
    \end{align*}
    \[ \]
\end{proof}

\subsection{Verallgemeinerung der Determinantenformel}

Wir haben einen Kandidaten für $\deg$ identifiziert; für $f\in D^1_y(\ol\Omega,\R^n)$, $y\in \R^n\cap
\RV(f)$ gilt
\[
    \deg(f,\Omega, g)=\sum_{x_j\in f^{-1}(y)} \sign J_f(x_j).
\]

\noindent \textbf{Problem:} Nur für glatte $f$, und reguläre Punkte zu definieren. Programm zum 
Existenzbeweis: Verallgemeinerung der Determinanten-Formel auf
\begin{description}
    \item{-} kirtische Werte $y$.
    \item{-} nur stetige Funktionen $f$.
\end{description}

\noindent Drei Beispiele zur Illustration:
\begin{enumerate}
    \item $f(x)=x^2$, $U=(-1,1)$
    \item $f(x)=x^2$, $U=(-1,2)$
    \item $f(x)=x+2\sin(x)$, $U=(-10,10)$
\end{enumerate}
Wie aus den Beispielen ersichtlich ist, haben wir noch (kleine) Schwierigkeiten, den Abbildungsgrad
an kritischen Werten von $f$ zu definieren. Stetige Fortsetzungen liegt aber nahe. Das funktioniert
aber nur falls es von diesen nicht „zu viele“ gibt.

\subsubsection*{Schritt 1: Kritische Werte von $f$}

\begin{lem}[Sard]\label{2.5}

Es sei $f\in C^1(\Omega, \R^n)$, $\Omega\subset\R^n$ sei offen und beschränkt. Dann ist
$\CV(f)$ eine Lebesgue-Nullmenge.
\end{lem}
    
\begin{proof}
    Klar, falls $f$ eine affine Abbildung ist. (Dimension des Bildraumes der (konstanten) Ableitung!)
    Wir linearisieren. 

    \noindent Es sei $\CP(f):=\{x\in \Omega\,|\, J_f(x)=0\}$ die Menge der kritischen Punkte von $f$.
    Sei $\{ Q_j \}_{j\in \N}$ eine abzählbare, offene Überdeckung von $\Omega$ bestehend aus Würfeln,
    so dass
    \[
        \ol Q_i\subset \Omega, \, i\in \N.
    \]
    Es gilt 
    \[
        \CV(f)=f(\CP(f))=\bigcup_{j\in \N} f(\CP(f)\cap Q_j)
    \]
    Es reicht also zu zeigen, dass
    \[
    |f(\CP(f)\cap Q_j)|
    \]
    für alle $j\in \N$ verschwindet. Sei nun $Q$ ein solcher Würfel, $\rho$ dessen Kantenlänge.
    Sei $\eps >0$, und sei $Q$ unterteilt in $N^n$ Würfel $Q^i$ der Seitenlänge $\frac{\rho}{N}$, so
    dass
    \begin{align}\label{4}
        |f(x)-f(\tilde x)-f'(\tilde x)(x-\tilde x)|&\leq \int_0^1|f'(\tilde x+t(x-\tilde x))-
        f'(\tilde x)|\cdot|x-\tilde x|\d t\leq \frac{\eps\rho}{N} \qq \text{für alle} 
            \qq x,\tilde x \in Q^i
    \end{align}
    So ein $N$ existiert, nachdem $f'$ auf $Q$ gleichmäßig stetig ist. Nun enthalte $Q^i$ einen
    kirtischen Punkt $\tilde x_i \in \CP(f)$, ohne Einschränkung sei $\tilde x_i=0$, $f(\tilde x_i)=0$,
    und setzen $M=f'(\tilde x _i)$. Wegen $\det M =0$ existiert eine ONB $\{b^j\}_{j=1}^n$ mit
    \[
    b^n \bot \Bild(M)
    \]
    Weiter gilt 
    \begin{align*}
        Q^i&\subset \left\{ \sum_{j=1}^n \lambda _j b^j \, \big| \, \| \lambda \|_2\leq \sqrt{n} 
            \frac{\rho}{N} \right\}\subset\left\{\sum_{j=1}^n\lambda _jb^j \, \big|
            \, |\lambda_j|\leq \sqrt{n}\frac{\rho}{N} \, \forall 1\leq j\leq n\right\}
    \end{align*}
    Damit existiert $C>0$, (unabhängig von $i$), so dass
    \[
        MQ^i\subset \left\{\sum_{j=1}^{n-1}\lambda_jb^j\, \big| \, |\lambda_j|
            \leq C\cdot \frac{\rho}{N} \right\}
    \]
    mit $C=\sqrt{n} \max_{x\in \ol Q}|f'(x)| $. Damit gilt nach (\ref{4}) sogar, dass
    \[
        f(Q^i)\subset\left\{ \sum_{j=1}^n \lambda_jb^j\, \big| \, |\lambda_j|\leq 
        (C+\eps)\frac{\rho}{N} \, \forall 1\leq j\leq n-1, \, |\lambda_n|\leq \frac{\eps\rho}{N}\right\}
    \]
    Es folgt
    \[
        |f(Q^i)|\leq \frac{\tilde C\eps}{N^n},
    \]
    falls in $Q^i$ ein kritischer Punkt liegt. Es gibt maximal $N^n$ Unterwürfel $Q^i$ mit kritischen
    Punkten. Somit gilt
    \[
        |f(Q\cap \CP(f))|\leq C\cdot \eps
    \]
\end{proof}

\noindent Dank Lemma \ref{2.5} ist $\R^n\setminus \CV(f)$ dicht in $\R^n$. Das reicht leider (?)
noch nicht.

\noindent Wir brauchen $d_1=d_2$, um den Abbildungsgrad sinnvoll durch die Determinanten-Formel
definieren zu können. (denn $\deg$ soll konstant sein auf Zusammenhangskomponenten, unabhängig
von krit. Werten.)

\textbf{Idee:} Umschreiben der Determinantenformel als \textit{Integral}.
Es sei im Weiteren $\eta_\eps$ ein Standard-Mollifier, \text{d. h.}
\[
    \eta_\eps \in C^\infty_c(B_\eps(0)\subset \R^n), \qq \int_{\R^n} \eta_\eps=1,\qq \eta_\eps\geq 0
\]

\begin{lem}\label{2.6}
    Sei $f\in D^1_y(\ol \Omega, \R^n)$, $y\nin \CV(f)$. Dann gilt
    \begin{align}\label{5}
        \deg(f,\Omega,y)=\sum_{x_j\in f^{-1}(y)}\sign J_f(x_j)=\int_\Omega \eta_\eps (f(x)-y) J_f(x)\d x
    \end{align}
    für alle $\eps$ hinreichend klein, d. h.
    \[
        \eps_0>\eps>0, \qq \text{mit} \qq \eps_0=\eps_0(f,y)
    \]
    Es gilt $\supp (\eta_\eps(f(\cdot)-y))\subset \Omega$ für $\eps < \dist(y,f(\partial \Omega))$
\end{lem}

\begin{proof}
    \begin{description}
        \item{1)} 
        Falls $f^{-1}(y)=\varnothing$, dann sei $\eps_0=\dist(y,f(\partial \Omega))$
        \item{2)}
        Falls $f^{-1}(y):= \{x^i\}_{1\leq i \leq N}$, sei $\eps_0>0$, so dass
        \[
            f^{-1}(B_{\eps_0}(y)) = \bigcup_{i=1}^N U(x_i)=:\bigcup_{i=1}^N U_i
        \]
        mit $U_i\cap U_j=\varnothing$ für $i\neq j$.
        Aus dem \textit{Satz über die implizite Funktion} (war ja klar, dass wir den mal brauchen)
        folgt für evtl. noch kleiners $\eps_0$, dass
        \[
            f\big|_{U_i} \qq \text{bijektiv,} \qq J_f(x)\neq 0 \qq \forall x \in U_i.
        \]
        Wieder gilt
        \[
            \eta _{\eps}(f(x)-y)=0 \qq \forall x\in \ol \Omega \setminus \bigcup _{i=1}^N U_i
        \]
        Damit gilt
        \begin{align*}
            \int_\Omega \eta_\eps(f(x)-y) J_f(x) \d x&= \sum_{i=1}^N \sign J_f(x^i)\cdot
            \int_{B_{\eps_0}(x)}\eta_\eps (\tilde x ) \d \tilde x\\
            &=\sum_{x^i\in f^{-1(y)}} \sign  J_f(x^i)
        \end{align*}
        mit $\tilde x=f(x)-y$.
    \end{description}
    \[ \]
\end{proof}
Die Integraldarstellung ergibt auch für kritische Punkte Sinn. Aber:
Wegen der Abhängigkeit von $\eps_0$ von $f$ und $y$ ist auch der Wert des Integrals nicht
\textit{a priori} stetig in $f,y$.


Dieses Problem können wir beseitigen, wenn wir $f\in C^2$ fördern. Wir brauchen zunächst einen Hilfssatz.

\begin{prop}\label{2.7}
    \begin{description}
        \item{a)}
        Sei $u\in C^1(\R^n)$, $\supp u \subset K$, K kompakt. Sei $\vp=\div u$. Dann gilt
        \[
            \int_{\R^n} \vp (x) \d x=0
        \]
        \item{b)}
        Sei $\vp \in C^1(\R^n)$, und sei $z\in \R^n$. Dann gilt
        \[
            \vp (x+z)-\vp(x)=\div \left( z\cdot \int_0^1 \vp (z+tz) \d t\right)
        \]
        \item{c)}
        Seien $u$, $K$, $\vp$ wie in $a)$, $\Omega \subset \R^n$ offen, beschränkt.
        Sei $f\in C^2(\ol \Omega)$ mit $K\cap f(\partial \Omega)=\varnothing$. Dann existiert
        $v\in C^1(\ol \Omega)$, mit $\supp v \subset \Omega$, so dass
        \[
            \vp (f(x))J_f(x)=\div v(x) \qq \text{auf} \qq \Omega
        \]
    \end{description}
\end{prop}

\begin{proof}
    \begin{description}
    \item{a)}
    Klar!
    \item{b)}
    Sei
    \[
        \eta (x):= \int_0^1 \vp (x-sz) \d s.
    \]
    \begin{align*}
        (z\cdot \eta(x))&=\frac{\d}{\d t } \eta (x+tz)\big|_{t=0}=\int_0^1 \frac{\d}{\d t} \vp(x+sz+tz)
       \big|_{t=0} \d s\\
        &= \int_0^1 \frac{\d}{\d t} \vp(x+sz)\d s=\vp(x+z)-\vp(x)
    \end{align*}
    \item{c)}
    Es sei $\d _{ij}$ der $(i,j)$- Eintrag der Kofaktormatrix von $(f')_{ij}= \partial _if_j(x)$.
    Es sei $v_j(x)=\sum_{j=1}^n u_j(f(x))\d _{ij}(x),i=1,…,n$. Es gilt $K\cap f(\partial \Omega) = \Phi$,
    $f\in C(\ol \Omega)$, damit existiert $\delta>0$ mit $\dist(K,f(\ol \Omega \setminus 
                f(\Omega_\delta)))>0$.
    Somit gilt $\supp v\subset \Omega_\delta\subset \Omega$.
    Wir rechnen:
    \begin{align*}
        \partial_iv_i&= \sum_{\d _{ij}(x)\partial_ku_j(f(x))}\partial _i f_k.+\sum_{j=1}^nu_j(f(x))
        \partial_i \d_{ij}x
    \end{align*}
    \textit{Behauptung 1:}
    \[
        \sum_{i=1}^n \partial_i\d_{ij}(x)=0
    \]
    \textit{Behauptung 2:}
    \[
        \sum_{i=1}^n \d_{ij} (x) \partial_if_k(x) =\delta_{ij}\cdot J_f(x)
    \]
    Somit gilt:
    \begin{align*}
        \div u(x)&=\sum_{k,j}\partial _k u_j(f(x))\cdot \left( \sum_{i=1}^n \d_{ij}(x) \partial f_k(x) 
                \right) +\sum_{i=1}^nu_j(f(x)) \left( \sum \partial _i \d_{ij}(x) \right)\\
        =&\sum_{k,j}\partial _k u_j(f(x))\delta _{jk} J_f(x)=\vp(f(x))J_f(x)
    \end{align*}
    Behauptungen: Siehe Übungsblatt 2.
    Wir wollen nun zeigen, dass der über die Determinantenformel definierte Abbildungsgrad konstant
    ist auf Zusammenhangskomponente. Zu zeigen: $\deg(f,\Omega,y_1)=\deg(f,\Omega,y_2)$.

    \begin{lem}\label{2.8}
        Sei $f\in C^2(\ol \Omega,\R^n),$ $ y_0\nin f(\partial \Omega) $,
        \[
            \rho:=\dist(y_0,f(\partial \Omega))
        \]
        Dann ist $\deg(f,\Omega,\cdot)$ (Definiert durch die Determinantenformel) konstant auf
        \[
            B_\delta (y_0)\cap \RV(f)
        \]
    \end{lem}

    \begin{proof}
        Sei $y^j\in B_\delta (y_0)\cap \RV(f),$ $j=1,2$, sei $\delta := \rho-\max_{j=1,2}\|y^j-y_0\|$.
        Sei $\eps>0$, so dass
        \[
            \deg(f,\Omega,y^j)=\int \eta_\eps(f(x)-y^j)J_f(x) \d x \qq \text{(nach Lemma $\ref{2.6}$)}
        \]
        mit Proposition $\ref{2.7}$ b) gilt
        \begin{align*}
            \eta_\eps(x-y^2)-\eta_\eps(x-y^1)&= \eta_\eps (x-y^1+(y^1-y^2))-\eta_\eps(x-y^1)\\
            &= \div w(x) \qq\text{mit}\\
            w(x)&= (y^1-y^2)\int_0^1 \eta_\eps (x-y^1+t(y^1-y^2))\d t
        \end{align*}
    \textit{Behauptung:} Es gilt $\supp w\subset B_\delta(y_0)$
    \begin{proof}
        Sei $x\in \supp w$. Damit existert $t\in [0,1]$ mit $|x-y^1+t(y^1-y^2)|<\eps$
        \begin{align*}
            \Ra \, |x-y_0|&\leq \eps + |y^1-t(y^1-y^2)-y_0|\leq \eps +|(1-t)(y^1-y_0)+t(y^2-y_0)|\\
                \eps + \rho-\delta&<\rho
        \end{align*}
        Damit gilt aber, dass $f(\partial \Omega)\cap \supp w=\varnothing$. Mit Proposition 
        $\ref{2.7}$ e) existiert
        \[
            v\in C^1(\Omega),\, \supp v \subset \Omega.
        \]
        und
        \[
            \left( \eta_\eps(f(x)-y^2)-\eta_\eps (f(x)-y^1) \right)J_f(x)=\div v(x)
        \]
        Mit der Proposition $\ref{2.7}$ a) folgt die Behauptung. \[  \]
    \end{proof}

    \end{proof}
    \end{description}
\end{proof}

\begin{defi}\label{2.9}
    Sei $f\in C^2(\ol \Omega,\R^n)$, $y\in f(\partial \Omega)$. Wir setzen
    \[
        \deg (f,\Omega,y)= \deg(f,\Omega,\tilde y)=\sum_{x\in f^{-1}(\tilde y)} \sign J_f(x),
    \]
    wobei $\tilde y \in \RV(f)$ mit
    \[
        |\tilde y- y| < \dist(y,f(\partial \Omega))
    \]
\end{defi}

\noindent\textit{Behauptung:} $\deg$ ist somit nach dem vorherstehenden Überlegungen wohldefiniert.
\subsubsection*{2. Schritt: Nur stetige Funktionen $f$}

Auch hier die Idee: Sei $f\in D_y(\ol \Omega, \R^n)$. Wir suchen $\tilde f \in C^2 (\ol \Omega, \R^n)$
hinreichend nahe an $f$ (in derselben Zushgskomponente von $D_y(\ol \Omega, \R^n)$) und übertragen
den Wert von $\deg(\cdot , \Omega, y)$ von $\tilde f$ auf $f$. Es bleibt zu zeigen, dass die so
definierte Funktion stetig ist.

\begin{lem}\label{2.10}
    Sei $f\in D_y^2(\ol \Omega, \R^n)$, sei $g\in C^2(\ol \Omega, \R^n)$. Dann existiert $\eps >0$, so 
    dass
    \[
        \deg (f+tg,\Omega,y)= \deg(f,\Omega,y) \qq \forall t\in (-\eps,\eps)
    \]
\end{lem}
\begin{proof}
    \begin{description}
        \item{1)}
        $f(y)=\varnothing \, \Ra \, (f+tg)^{-1}(y)=\varnothing$ falls
        \[
            |t| < \frac{\dist(y,f(\ol \Omega))}{\|g\|_\infty}
        \]
        \item{2)}
        $y\in \RV (f) \, \Ra \, f^{-1}(y)=\{x\}_{i=1}^N$. Mithilfe des \textit{Satzes über die 
            implizite Funktion} finden wir
            \[
                U(x^i)=:U^i
            \]
            disjunkte Umegbungen, so dass eindeutige Lösungen $x^i(t)\in U^i$ existieren von
            \[
                (f+tg)(x)=y \qq \forall |t|<\eps
            \]
            Wir können (zumindest auf evtl. noch kleineren $U^i$) annehmen, dass das Vorzeichen
            von $J_{f+tg}$ konstant ist auf $U^i$. Sei 
            \[
                \eps _2= \frac{\dist(y,f(\Omega\setminus \bigcup_{i=1}^NU^i))}{\|g\|_\infty}.
            \]
            Dann gilt $\dist(y,(f+tg)(\partial \Omega))>0$ $\forall |t|<\eps_2$. Das Lemma gilt somit
            für $\eps = \min(\eps,\eps_2)$
        \item{3)}
            $y\in \CV(f)$. Sei dann $\tilde y\in \RV(f)$.
            \[
                |y-\tilde y|< \frac{\rho}{3} = \frac13 \dist(y,f(\partial \Omega)).
            \]
            Nach Definition \ref{2.9} gilt
            \[
                \deg (f,\Omega,y)=\deg(f,\Omega,\tilde y).
            \]
            Sei $\tilde \eps>0$ mit $\deg(f,\Omega,\tilde y)=\deg(f+tg,\Omega,\tilde y)$ für $|t|<\tilde
            \eps$ (Nach Schritt 2).
            Mit $\eps=\min(\tilde \eps , \frac{\rho}{3\|g\|_\infty})$ gilt:
            \[
                |\tilde y-(f+tg)(x)|\geq \frac{\rho}{3} \qq \forall x \in \partial \Omega
            \]
            Somit gilt
            \[
                |\tilde y-y|< \dist (\tilde y, (f+tg)(\partial \Omega))
            \]
            also
            \[
                \deg(f+tg,\Omega,\tilde y)= \deg (f+tg,\Omega,y)
            \]
            Es folgt
            \[
                \deg(f,\Omega, y)= \deg(f+tg,\Omega,y)
            \]
    \end{description}
    \[ \]
\end{proof}

\begin{theorem}[Existenz und Eindeutigkeit des Brower'schen Abbildungsgrades]\label{2.11}
    Es existiert eine eindeutige Abbildung
    \[\deg(f,\Omega,y)\]
    mit den Eigenschaften \textbf{(D1)}-\textbf{(D4)}. Weiter gilt
    \[
        \deg(\cdot, \Omega, y): D_y(\ol \Omega, \R^n)\ra \Z
    \]
    ist konstant auf Zusammenhangskomponente von $D_y(\ol \Omega, \R^n)$. Für $f\in D_y(\ol \Omega,\R^n)$
    ist $\deg(f,\Omega,y)$ gegeben durch
    \[
        \deg(f,\Omega,y)=\sum_{x\in \tilde f ^{-1}(y)} \sign J_{\tilde f}(x)  
    \]
    wobei
    \[
        \tilde f \in D_y^2(\ol \Omega, \R^n)
    \]
    beliebig aus derselben Komponente von $D_y(\ol \Omega,\R^n)$ wie $f$ gewählt werden kann mit $y
    \in \RV(\tilde f)$
\end{theorem}

\begin{proof}

Aus Lemma \ref{2.8} und \ref{2.10} folgt, dass $\deg$ wohldefiniert ist und lokal konstant mit 
Werten in $\Z$. Die Abbildung $\deg$ auf Zusammenhangskomponenten.
\begin{description}
\item[(D2)] $\ra$ klar.
\item[(D1)] gilt nachdem diese Bedingung per Konstruktion für die Determinatenformel gilt.
\item[(D3)] Wir wählen $\|f-\tilde f\|_\infty<\dist(y,f(\ol \Omega\setminus(\Omega_1\cap \Omega_2)))$.
    \textbf{D3} gilt per Konstruktion für $\tilde f$, sonst für $f$.
\item[(D4)] folgt aus der Konstruktion von $f$ auf Zusammenhangskomponente.
\end{description}
\[ \]
\end{proof}

\subsubsection*{Beispiele:}
\begin{description}
    \item{1)}
    Sei 
    \[
        f(x_1,x_2)=\begin{pmatrix} x_1-2x_2+\cos(x_1+x_2)\\x_2+2x_1+\sin(x_1+x_2) \end{pmatrix}
    \]
    und
    \[
        g(x_1,x_2):=\begin{pmatrix} x_1-2x_2 \\ x_2+2x_1 \end{pmatrix}
    \]
    Es gilt $|g(x)|=\sqrt{5} |x|$ und $|f(x)-g(x)|=\sqrt{\sin^2(x_1+x_2)+\cos^2(x_1+x_2)}=1$.
    Sei $h(t)=(1-t)g+tf=g+t(f-g)$.
    \begin{align*}
        |h(t)|&\geq |g|-t|f-g|>0 \qq \text{für} \qq |x|>\frac1{\sqrt{5}}\\
        \Ra \, \deg(f,B_r(0),0)&= \deg(g,B_r(0),0)=1 \qq \text{für} \qq r>\frac1{\sqrt{5}}
    \end{align*}
    Somit existiert eine Lösung $x:f(x)=0$.

    \item{2)}
    \begin{theorem}
        Ein stetiges Vektorfeld im $\R^n$, das auf einer Kugeloberfläche überall nach außen zeigt, muss
        auf einem Punkt im Innern der Kugel verschwinden. 
        
        Anders formuliert, sei $f:\ol{B_R(0)} \ra \R^n$ stetig, so dass $f(x)\cdot x>0$ $\forall |x|=R$. 
        Dann existiert ein $x_0\in B_R(0)$ mit $f(x_0)=0$.
    \end{theorem}

    \begin{proof}
        Wir haben $\deg(\Id, B_R(0),y)=1$ für $y\in B_R(0)$. Angenommen, $f(x)\neq 0$ für alle
        $x\in B_R(0)$. Dann gilt $f^{-1}(0)\cap B_R(0)=\varnothing$.
        \[
            \Ra \, \deg(f,B_R(0),0)=0
        \]
        Sei $H(t)=(1-t)\Id+tf$.
        \begin{align*}
            \deg(H(0),B_R(0),0)&=1\\
            \deg(H(1),B_R(0),0)&=0.
        \end{align*}
        Es existiert ein $t_0\in (0,1)$, so dass $H(t_0)\nin D_0(\ol{B_R(0)},\R^n)$.
        \[
            \exists x_0\in \partial B_R(0)\cdot (H(t_0))(x_0)=0.
        \]
        \begin{align*}
            0&=H(t_0)(x_0)\\
            \Ra \, 0&= H(t_0)(x_0)\cdot x_0= \underbrace{(1-t_0)R^n}_{>0}+\underbrace{t_0f(x_0)x_0}_{>0}
            >0
        \end{align*}
        Widerspruch!
        \[ \]
    \end{proof}
\end{description}

\subsection{Der Brouwer'sche Fixpunktsatz}

Der \textit{Brouwer'sche Fixpunktsatz} ist eine Folgerung aus den Eigenschaften des Abbildungsgrades.
Er besagt, dass stetige Abblidungen, die kompakte, konvexe Mengen im $\R^n$ (oder Mengen, die dazu
homöomorph sind) in sich selbst abbilden,einen Fixpunkt besitzen.

\begin{theorem}[Fortsetzung stetiger Funktionen]\label{2.12}
    Sei $X$ ein metrischer Raum, $Y$ normierter Raum. Sei $K\subset X$ abgeschlossen. Sei $F\in C(K,Y)$.
    Dann besizt $F$ eine stetige Fortsetzung Fortsetzung $G:X\ra Y$, so dass
    \[
        G(x)\subset\conv\, F(K)
    \]
\end{theorem}

\begin{proof}
    Wir betrachten die offene Überdeckung
    \[
        \left\{ B_{\rho(x)}(x) \right\}_{x\in X\setminus K} \qq \text{von} \qq X\setminus K
    \]
    mit $\rho(x)=\frac12 \dist(x,K)$.
    Wir wählen nun eine lokal endliche Zerlegung der Eins $\{\vp_\lambda\}_{\lambda\in \Lambda}$, welche
    der offenen Überdeckung untergeordnet ist.\\[.5cm]

    \noindent \textbf{Einschub:}
    \begin{description}
        \item[Zerlegung der Eins:]
        Sei $\{U_i\}_{i\in I}$ eine Überdeckung von $X$, topologischer Raum. $\{\vp_\lambda\}_{\lambda\in
        \Lambda}$ ist eine $\{U_i\}_{i\in I}$ untergeordnete lokal endliche Zerlegung der Eins,
        falls gilt
        \begin{itemize}
        \item[-] $\forall \lambda\in \Lambda\, \exists \, i \in I: \, \supp \vp_\lambda\subset U_i$
        \item[-] $\sum_{\lambda \in \Lambda} \vp _\lambda=1$
        \item[-] $\forall x \in X \, \exists \, U(x)$ Umgebung von $x$, so dass nur
        \textit{endlich viele} $\lambda \in \Lambda$ existieren, mit $U(x)\cap \supp \vp_\lambda\neq 
        \varnothing$ (lokal endlich).
        \item[-] $\vp_\lambda \in C(U,[0,1]) \,\forall\, \lambda$. 
        \end{itemize}
        \item[Konstruktion] aus einer lokal endlichen Überdeckung $(v_\lambda)_{\lambda\in\Lambda}$ 
        ($\forall x\in X \, \exists \, U(x)$ Umgebung: $U(x)\cap V_\lambda\neq \varnothing$ nur für
         endlich viele $\lambda\in \Lambda$.) in metrischen Räumen $(X,d)$.
        \begin{itemize}
            \item[-] Sei $\alpha(x)=\sum_{\lambda\in \Lambda} \dist(x,X\setminus V_\lambda)>0 \, \forall
            x \in X$, mit $\dist(x,\varnothing):=1$.
            \item[-] $\vp _\lambda(x):=\frac{1}{\alpha(x)}d(x,X\setminus V_\lambda) \, \in [0,1]$,
            $\vp_\lambda=0$ für $x\nin V_\lambda$.
        \end{itemize}
        \item[„Konstruktion“] einer lokal endlichen Überdeckung $\{V_\lambda\}_{\lambda\in\Lambda}$, die
        eine offen Überdeckung $\{U_i\}_{i\in I}$ verfeinert (d.h. $\forall \, \lambda\in \Lambda \, 
        \exists i \in I, \, V_\lambda\subset U_i$).
        \item{Die Existenz} einer solchen lokal endlichen verfeinernde Überdeckung zu jeder offenen
        Überdeckung ist die Definition der \textit{Parakompaktheit}.
        \begin{theorem*}
            Jeder metrische Raum ist parakompakt.
        \end{theorem*}
        
        \noindent \textsc{Beweisidee:} Angenommen $I=\N$ also wohlgeordnet. (jede Teilmenge besitzt ein
                eindeutiges kleinstes Element.) Wir setzen für $i\in I, \, n\in \N$.
        \[
            D_{in}=\bigcup_{x\in \Phi(i,n)} B_{2^{-n}}(x)
        \]
        \[
            \Phi(i,n):=\left\{x\in X \, \Big| \, \text{mit}\, \begin{cases}\text{$i$ ist die kleinste Zahl,
            so dass $x\in U_i$}\\ \text{$x\nin D_{jm}$ für $m<n$}\\ B_{3\cdot2^{-n}}\subset U_i  
            \end{cases}\right\}
        \]
        \item klar: $\{ D_{in} \}$ verfeinert $\{U_i\}$
        \item[Überdeckung:] Klar, denn für jedes $x\in X$ finden in ein kleinstes $i\in I=\N$ und ein
        $n$ hinreichend groß.
        \item[lokale Endlichkeit] (s.Artikel).
    Ist $I$ nun eine beliebige Indexmenge, so läßt sich diese wohlordnen (\textit{Zorn'sches Lemma}),
    und die Wohlordnung ist die einzige Eigenschaft von $I$, die wir benutzt haben.   
    \end{description}
    Einschub Ende, weiter im Beweis.

    Sei
    \[
    G(x):= \begin{cases} F(x)& \text{für} \qq x\in K\\
           \sum_{\lambda \in \Lambda } \vp_\lambda(x) \cdot F(x_\lambda) & \text{für} 
           \qq x\in X\setminus K\end{cases},
    \]
    wobei $x_\lambda$ beliebig in $K$ so gewählt ist, dass 
    \[
        \dist(x_\lambda,\supp \vp_\lambda)\leq 2\cdot \dist(K,\supp \vp_\lambda).
    \]
    $G$ ist offensichtlich stetig auf $X\setminus \partial K$.

    $G$ ist offensichtlich stetig auf $X\setminus\partial K$ und
    \[
        G(X) \subset \conv\,G(K).
    \]
    Sei also $x_0\in\partial K$ und sei $\eps>0$. Wir wählen $\delta >0$, so dass
    \[
        \|F(x)-F(x_0)\|\leq\eps \qq \forall x\in K \qq \text{mit} \qq d(x,x_0)\leq 9\delta.
    \]
    Es bleibt zu zeigen, dass
    \[
        \|G(x)-F(x_0)\|\leq \eps \qq \forall x\in X\setminus K \qq \text{mit} \qq d(x,x_0)\leq \delta
    \]a
    Sei also $x\nin K$, dann gilt
    \[
        \|G(x)-F(x_0)\|\leq \sum_{\lambda\in \Lambda} \vp_\lambda (x) \|F(x_\lambda)-F(x_0)\|
    \]
    Nach Konstruktion liegt $x_\lambda$ nicht weit von $x$ entfernt, damit nicht weit von $x_0$, falls
    $x\in\supp \, \vp _\lambda$. In der Tat gilt für $x\in \supp\,\vp_\lambda$:
    \begin{align*}
        d(x,x_\lambda)&\leq \dist(x_\lambda,\supp\,\vp_\lambda)
        +\underbrace{\diam(\supp\, \vp_\lambda)}_{=\sup_{x,y\in\supp\,\vp_\lambda} d(x,y)}\\
            &\leq 2\dist(K,\supp \, \vp _\lambda)+\diam(\supp\, \vp _\lambda)
    \end{align*}
    Nachdem $\{\vp _\lambda\}_{\lambda\in\Lambda}$ der Überdeckung $\{B_{\rho(x)}\}_{x\in X\setminus K}$
    untergeordnet ist, existiert $\tilde x\in X\setminus K$, so dass
    \begin{align*}
        \supp\, \vp _\lambda &\subset B_{\rho(\tilde x)}(\tilde x)
    \end{align*}
    \begin{align*}
        \Ra \, \diam (\supp \, \vp _\lambda) &\leq 2\rho(\tilde x)=\dist(\tilde x,K)
        \leq2\dist(K,B_{\rho(\tilde x)}(\tilde x))\leq 2 \dist(K,\supp \vp_\lambda)\\
        \Ra \, d(x_0,x_\lambda)&\leq 4 \dist(K,\supp\, \vp_\lambda).
    \end{align*}
    Und es folgt
    \begin{align*}
        d(x_0,x_\lambda) &\leq d(x_0,x)+d(x,x_\lambda)\\
                &\leq d(x_0,x)+4\dist(K,\supp\,\vp_\lambda)\\
                &\leq d(x_0,x)+8\dist(K,\supp\, \vp_\lambda)\\
                &\leq d(x_0,x)+8d(x_0,x)=9d(x_0,x)\leq \delta.
    \end{align*}
    Nach Wahl von $\delta$ gilt
    \[
        \|F(x_{\lambda})-F(x_0)\|\leq \eps \qq \forall \lambda\in\Lambda:\, \vp_\lambda(x)\neq 0.
    \]
    Somit gilt
    \[
        \|G(x)-G(x_0)\|\leq \eps \qq \text{für} \qq d(x,x_0)\leq \delta.
    \]
\end{proof}

\begin{remark}
    Mit Hilfe dieses Satzes und dem Abbildungsgrad lässt sich der sog. „Igelsatz“ zeigen, der besagt,
    dass man einen Igel nicht stetig kämmen kann. Übungsaufgabe. 
\end{remark}

\begin{theorem}[Brouwerscher Fixpunktsatz]\label{2.13}
    Sei $K$ ein topologischer Raum, der zu einer kompakten konvexen Teilmenge des $R^n$ homöomorph ist.
    Sei $f\in C(K,K)$. Dann besitzt $f$ einen Fixpunkt.
\end{theorem}

\begin{proof}
    \begin{description}
    \item{1.}
    $K=\ol{B_r(0)}\subset\R^n$. Falls ein Fixpunkt auf dem Rand existiert, dann sind wir fertig.
    Ansonsten gilt für $H(t)=\Id-tf$, dass $0\nin H(t)(\partial B_r(0))$, nachdem
    \[
        |H(t)(x)|\geq |x|-t|f(x)|\geq (1-t)r>0 \qq \text{für} \qq 0\leq t< 1, \qq x\in\partial B_r(0).
    \]
    Nach der Annahme der Nichtexistenz eines Fixpunktes auf $\partial B_r(0)$ ist auch $H(1)(x)\neq 0$
    $\forall x \in \partial B_r(0)$.
    \[
        \Ra \, \deg(\Id-f,B_r(0),0)=\deg(\Id,B_r(0),0)=1.
    \]
    Somit existiert $x\in B_r(0)$ mit $x=f(x)$.
    \item{2.}
    Sei nun $K\subset \R^n$ konvex, kompakt. Für ein $\rho>0$ gilt $K\subset B_\rho(0)$ und gemäß Satz
    \label{2.12} können wir $f$ stetig durch $g$ auf $\ol{B_{\rho}(0)}$ fortsetzen mit
    \[
        g\left( \ol{B_\rho (0)} \right)\subset \conv \, K=K.
    \]
    Nach 1. finden wir $x\in \ol{B_\rho(x)}$ mit $x=g(x)$.
    Es gilt $g(x)\in K$, somit folgt $x\in K$ und wir haben $x \in K$ mit $f(x)=x$.
    \item{3.}
    Sei $K$ homöomorph zu $K^\star \subset \R^n$ kompakt, konvex. Sei $h$ die entsprechende Homöomorphie.
    nach 2. besitzt $h\circ f \circ h^{-1}$ einen Fixpunkt $x^\star\in K^\star$. Damit ist aber $x=h^{-1}
    (x^\star)\in K$ ein Fixpunkt von $f$.
    \end{description}
    \[ \]
\end{proof}

\begin{remark}
    \begin{description}
    \item{1.}
    Die Bedingungen sind tatsächlich notwendig. Gegenbeispiele siehe Übung.
    \item{2.}
    Es existiert auch eine stetige Abbildung $f\in C(\ol{B_1(0)},\ol{B_1(0)})$, $B_1\subset X$ separabler
    $\infty$-dimensionaler Hilbertraum ohne Fixpunkt.
    \begin{description}
    \item{$\ra$}
    Beispiel von Kakutani später.
    \item{$\ra$}
    \textit{Schauderscher Fixpunktsatz}.
    \end{description}
    \item{3.}
    Im eindimensionalen Fall ist \ref{2.13} nichts anderes als der Zwischenwertsatz angewendet auf
    $x-f(x)$.
    \item{4.}
    Vergleich mit dem \textit{Banach'schen Fixpunktsatz}: Wesentlich geringere Anforderungen an den
    Operator (nur Stetigkeit), dafür hohe Anforderungen an den Raum (endlichdim., kompakt, konvex).
    Wir bekommen keine Eindeutigkeitsaussage.
    \end{description}
\end{remark}

\subsubsection*{Ein Anwendugsbeispiel}

Existenz positiver (bzw. nichtnegativer) Eigenwerte und Eigenfunktionen. Sei $A=(a_{ij})_{i,j=1}^n$ eine
Matrix und sei $a_{ij}\geq 0$ $\forall (i,j)$. Dann existiert $\lambda\geq 0$, $x=(x_i)_{i=1}^n$,
$x_i\geq 0$ $\forall i$ mit
\begin{align}\label{6}
    Ax=\lambda x
\end{align}

\begin{proof}
    Sei 
    \[
        K=\{ x\in \R^n \, |\, x\geq 0, \, \sum_{i}x_i=1 \}
    \]
    kompakt und konvex.
    \begin{description}
        \item{1)}
        Falls $Ax=0$ für ein $x\in K$, gilt \ref{6} mit $\lambda=0$.
        \item{2)}
        Sei $Ax\neq 0$ $\forall x\in K$. Dann existiert $\alpha>0$ mit
        \[
            \sum_i(Ax)_i\geq \alpha \qq \forall x\in K.
        \]
        Es sei also
        \[
            f:x\mapsto \frac{Ax}{\sum_i (Ax)_i}; \qq (f(x))_i \geq 0 \qq \forall i
        \]
        und es gilt
        \[
            \sum_i (f(x))_i=1 \qq \forall x\in K.
        \]
        Dann ist $f(K)\subset K$ und nach Satz $\ref{2.13}$ existiert ein Fixpunkt $x\in K$ mit $x=f(x)$,
        d.h.
        \[
            Ax=\lambda x \qq \text{mit} \qq \lambda =\sum_i(Ax)_i >0
        \]
    \end{description}\[ \]
\end{proof}

\subsection{Weitere Eigenschaften des Abbildungsgrades}

Sei $m<n$, wir identifizieren im Folgenden den $\R^m$ mit dem Unterraum
\[
    \{x\in \R^n \, | \, x_{m+n}=…=x_n=0\}\subset \R^n.
\]

\begin{theorem}[Reduktionseigenschaften des Abbildungsgrades]\label{2.14}
    Sei $\Omega \subset \R^n$ offen und beschränkt, $\Omega \cap \R^m \neq \varnothing$ und sei
    \[
        f:\ol \Omega \ra \R^m\, \text{stetig,} \qq g=\Id-f
    \]
    Sei $y\in \R^m$, $y\nin g(\partial \Omega)$. Dann gilt
    \[
        \deg(g,\Omega,y)=\deg(g|_{\ol{\Omega}\cap\R^m},\Omega|_{R^m},y).
    \]
\end{theorem}

\begin{proof}
    Sei $\Omega_m=\Omega\cap\R^m\neq \varnothing$ offen, beschränkt im $\R^m$, und
    $g_m:=g|_{\ol{\Omega_m}}$. Es gilt
    \[
        \partial \Omega_m=\partial\Omega\cap\R^m.
    \]
    und
    \[
        g_m(\ol{\Omega_m})\subset\R^m,\, y\nin g_m(\partial\Omega_m).
    \]
    $\Ra$ $\deg(g_m,\Omega_m,y)$ ist definiert (als Abbildungsgrad im $\R^m$).
    \begin{description}
        \item{1)}
        Sei $g\in C^1(\ol \Omega)$ und $y\in \RV(g)$ und $x\in \Omega$, $g(x)=y$.
        \[
        \Ra \, x=y=f(x)\in \R^m \qq \Ra \qq x\in \Omega_m.
        \]
        Somit haben wir
        \[
            g^{-1}(y)=g_m^{-1}(y).
        \]
        Nach der Determinantenformel genögt es nun zu zeigen, dass
        \[
            J_{g_m}(x)=J_g(x) \qq \text{für} \qq x\in \Omega_m.
        \]
        Es sei $\Id_k$ die $k\times k$-Einheitsmatrix. Es gilt
        \begin{align*}
            J_{g_m}(x)&=\det(\Id_m-f'(x)) \qq (f|_{\Omega_m}')\\
            J_g(x)&= \det\begin{pmatrix} \Id_m-(\partial_jf_i(x)) & -\partial_jf_i\\
            0& \Id_{n-m}\end{pmatrix} \qq \text{wegen} \, f(\ol\Omega)\subset\R^m.
        \end{align*}
        Die gewünschte Aussage folgt sofort durch Entwicklung der Determinante nach den letzten
        $n-m$ Zeilen.
        \item{2)}
        Der allgemeine Fall folgt durch Wahl von $\tilde g=\Id-\tilde f$ hinreichend nah an $g$, so dass
        $\tilde g$ die in 1. geforderten Eigenschaften besitzt.
    \end{description}
    \[ \]
\end{proof}


Es sei nun wieder $\Omega \subset \R^n$ offen, beschränkt, $f\in C(\ol \Omega,\R^n)$. Nach Satz \ref{2.2}
gilt, dass $\deg(f,\Omega,\cdot)$ konstant ist auf Zusammenhangskomponenten von $\R^n\setminus f(\partial 
\Omega)$. Wir bezeichnen diese Zusammenhangskomponenten mit $\{G_j\}_{j\in I}$ und schreiben $\deg(f,
\Omega,y)=\deg(f,\Omega,G_j)$ für $y\in G_j$

\begin{theorem}[Produktregel]\label{2.15}
    Sei $\Omega\subset\R^n$ offen, beschränkt. Seien $G_j$ die Zusammenhangskomponenten von $\R^n
    \setminus f(\partial \Omega)$ für $f\in C(\ol\Omega, \R^n)$ und sei
    \[
        g\circ f \in D_y(\ol \Omega,\R^n), \qq y\in \R^n
    \]
    Dann gilt
    \begin{align}\label{9}
        \deg(g\circ f, \Omega, y)=\sum_j\deg(f,\Omega,G_j)\cdot \deg(g,G_j,y)
    \end{align}
\end{theorem}

\begin{proof}
    (Endlichkeit der Summe) Es gilt $f(\ol\Omega)$ ist kompakt, somit existiert $r>0$ mit $f(\ol \Omega)
    \subset B_r(0)$. Nachdem $g^{-1}(y)$ abgeschlossen ist, gilt $g^{-1}(y)\cap B_r(0)$ kompakt ist.
    $\{G_j\}_{j\in I}$ ist eine offene Überdeckung dieser Menge, es genügen somit endlich viele $\{  
    G_j \}_{j=1}^N$ um $g^{-1}(y)\cap B_r(0)$ zu überdecken.
    \begin{description}
    \item{1)}
    Wir nehmen an, dass $f\circ g\in C^1(\ol \Omega)$, $y\in \RV(g\circ f)$. Es gilt nach der
    Kettenregel, dass
    \[
        (g\circ f)'(x)=g'(f(x))\circ f'(x)
    \]
    Die Behauptung folgt dann durch eine längere Rechnung.
    \begin{align*}
        \deg(g\circ f,\Omega,y)&=\sum_{x\in(g\circ f)^{-1}(y)}\sign J_{g\circ f}(x)=\sum_{x\in (g\circ f)
        ^{-1} (y)}(\sign J_g(f(x)))\cdot(\sign J_f(x))\\
        &= \sum_{z\in g^{-1}(y)} \sign J_g(z)\cdot \sum_{x\in f^{-1}(z)} \sign J_f(x)
        =\sum_{z\in g^{-1}(y)} \sign J_g(z)\cdot \deg(f,\Omega,z)
    \end{align*}
    Mit der Überdeckung $\{G_j\}_{j\in I}$ von $g^{-1}(y)$ gilt
    \begin{align*}
        \deg(g\circ f,\Omega,y)&=\sum_{j=1}^m\sum_{z\in g^{-1}(y)\cap G_j}\sign J_g(z)\cdot \deg(f,\Omega,z)
        =\sum_{j=1}^N \deg(f,\Omega,G_j) \cdot \sum_{z\in g^{-1}(y)\cap G_j} \sign J_g(z)\\
        &= \sum_{j=1}^N \deg(f,\Omega,G_j)\cdot \deg(g,G_j,y)
    \end{align*}
    Diese Formel gilt nun natürlich auch für $y\in \CV(g\circ f)$ und für $g$ nur stetig. Etwas
    problematischer ist der Fall, dass $f$ nur stetig ist, da sich bei der Modifikation von $f$ die
    Mengen $G_j$ ändern.

    Es sei
    \[
        L_l=\{z\in \R^n\setminus(\partial \Omega)\, | \, \deg(f,\Omega,z)=l\}
    \]
    Für $l\neq 0$ gilt, dass $L_l$ aus einer Vereinigung von Mengen $G_j$ bestehen muss. Sei nun
    $\tilde f\in C^1 (\ol \Omega)$, so dass $|f(x)-\tilde f(x)|< \frac 12 \dist(g^{-1}(y),
    f(\partial\Omega))$ für $x\in \ol \Omega$. Wir definieren $\tilde G_j$ und $\tilde L_l$ entsprechend
    für die Funktion $\tilde f$.
    Es gilt
    \[
        L_l\cap g^{-1}(y)=\tilde L_l\cap g^{-1}(y).
    \]
    nach Satz \ref{2.1} (iii). ($\|f(x)-\tilde f(x)\|<\dist(y,f(\partial \Omega)\qq \forall
    x\in \partial \Omega)\qq \Ra \qq \deg(f,\Omega,y)=\deg(\tilde f,\Omega,y)$.) Es gilt somit
    \begin{align*}
        \deg(g\circ f, \Omega, y)&=\deg(g\circ\tilde f,\Omega,y)\\
        &=\sum_{j}\deg(f,\Omega,\tilde G_j)\cdot \deg(g,\tilde G_j,y)=\sum_{l\neq0} l\cdot\deg( g,\tilde
        L_l,y)=\sum_{l\neq 0} l\cdot \deg(g,L_l,y)\\
        &=\sum_j \deg(f,\Omega,G_j)\cdot \deg(g,G_j,y)
    \end{align*}
    \end{description}
    \[ \]
\end{proof}

Eine wichtige Anwendung der Produktformel ist das folgende
\begin{theorem}[Jordanische Kurvensatz]\label{2.16}
Es seien $C_1$ und $C_2$ zwei kompakte, zueinander homöomorphe Teilmengen des $\R^n$. Dann besitzen
$\R^n\setminus C_1$ und $\R^n\setminus C_2$ die selbe Anzahl von Zusammenhangskomponenten.
\end{theorem}
 %% Nach dem Einschub geht es weiter mit 2.4 Eigenschaften des Abbildungsgrades.

\subsection{Der Fixpunktsatz von Kakutani und eine Anwendung in der Spieltheorie}

Im Folgenden beweisen wir die sogenannten \textit{Nash-Gleichgewichte} in n-Personen.
Als Vorbereitung ist es notwendig, den \textit{Brouwer'schen Fixpunktsatz} auf mengenwertige Funktionen
zu verallgemeinern.

Für $K\subset \R^n$ konvex, kompakt, bezeichnen wir mit $\CS(K)$ die Menge der konvexen (Teil-)Mengen
von $K$.

\begin{theorem}[Kakutani]\label{2.16}
    Sei $K\subset \R^n$ kompakt, konvex $f:K\ra \CS(K)$. Falls die Menge
    \[
        \Gamma := \{(x,y)\in K\times K\, |\, y\in f(x)\}
    \]
    in $K\times K$ abgeschlossen ist, dann existiert $x\in K$ mit $x\in f(x)$.
\end{theorem}

\begin{proof}
    In einer Dimension: $K=[v_1,v_2]$, wir wählen $y_j\in f(v_j)$ und definieren
    \[
        f^1(x)=\sum_{j=1}^{2} \lambda_j(x)y_j,
    \]
    wobei $\lambda _j(x)$ die baryzentrischen Koordinaten von $x$ bezeichne, d.h.
    \[
        \lambda_j(x)\geq 0, \qq \sum_{i=1}^2 \lambda_1(x)=1,\qq x=\sum_{j=1}^2 \lambda_j(x)v_j
    \]
    Nach Konstruktion ist $f^1:K\ra K$ stetig, besitzt mit Brouwer also einen Fixpunkt. Das hilf
    noch nicht viel. Wir betrachten die $k-1$-te baryzentrische Unterteilung der Menge $K$ gegeben
    durch die Vertizes 
    \[
        \{v_i\}_{i=1}^{2^{k-1}+1}
    \]
    Wir wählen wieder $y_j\in f(v_j)$ und definieren $f^k(v_j)=g_j$ interpolieren auf den unterteilten
    Intervallen wie gehabt durch baryzentrische Koordintaten. Es gilt
    \[
        f^k: K\ra K
    \] 
    ist stetig, also mit Fixpunkt
    \begin{align}\label{7}
        x^k=\sum_{i=1}^2 \lambda_i^kv_i^k= \sum_{i=1}^2 \lambda_i^ky_i^k \qq \text{mit} \qq 
        y_i^k=f^k(v_i^k)
    \end{align}
    in einem Teilintervall  (in einem Simplex der $k$-fachen baryzentrischen Koordinaten). Es gilt:
    \[
        \left(x^k,\lambda_1^k,\lambda_2^k,y_1^k,y_2^2\right)\in K\times[0,1]^2\times K^2
    \]
    Somit konvergiert eine Teilfolge in $K$ gegen
    \[
        \left(x^0,\lambda_1^0,\lambda_2^0,y_1^0,y_2^0\right)
    \]
    Nachdem die Teilintervalle zu Punkten degenerieren gilt:
    \[
        v_i^k\ra x^0 \qq i=1,2
    \]
    Wir haben 
    \[
        \left(v_i^k,y_i^k\right)\in\Gamma\ra \left(v_i^0,y_i^0\right)\in \Gamma \qq 
        \text{wegen Abgeschlossenheit von }\Gamma
    \]
    Somit gilt $y_i^0\in f(x^0)$, $i=1,2$. Mit (\ref{7}) folgt
    \[
        x_0 =\lim_{k\ra\infty} \sum_{i=1}^2\lambda_i^ky_i^k=\sum_{i=1}^2 \lambda^0_iy_i^0 
        \in f\left(x^0\right),
    \]
    $f\left(x^0\right)$ ist konvex. 

    In $n$ Dimensionen nehmen wir zunächst an, dass $K$ ein Simplex (Konvexe Hülle von $n+1$
    nichtdegenerierten Vertizes) ist. Wir wenden wiederholt die Baryzentrische Unterteilung an,
    der Durchmesser der Simplizes schrumpft mit einem Faktor $\frac{n}{n+1}$.
    Zur $k$-ten baryzentrischon Unterteilung mit Vertizes $\{v_j\}_{j=1}^{N_k}$. Wählen wir wieder
    $y_j\in f(v_j)$ mit $f^k(x)$ durch Interpolation mittels baryzentrischen Koordinaten in jdn.
    Unter-Simplex. Wir erhalten wieder Fixpunkte $x^k$, und eine Teilfolge von
    \[
        (x^k,\lambda_1^k,…,\lambda_{n+1}^k,y_1^k,…,y_{n+1}^k)
    \]
    konvergiert im „richtigen“ Unter-Simplex. Der Rest des Beweises folgt exakt wie im eindimensionalen
    Fall. Ist $K$ kein Simplex, so können wir $f^k$ jeweils auf einen Simplex $\tilde K\supset K$ stetig
    fortsetzen und verfahren wie im Beweis des \textit{Brouwer'schen Fixpunktsatzes}.
    \[ \]
\end{proof}

\subsubsection*{Spieltheorie:}

Ein $n$-Personen Spiel besteht aus $n$ Spielern, wobei der $i$-te Spieler $m_i$ mögliche Aktionen
ausführen kann. Die Menge der möglichen Aktionen des $i$-ten Spielers bezeichnen wir mit
$\Phi_i=\{ 1,…,m_i \}$. Nachdem jeder Spieler eine Aktion ausgeführt hat, wird abgerechnet. Der $i$-te
Spiele bekommt den Payoff
\[
    R_i(\vp), \qq \vp=(\vp_1,…,\vp_n)\in \Phi=\prod_{j=1}^n \Phi_j
\]
Wir betrachten den Fall, dass das Spiel häufig ausgeführt wird, und jeder Spieler seine Aktion nach
einer Wahrscheinlichkeitsverteilung (einer Strategie) auf $\Phi_i $ wählt. Diese bezeichenen wir mit
\[
    S_i=\{s_i^1,…,s_i^{m_i}\},\qq s_i^k\geq 0, \qq \sum_{k=1}^{m_i}s_i^k=1 \qq \forall i
\]
Die Menge aller möglichen Strategien für den $i$-ten Spieler nennen wir $S_i$. $S_i^k$ ist die
Wahrscheinlichkeit, dass der $i$-te Spieler die Aktion $k$ wählt. Die Wahrscheinlichkeit, dass jeder
Spieler eine bestimmte Aktion $s_i^k$ wählt ist
\[
    S(\vp)=\prod_{i=1}^n s_i(\vp), \qq s_i(\vp)=s_i^{k_i}
\]
mit $\vp=(k_1,…,k_n)\in \Phi$. Wir nehmen hier an, dass die Spieler ihre Aktionen unabhängig voneinander
wählen. Der mittlere Payoff für Spieler $i$ ist dann der Erwartungswert für $R_i(\vp)$. Wir nennen diesen
$R_i(s)$ und erhalten
\[
    R_i(s)=\sum_{\vp\in\Phi}s(\vp) K_i(\vp)
\]
Nach Konstruktion ist
\[
    R_i(s)
\]
stetig in $s$. Was ist nun die optimale Strategie für Spiel $i$? Falls alle anderen Spiele nach einer
bekannten  Strategie handeln, ist es optimal $\ol{s}_i$ so zu wählen, dass
\begin{align}\label{8}
    R_i(s\setminus \ol{s_i})=\max_{\tilde s_i \in S_i} R_i(s\setminus \tilde s_i)
\end{align}
$s\setminus \tilde s_i$: Strategiekomination, die sich ergibt, wenn man $s_i$ durch  $\tilde s_i$
ersetzt. Wir bezeichnen mit $B_i(s)$ die Menge aller Strategien $\ol s_i\in S_i$, die $(\ref{8})$ erfüllen.
Insbesondere gilt $\ol s_i\in B_i(s)$ genau dann wenn $\ol s_i^k=0$ falls
\[
    R_i(s\setminus (0,…,1,…)) < \max_{1<l<m_i}R_i(s\setminus(0,…,1,…)) \qq \text{mit der 1 als $l$-ter
    Eintrag}
\]
Insbesondere existiert immer eine reine (immer die selbe Aktion wird gewählt) optimale Strategie
zu gegebenem $S$. Weiter sieht man, dass $B_i(s)$ konvex ist und nicht leer. Es seien nun $s,\ol s\in S$.
Wir nennen $\ol s$ die optimale Antwort gegen $s$, falls $\ol s_i\in B_1(s)$ $i=1,…,n$.
Es sei $B(s)$ die Menge aller optimalen Antworten gegen $s$, wir haben
\[
    B(s)=\prod_{i=1}^n B_i(s).
\]

\begin{defi}[Nash-Gleichgewicht]
    Eine Strategiekombination $\ol s\in S$ heißt Nash-Gleichgewicht, falls gilt
    \[
        \ol s \in B(\ol s)
    \]
    d.h. $\ol s$ ist eine optimale Antwort gegen sich selbst. Anders gesagt, $\ol s$ ist ein 
    Nash-Gleichgewicht, falls kein Spieler durch eigene Änderung seiner Strategie seinen Payoff erhöhen
    kann.
\end{defi}

\subsubsection*{Beispiele:}

%\begin{description}
    \textbf{Das Gefangenendilemma:}
    Zwei Gefangene können entweder schweigen, oder mit der Polizei kooperieren, das heißt den jeweils
    anderen belasten.
    \[
        \begin{array}{c|cc||c|cc} R_1& k_2& s_2& R_2&k_2&s_2\\ \hline k_1&0&2&k_1&0&-1\\
            s_1&-1&1&s_1&2&1\end{array}
    \]
%\end{description}
\begin{theorem}\label{2.18}
    Jedes $n$-Personen Spiel besitzt ein Nash-Gleichgewicht.
\end{theorem}

\begin{proof}
    Wir betrachten $s\mapsto B(s)$, $S$ ist kompakt, konvex, $B(s)\subset S$ ist nicht leer, konvex für
    alle $s$. Abgeschlossenheit des Graphen folgt mit Folgenkompaktheit.
    \[
        s^n\in S, \qq \ol s^m\in B(s^m), \qq s^m\ra s, \qq \ol s^k\ra \ol s
    \]
    Wir haben $R_i(s^m\setminus \tilde s_i)\leq R_i(s^m\setminus \ol s_m)$ für alle $\tilde s_i \in S_i$
    mit Stetigkeit von $R_i$ folgt sofort
    \[
        R_i(s\setminus \tilde s_i) \leq R_i(s\setminus \ol s) \qq \forall s_i\in S_i
    \]
    Somit ist $\ol s \in B(s)$ und $\Gamma= \{ (s,B(s))\subset S^2 \}$ ist abgeschlossen.
    Der \textit{Fixpunktsatz von Kakutani} garantiert Existenz von
    \[
        \ol s\in S \qq \text{mit} \qq \ol s \in B(\ol s).
    \]
\end{proof}

\section{Der Leray-Schauder Grad und der Schauder'sche Fixpunktsatz}

Das Ziel ist es, den Fixpunktsatz von Brouwer auf $\infty$-dimensionale Räume ………

\begin{theorem}[Kakutani]\label{3.1}
    Sei $H$ ein $\infty$-dimensionaler separabler Hilbertraum. Dann existiert eine stetige Abbildung
    \[
        f:H\ra H,
    \]
    welche die Einheitenkugel in sich selbst abbildet und die keinen Fixpunkt besitzt.
\end{theorem}

\begin{proof}
    Sei $(e_n)_{n\in \Z}$ eine Orhonormalbasis von $H$. Für $x=\sum_{j=-\infty}^\infty\alpha_je_j$
    definieren wir
    \[
        U(x):=\sum_{j=-\infty}^\infty \alpha _je_{j+1};\qq f(x):=\frac12 U(x)+\frac 12 e_0
    \]
    Zu zeigen:
    \begin{description}
    \item{-}
    $U$ ist stetig.
    \item{-}
    $\|U(x)\|<1$ für $\|x\|<1$
    \item{-}
    $f$ besitzt keinen Fixpunkt in $\ol{B_1(0)}$
    \begin{itemize}
        \item bei $x=0$.
        \item auf dem Rand.
        \item im Innern.
    \end{itemize}
    \end{description}
    \[  \]
\end{proof}

\subsection{Der Abbildungsgrad auf endlichen Banachräumen}

\begin{defi}\label{3.2}
    Sei $X$ ein (reeller) Banachraum von Dimension $n$, und sei $\Phi$ ein Isomorphismus von $X$ nach 
    $\R^n$. Für $f\in D_y(\ol \Omega,X)$, $\Omega \subset X$ offen, beschränkt, $y\in X$ deinieren wir
    \[
        \deg(f,\Omega,y)=\deg(\Phi\circ f \circ \Phi^{-1},\Phi(\Omega),\Phi(y))
    \]
\end{defi}

\begin{prop}\label{3.3}
    Der in Definition \ref{3.2} definierte Abbildungsgrad ist unabhängig von der Wahl von $\Phi$.
\end{prop}

\begin{proof}
    Sei $\Psi$ ein zweiter Isomorphismus. Es gilt $A=\Psi\circ\Phi^{-1}\in \Gl(n)$. Wir schreiben
    \[
        f^\star:=\Phi \circ \phi^{-1}, \qq y^\star\in \RV(\tilde f^\star)
    \]
    und wählen wie üblich $\tilde f^\star\in C^1(\Phi(\ol\Omega),\R^n)$ aus derselben Komponente von
    $D_y(\Phi(\ol \Omega),\R^n)$ wie $f^\star$, so dass $y^\star\in \RV(\tilde f^\star)$. Dann ist auch
    \[
        A\circ \tilde f^\star \circ A^{-1}\in C^1(\Psi (\ol \Omega),\R^n)
    \]
    in derselben Komponente von $D_y(\Psi(\ol \Omega),\R^n)$ wie $A\circ f^\star\circ A^{-1}=\Psi\circ
    f\circ \Psi^{-1}$.
    Nach der Kettenregel gilt
    \begin{align*}
        J_{A\circ \tilde f^\star\circ A^{-1}}(Ay^\star)&= \det(A) J_{\tilde f^\star}(y^\star)
        \det(A^{-1})=J_{\tilde f^\star} (y^\star) \deg(\Phi\circ f \Phi^{-1},\Phi(\Omega), \Phi (y)).
    \end{align*}
    \[ \]
\end{proof}

Die Eigenschaften des Abbildungsgrades im $\R^n$ übertragen sich mit Definition \ref{3.2} auf den
Abbildungsgrad in endlichdimensionalen Banachräumen. Das gilt ebenfalls für die Reduktionseigenschaft aus
Satz \ref{2.14}, nachdem $\Phi:X\ra\R^n$ so gewählt werden kann, dass $\Phi(X_1)=\R^n$, wobei $X_1$ ein
$m$-dimensionaler Untervektorraum von $X$ ist.

\subsection{Kompakte Operatoren}

Es seien $X$ und $Y$ Banachräume, $\Omega\subset X$ (nicht unbedingt offen).

\begin{defi}
    Ein stetiger Operator $F:\Omega\ra Y$ heißt
    \begin{description}
        \item{•} \textit{endlichdimenisional}, falls $F$ ein endlichdimensionaler UVR $Y_1\subset Y$
        \item{•} \textit{kompakt}, falls $F$ beschränkte Teilmenge von $\Omega$ relativ kompakte
        Teilmengen von $Y$ abbildet.
    \end{description}
    Wir bezeichnen die Menge der endlichdimensionalen Operatoren von $\Omega$ nach $Y$ mit
    \[
        \ms F(\Omega,Y)
    \]
    und die kompakten mit
    \[
        \ms C(\Omega,Y)
    \]
\end{defi}

\begin{remark}
    \begin{description}
        \item{1)}
        Es gilt
        \[
            \ms F(\Omega,Y)\subset \ms C(\Omega,Y)\subset C(\Omega,Y)
        \]
        \item{2)}
        Falls $\Omega$ kompakt ist, so gilt
        \[
            \ms C(\Omega,Y)\subset C(\Omega,Y)
        \]
        \item{3)}
        Falls $\dim(Y)<\infty$, so gilt
        \[
            \ms F(\Omega,Y)=\ms (\Omega,Y)
        \]
        \item{4)}
        Falls $\Omega\subset\R^n$ beschränkt, so gilt
        \[
            \ms F (\ol \Omega,\R^n)=\ms C(\ol \Omega,\R^n)=C(\ol\Omega,\R^n)
        \]
    \end{description}
\end{remark}




\subsubsection*{Korrekturen}

Bei Satz \ref{2.15}. Vorsicht vor der unbeschränkten Komponente $K_\infty$ von $\R^n\setminus f(\partial
\Omega)$! Diese taucht jedoch in der Produktformel ausschließend nicht auf, denn
\[
    f(\ol \Omega)\subset B_r(0), \, \text{mit}\, f(\ol \Omega) \text{ kompakt}
\]
\[
    \Ra \, \deg(f,\Omega,K_\infty\cap B_{2r}(0))=0 \qq \text{und} \qq f(\ol \Omega) \cap (\R^n\setminus
    B_{2r}(0))=\varnothing
\]
somit können wir vereinfacht schreiben
\[
    \deg(f,\Omega,K_\infty)=0
\]
Der in der Produktformel auftauchende Term
\[
    \deg(f,\Omega,K_\infty)\cdot \deg(g,K_\infty,y)
\]
ist also unproblematisch, da er verschwindet.\\[0.5cm]

\begin{lem}\label{3.5}
    Sei $K\subset X$ kompakt, $\eps>0$. Dann existiert ein endlichdimensionaler Untervektorraum $X_\eps
    \subset X$ und eine stetige Abbildung
    \[
        P_\eps:K\ra X_\eps,
    \]
    so dass
    \[
        \|P_\eps(x)-x\|\leq \eps \qq \forall x\in K
    \]
\end{lem}

\begin{proof}
    Es sei $\{x_j\}_{j=1}^N\subset K$, so dass $\bigcup _{j=1}^NB_\eps(x_j)$ eine Überdeckung von $K$
    darstellt. Sei $\{\Phi_j\}_{j=1}^N$ eine Zerleung der Eins, welche $\{B_\eps(x_j)\}_{j=1}^N$
    untergeordnet ist. Wir definieren
    \[
        P_\eps(x):= \sum_{j=1}^N \Phi_j(x)x_j
    \]
    Damit gilt
    \[
        \|P_\eps(x)-x\|=\left\|\sum_{j=1}^N\Phi_j(x)x_j-\sum_{j=1}^N \Phi_j(x)x\right\|
        \leq \sum_{j=1}^N \Phi_j(x)\|x-x_j\|\leq \eps
    \]
    Es ist nämlich $\|x-x_j\|< \eps$ für alle $x\in B_\eps(x_j)$, aber für $\|x-x_j\|>\eps$ ist
    $\Phi_j(x)=0$.
    \[ \]
\end{proof}

\begin{theorem}\label{3.6}
    $X,Y$ Banachräume. Sei $\Omega\subset X$ beschränkt. Dann gilt
    \[
        \ol{\ms F(\Omega,Y)}^{C(\Omega,Y)}=\ms C(\Omega,Y)
    \]
    (D.h. der Abschluss von $\ms F(\Omega,Y)$ bzgl. der Norm der gleichmäßigen Konvergenz ist der Raum
     der kompakten Operatoren $\ms C(\Omega,Y)$.)
\end{theorem}

\begin{proof}
    Es sei $(F_n)_{n\in \N}$ eine Folge in $\ms C(\Omega,Y)$, sodass  $F_n\ra F$. Wir nehmen an, dass
    $F\nin \ms C(\Omega,Y)$. Dann existiert eine (beschränkte) Folge $(x_n)_{n\in \N}$, so dass
    \[
        \|F(x_k)-F(x_l)\|\geq \rho>0 \qq\text{für}\qq k\neq l
    \]
    Es sei $n$ so groß, dass
    \[
        \|F_n-F\|_{ C(\Omega,Y)}<\frac{\rho}{4}
    \]
    Damit gilt
    \[
        \|F_n(x_k)-F_n(x_l)\|\geq \frac\rho2.
    \]
    Das ist ein Widerspruch zur Annahme, dass $F_n\in \ms C(\Omega,Y)$. Somit gilt auch
    \[
        \ol{\ms F(\Omega,Y)}^{C(\Omega,Y)}\subset \ms C(\Omega,Y)
    \]
    Sei andererseits $F\in \ms C(\Omega,Y)$,und sei $K=\ol{F(\Omega)}$ kompakt. sei $P_\eps$ gewählt wie
    in Lemma \ref{3.5}. Es gilt
    \[
        F_\eps=P_\eps\circ F\in \ms F(\Omega,Y) \qq \text{und}\qq F_\eps\ra F
    \]
\end{proof}

Zur Definition des Abbildungsgrades auf unendlichdimensionalen Räumen betrachten wir insbesondere
kompakte Störungen der Identität (d.h. Operatoren der Form $\Id+F$, $F\in \ms C$). Wir bezeichnen deshalb
zunächst einige Interessante Eigenschaften von $\Id+F$.

\begin{lem}\label{3.7}
    Sei $X$ ein Banachraum und $\Omega\subset X$ beschränkt und abgeschlossen, $F\in \ms C(\Omega,X)$.
    Dann ist $\Id + F$ eine eigentliche Abbildung (d.h. $(\Id+F)^{-1}(K)$ ist kompakt für $K\subset X$ 
    kompakt), die abgeschlossene Teilmengen von $\Omega$ auf abgeschlossene Teilmengen von $X$ abbildet.
\end{lem}

\begin{proof}
    Sei $A\subset \Omega$ abgeschlossen, und sei
    \[
        y_n=(\Id+F)(x_n), \qq \text{mit } x_n \in A,
    \]
    so dass $y_n \ra y$ in $X$. Zu zeigen ist, dass $y\in (\Id+F)(A)$. Es gilt
    \[
        y_n-x_n=F(x_n),
    \]
    somit folgt $y_n-x_n\ra z$ in $X$, nach Extraktion einer Teilfolge (da $F$ kompakt, $(x_n)$
    beschränkt in $\Omega$). Es folgt $x_n\ra x \in A$, da $A$ abgeschlossen und $y_n$ konvergent
    nach Annahme. Wir haben aber $x=y-z\in A$. Nachdem aber $y=x+F(x)\in (\Id+F)(A)$ folgt die Behauptung
    mit $F(x)=z$

    Sei nun $\Omega$ abgeschlossen und $K\subset X$ kompakt. Sei $(x_n)_{n\in \N}$ eine Folge in
    $(\Id+F)^{-1}(K)$. Wir können wieder eine Teilfolge
    \[
        K\ni y_n=x_n+F(x_n)
    \]
    wählen, so dass $y_n\ra y$. Wie im vorhergehenden Teil folgt, folgt dass $x_n\ra x\in K$, somit ist
    $(\Id+F)^{-1}(K)$ kompakt.
    \[ \]
\end{proof}

\subsection{Der Leray-Schauder Grad}

Sei $\Omega\subset X$, $X$ ein Banachraum. Wir setzten
\[
    \ms D_y(\ol \Omega,X)=\{ F\in \ms C(\ol\Omega,X)\, | \, y\nin(\Id + F)(\partial \Omega) \}
\]
und
\[
    \ms G_y(\ol\Omega,X):=\{F\in \ms F(\ol \Omega,X)\, |\, y\nin(\Id+F)(\partial\Omega)\}
\]
Es gilt für $F\in \ms D_y(\ol \Omega,X)$, dass $\dist(y,(\Id+F)(\partial\Omega))>0$, da $\Id+F$
abgeschlossene Mengen auf abgeschlossene Mengen abbildet.

\begin{defi}{\label{3.8}}
    Sei $\Omega\subset X$ offen, beschränkt, $y\in X$, $F\in \ms D_y(\ol \Omega,X)$. Sei $\rho:=
    \dist(y,(\Id+F)(\partial \Omega))$ und sei
    \[
        F_1\in \ms F(\ol \Omega, X),
    \]
    sodass
    \[
        \|F-F_1\|<\rho. \qq (\Ra \, F_1\in \ms G_y(\ol \Omega,X))
    \]
    Nun sei $X_1\subset X$ ein endlichdimensionaler Untervektorraum, so dass
    \[
        F_1(\Omega)\subset X_1,\, y\in X_1
    \] 
    und sei $\Omega_1=\Omega\cap X_1$. Damit ist $F_1\in \ms G_y(\ol\Omega_1,X_1)$ und wir definieren
    \[
        \deg(\Id+F,\Omega,y):=\deg(\Id+F_1,\Omega_1,y)
    \]
\end{defi}

\begin{prop}\label{3.9}
    Die obige Definition ist unabhängig von der Wahl von $F_1$ und $X_1$.
\end{prop}

\begin{proof}
    Sei $F_2\in \ms F (\ol \Omega,X)$, $\|F_2-F\|<S$, $X_2$ entsprechend endlichdim UVR von $X$. Es sei
    $X_0=X_1+X_2$, $\Omega_0=\Omega\cap X_0$. Dann gilt
    \[
        F_i\in \ms F_y(\ol\Omega_0,X_0) \qq i=1,2
    \]
    und es folgt wegen der Reduktionseigenschaft des endlichdimensionalen Abbildungsgrades, dass
    \[
        \deg(\Id + F_i,\Omega_0,y)=\deg(\Id+F_i,\Omega_i,y)\qq i=1,2
    \]
    Sei $H(t)=\Id + (1-t)F_1+tF_2.$ Es folgt, dass
    \[
        H(t)\in D_y(\ol\Omega_0,X_0) \qq t\in [0,1],
    \]
    nachdem gilt $\|H(t)-(\Id+F)\|<\rho$ $\forall t\in [0,1]$.
    \[
        \Ra \, \deg(\Id+F_1,\Omega_0,y)=\deg(\Id+F_2,\Omega_0,y)
    \]
\end{proof}

\begin{theorem}\label{3.10}
    Sei $\Omega\subset X$ beschränkt und offen und sei $F\in \ms D_y(\ol\Omega,X)$, $y\in X$. Dann gilt
    \begin{description}
        \item{i)}
        $\deg(\Id+F,\Omega,y)=\deg(\Id+F-y,\Omega,0)$
        \item{ii)}
        $\deg(\Id, \Omega,y)=1$, falls $y\in \Omega.$
        \item{iii)}
        Falls $\Omega_1,\Omega_2$ offene, disjunkte Teilmengen von $X$ sind, so dass
        $y\nin (\Id+F)(\ol\Omega\setminus (\Omega_1\cup\Omega_2)),$ dann gilt
        \[
            \deg(\Id+F,\Omega,y)=\deg(\Id+F,\Omega_1,y)+\deg(\Id+F,\Omega_2,y).
        \]
        \item{iv)}
        Falls $H:[0,1]\times\ol\Omega\ra X$ stetig, so dass $\forall t_0$, $\forall\eps>0$ ein
        $\delta>0$ gibt mit
        \[
            \|H(t_0)-H(t)\|\leq \eps, \, \text{falls } |t-t_0|<\delta, \qq y[0,1]\ra X \text{ stetig,}
        \]
        und sei $H(t)\in \ms D _{y(t)}(\ol\Omega,X)$ $\forall t\in[0,1]$. Dann gilt
        \[
            \deg(\Id+H(0),\Omega,y(0))=\deg(\Id+H(1),\Omega,y(1))
        \]
    \end{description}
\end{theorem}

\begin{proof}
    \textit{i)-iii)} folgen sofort aus der Definition und den entsprechenden Eigenschaften des
    endlichdimensionalen Abbildungsgrades.

    \noindent\textit{iv)} 
        O.E. sei $y(t)=0$ ($\tilde H(t)(x)=H(t)(x)-y(t)$).\\
        \textbf{Behauptung:} $H\in \ms C([0,T]\times \ol\Omega,X)$
        \begin{proof}
            Sei $\eps >0,\, \forall t\in [0,1]$ sei $H_\eps^t$ eine endlichdimensionale Approximation
            von $H(t)$, so dass
            \[
                \|H_\eps^t-H(t)\|\leq\frac\eps4 \qq \forall t \qq \text{nach Satz \ref{3.6}}
            \]
            Nun sei $\{t_j\}_{j=1}^N$, so dass $\bigcup _{j=1}^N B_{j_j}(t_j)\supset [0,1]$ mit $\delta_j
            $, so dass $\|H(t_j)-H(t)\|\leq \frac\eps4$ für $t\in B_{f_j}(t_j)$. Nun sei $F_\eps$ die
            stückweise affine Interpolation von $H_\eps^{t_j}$. Damit ist $F_\eps$ endlichdimensional und
            $\|F_\eps(t)-H(t)\|<\eps$. Somit ist nach Satz \ref{3.6} $H\in \ms C([0,1]\times 
            \ol \Omega, X)$.
            \[ \]
        \end{proof}
        Mit dem selben Argument, das in Satz \ref{3.7} benutzt wird gilt, dass
        \[
            A:=\bigcup_{t\in [0,1]} (\Id+H(t))(\partial \Omega)
        \]
        ist abgeschlossen als Bild einer abgeschlossenen Menge, somit gilt
        \[
            \dist(A,y)=\rho>0
        \]
        Jetzt benutzen wir einfach die bereits vorher konstruierte endlichdimensionale Approximation 
        $F_{\frac\rho2}$ von $H$ und wenden die Homotopieinvarianz im endlichdimensionalen Fall an.
        \[ \]
\end{proof}

Auch die anderen Aussagen über den endlichdimensionalen Abbildungsgrad übertragen sich.
Die Beweise bleiben gleich.

\begin{theorem}\label{3.11}
    Seien $F,G\in \ms D_y(\ol\Omega,X)$. Es gilt
    \begin{description}
        \item{i)}
        $\deg(\Id+F,\varnothing,y)=0$. Weiter gilt, falls $\Omega_i$, $1\leq i\leq N$ offen, disjunkt,
        so dass
        \[
            y\in (\Id+F)(\ol\Omega\setminus\bigcup_{i=1}^N\Omega_i),
        \]
        folgt dass $\deg(\Id+F,\Omega,y)=\sum_{i=1}^N \deg(\Id+F,\Omega_i,y)$
        \item{ii)}
        Falls $y\nin (\Id+F)(\Omega)$, dann gilt
        \[
            \deg(\Id+F,\Omega,y)=0
        \]
        (die Umkehrung gilt nicht notwendigerweise!)

        Äquivalent dazu gilt
        \[
            \deg(\Id+F,\Omega,y)\neq 0 \, \Ra \, y\in (\Id+F)(\Omega)
        \]
        \item{iii)}
        Falls $\|F(x)-G(x)\|<\dist(y,(\Id+F)(\partial\Omega))$ $\forall x\in \partial \Omega$, so gilt
        \[
            \deg(\Id+F,\Omega,y)=\deg(\Id+G,\Omega,y).
        \]
        Insbesondere gilt das auch für den Fall, dass
        \[
            F=G\qq \text{auf } \partial \Omega
        \]
        \item{iv)}
        $\deg(\Id+\cdot,\Omega,y)$ ist konstant auf Zusammenhangskomponenten von $D_y(\ol\Omega,X)$.
        \item{v)}
        $\deg(\Id+F,\Omega,\cdot)$ ist konstant auf Zusammenhangskomponenten von $X\setminus 
        (\Id+F)(\partial \Omega)$.
    \end{description}
\end{theorem}
 %% Korrekturen, in der Vorlesung vom 23.05.11

\section{Das Leray-Schauder-Prinzip und der Schauder'sche Fixpunktsatz}

Fixpunktsätze in schneller Folge!

\begin{theorem}[Leray-Schauder-Prinzip]\label{3.12}
    Es sei $F\in \ms C(X,X)$ und es existiere $M>0$, so dass jede Lösung $x$ von
    \[
        x=tF(x) \qq t\in [0,1]
    \]
    die \textit{a priori} Abschätzung
    \[
        \|x\|\leq M
    \]
    erfüllt. Dann besitzt $F$ einen Fixpunkt.
\end{theorem}

\begin{proof}
    Sei $\rho>M$. Es gilt
    \[
        \deg(\Id-F,B_\rho(0),0)=\deg(\Id,B_\rho(0),0)=1
    \]
    Aus der \textit{a priori} Abschätzung folgt
    \[
        H(t)\in \ms D_0(B_\rho(0),X) \qq \forall t\in [0,1]
    \]
    Somit existiert eine Lösung von $x-F(x)=0$.
    \[ \]
\end{proof}

\begin{theorem}[Schauder'sche Fixpunktsatz]\label{3.13}
    Sei $K$ ein abgeschlossene, konvexe, beschränkte Teilmenge von $X$, ein Banachraum. Sei
    $F\in \ms C(K,K)$. Dann besitzt $F$ einen Fixpunkt.
\end{theorem}

\begin{proof}
    $K$ beschränkt, also sei $\rho>0$ mit $K\subset B_\rho(0)$. Es sei $R$ die stetige Fortsetzung von
    $\Id:K\ra K$ auf $X$ nach Satz \ref{2.12}. Es gilt $R:X\ra K$, $R(x)=x$ für $x\in K$.

    \noindent\textit{Anmerkung}: So eine Abbildung heißt „Retraktion“, $K$ ist dann ein Retrakt von $X$.

    \noindent Wir betrachten $\tilde F=F\circ R\in \ms C(\ol{B_\rho(0)},\ol{B_\rho(0)})$. Mit $H(t)
    -t\tilde F$ folgt
    \[
        \deg(\Id-\tilde F, B_\rho(0),0)=\deg(\Id,B_\rho(0),0)=1
    \]
    Somit existiert $x_0=\tilde F(x_0)\in K \, \Ra \, \tilde F(x_0)=F(x_0)=x_0$. 
    \[ \]
\end{proof}

\begin{remark}
    Existenz eines Fixpunktes gilt natürlich auch für $F:G\ra G$, wobei $G$ nur homöomorph zu einer
    konvexen, abgeschlossen, beschränkten Teilmenge eines Banachraumes $X$ ist.
\end{remark}

\begin{theorem}\label{3.14}
    Sei $\Omega\subset X$ offen und beschränkt, $F\in \ms C(\ol\Omega,X)$. Angenommen, es existiert ein
    $x_0\in \Omega$, so dass
    \[
        F(x)-x_0\neq \alpha(x-x_0) \qq \forall x\in \partial \Omega, \, \forall \alpha\in(1,\infty)
    \]
    Dann besitzt $F$ einen Fixpunkt.
\end{theorem}

\begin{proof}
    Sei $H(t)(x)=x-x_0-t(F(x)-x_0)$. Dann gilt aber
    \[
        H(t)\neq 0 \qq \forall x \in \partial\Omega, \, t\in [0,1)
    \]
    Falls $H(1)(x)=0$ für ein $x\in \partial \Omega$, dann ist $x$ ein Fixpunkt von $F$ und wir
    sind fertig. Ansonsten gilt allerdings
    \[
        \deg(\Id-F,\Omega,0)=\deg(\Id-x_0,\Omega,0)=\deg(\Id,\Omega,x_0)=1
    \]
    und es existiert wieder ein Fixpunkt von $F$. 
    \[ \]
\end{proof}

\begin{cor}\label{3.15}
    Sei $\Omega\subset X$ offen und beschränkt, $F\in \ms C (\ol \Omega, X)$. Dann besitzt $F$ einen
    Fixpunkt falls \textit{eine} der folgenden Bedingungen erfüllt ist:
    \begin{description}
        \item{i)}
        $\Omega=B_\rho(0)$, $F(\partial \Omega)\subset \ol\Omega$ (Rohte)
        \item{ii)}
        $\Omega=B_\rho(0)$, $\|F(x)-x\|^2\geq \|F(x)\|^2-\|x\|^2$ $\forall x\in \partial \Omega$
        (Altmann)
        \item{iii)}
        $X$ ist ein Hilbertraum, $\Omega=B_\rho(0)$, $(F(x),x)\leq \|x\|^2 \qq \forall x 
        \in \partial \Omega$ (Krasnosel'shii).
    \end{description}
\end{cor}

\begin{proof}
    Übungsaufgabe. \[ \]
\end{proof}

\noindent \textbf{Beispiele:}

\begin{description}
    \item[1. nichtlineares 2-Punkt Randwertsproblem:]
    Wir betrachten auf $[0,T]$ das Randwertsproblem
    \begin{align}\label{11}
        \left.
        \begin{array}{rcll}
        g(t,u(t),u'(t))&=&u''(t)& 0\leq t \leq T \\
        u(0)=u(1)&=&0&
        \end{array}\right\}
    \end{align}
    Es sei $C_0^2=\{u\in C^2\, | \, u(0)=u(T)=0\}$

    \textbf{Behauptung:} $L$ ist bistetig.
    \begin{proof}
        \begin{description}
            \item{Injektivität:} klar.
            \item{Surjektivität:} Beispielsweise mittels \textit{Green'scher Funktion}:
            \begin{align*}
                K(s,t)&=2T\cdot\sum_{n=1}^\infty \frac1{ \pi^2n^2 } \sin \frac{n\pi s}{T}
                \sin\frac{n\pi t}{T}\\
                u(s)&=L^{-1} (f)(s)=-\int_0^TK(s,t)f(t)\d t
            \end{align*}
            \item{Stetigkeit:} klar.
            \item{Stetigkeit:} \textit{Satz über die offene Abbildung}.
        \end{description}
        \[ \]
    \end{proof}
    \begin{theorem}
        Sei $g:[0,T]\times \R \times \R\ra \R$ eine stetige beschränkte Funktion. Dann bestizt
        (\ref{11}) eine Lösung $u\in C_0^2$.
    \end{theorem}
    \begin{proof}
        Es sei $G:C^1([0,1])\ra C([0,1])$ definiert durch $G(u)(t)=g(t,u(t),u'(t))$ und es sei
        \[
            J: C_0^2 \ra C^1([0,T])
        \]
        die Einbettung. Dann ist (\ref{11}) äquivalent zu
        \[
            L(u)=GJ(u) \, \LRa \, u = GJL^{-1}(u).
        \]
        Es gilt $JL^{-1}$ ist kompakt, das folgt aus \textit{Arzelá-Ascoli}.
        $G$ ist ein beschränkter Operator, denn $g$ ist stetig und beschränkt, 
        somit ist $GJL^{-1}$ kompakt und es existiert $\rho>0$ mit $GJL^{-1}(B_\rho(0))\subset 
        B_\rho(0)$. Damit besitzt $GJL^{-1}$ einen Fixpunket.
        \[ \]
    \end{proof}

\item[2. Existenz von Lösungen der stationären Navier-Stokes-Gleichung]
Es sei $\Omega\subset \R^3$ offen, zusammenkängend und beschränkt, und es sei $K:\Omega\ra
\R^3$ eine gegeben Funktion. Wir suchen eine Lösung $v:\Omega\ra\R^3$ zum Problem
\[
\begin{cases}
0= \eta \Delta v - v\cdot \nabla v - \nabla \rho + K & \text{ in } \Omega\\
   0= \div v & \text{ in } \Omega\\
   0= v & \text{ auf } \partial \Omega
   \end{cases}
   \]
   mit hinreichend glatten $v,\rho$.

   Eine sehr schöne Herleitung der Gleichung findet man in einem Vortrag auf der Website des Clay
   Mathematics Institute $\ra $ Link sehe Vorlesungssite.
   \end{description}

   \subsubsection*{Einschub: Sobolev-Räume}

   Wir betrachten die Menge $C^1(\Omega, \R)$ und das darauf definierte Skalarprodukt
   \[
   (u,v)_{H^1}=\int_\Omega u(x)v(x)\d x+ \int_\Omega\nabla u(x) \nabla v(x) \d x.
   \]
   Der normierte Vektorraum $(C^1(\Omega,\R),\sqrt{(\cdot,\cdot)_{H^1}})$ ist nicht vollstöndig. Die
   Vervollständigung bezeichnen wir mit $H^1(\Omega,\R)$. Die Vervollständigung von $C^1(\Omega,\R)$
   bezüglich der $H^1$-Norm bezeichnen wir mit $H_0^1(\Omega,\R)$. Die Räume $H^1$, $H^1_0$ heißen
   Sobolev-Räume und sind (separable) Hilberträume.

   Wir benötigen den folgenden wichtigen Satz aus der linearen Funktionalanalysis:

   \begin{theorem}[Rellich]\label{001}
   Die Einbettung
    \[
H_0^1(\Omega,\R)\hookrightarrow L^2(\Omega,\R)
    \]
    ist kompakt.
    \end{theorem}

    Weiter gilt die Abschätzung

    \begin{lem}[Poincaré-Friedrichs] \label{002}
    Sei $u\in H_0^1(\Omega,\R)$. Es gilt
    \[
    \int_{\Omega}u^2\leq (d_j)^2\int_\Omega(\partial_ju)^2,
    \]
    wobei $d_j=\sup\{|x_j-y_j|\, | \, (x_1,…,x_n),(y_1,…y_n)\in \Omega\}$, bzw. gilt
    \[
    \|u\|_{L^2}\leq \frac{\diam \,  \Omega}{\sqrt{n}}\|x\|_{H^1}
    \]
    \end{lem}

    \begin{proof}
    Übungsaufgabe. \[ \]
    \end{proof}

    Wir verfeinern nun das Einbettungresultat von Rellich. Dazu benötugen wir das folgende

    \begin{lem}[Ladyzhenshaya]
    Sei $\Omega\subset\R^3$. Für $u\in H_0^1(\Omega,\R)$ gilt
    \[
    \|u\|_{L^4}\leq \sqrt[4]{8} \|u\|_{L^2}^{\nicefrac14}\|\nabla u \|_{L^2}^{\nicefrac34}
    \]    
    \end{lem}

    \begin{proof}
    Wir betrachten zuerst den Fall, dass
    \[
    u^2(x_1,x_2,x_3)=\int_{-\infty}^{x_1} \partial_1 u^2(\xi,x_1,x_3)\d \xi\leq 2\int_{-\infty}
    ^{\infty}|u(\xi,x_2,x_3)\partial_1u(\xi,x_2,x_3)|\d\xi.
    \]
    Es folgt
    \[
    \max_{x_1\in\Omega}u^2(x_1,x_2,x_3)\leq2\int_{-\infty}^\infty|u\cdot\partial_1u| \d x_1
    \]
    Nun lassen wir $x_3$ fest und integrieren über $x_1$ und $x_2$:
    \begin{align*}
    \iint u^4(x_1,x_2,x_3) &\leq \int \max_{x_1} u^2(x_1,x_2,x_3) \d x_2 \cdot \int\max_{x_2}
    u^2(x_1,x_2,x_3)\d x_1\\
        &\leq 4 \iint |u\partial_1 u|  \d x_1 \d x_2 \cdot \iint|u \partial _2 u| 
        \d x_1 \d x_2\\
        &\leq 4\left( \iint u^2 \d x_1 \d x_2\right)^{\nicefrac22}\cdot\left(\iint(\partial_1u)^2
                \d x_1\d x_2\right)^{\nicefrac 12}\cdot\left( \iint(\partial_2 u)^2 \d x_1 \d x_2
                \right)^{\nicefrac 12} \qq \text{(nach Cauchy-Schwarz)}
    \end{align*}
   Jetzt integrieren wir über $x_3$ und bekommen
   \begin{align*}
   \iiint u^4\d x_1\d x_2 \d x_3 &\leq 4 \int \d x_3 \left(\iint u ^2 \d x_1 \d x_2\right) 
    \left( \iint(\partial_1u)^2 + (\partial_2 u)^2 \d x_1 \d x_2 \right)\\
        &\leq 4\left( \iint\max_{x_3} u^2(x_1,x_2,x_3) \d x_1 \d x_2 \right)\left( \iiint 
        (\partial_1 u )^2 + (\partial _2 u)^2 \d x_1 \d x_2 \d x_3\right)\\
        &\leq 8 \iiint | u \partial_3 u | \d x_1 \d x_2 \d x_3 \cdot\left(\iiint (\partial_1u)^2+
        (\partial _2 u)^2 \d x_1 \d x_2 \d x_3\right)\\
        & \leq 8\cdot \left( \iiint u^2 \right)^{\nicefrac 12}\cdot \left( \iiint (\partial_3u)^2
        \right)^{\nicefrac12}\cdot \left( (\partial_1 u)^2 + (\partial_2u)^2 \right)^{\nicefrac 22}\\
        &\leq 8 \|u\|_{L^2}\cdot\|\nabla u \|_{L^2}^3
    \end{align*}
    Sei nun  $u\in H_0^1(\Omega,\R)$. Wir wählen $(u_j)_{j=1}^\infty$ eine Folge in $C_0^1(\Omega,\R)$,
    so dass
    \[
    u_j\ra u \begin{cases} \text{ in } H_0^1\\ \text{ in } L^2  \end{cases}.
    \]
    Dies ist wegen der Ladyzhenskaya-Ungleichung eine Cauchy-Folge in $L^4$ und konvergiert somit
    gegen $v\in L^4$. Mit Hölder gilt $u=v$ und wir können den Limes in der Ungleichung bilden.
    \[ \]
    \end{proof}

    Mit \textit{Poincaré-Friedrichs} und \textit{Ladyzhenshaya} folgt sofort, dass
    \[
    \|u\|_{L^4}\leq \left( \frac{8\diam\, \Omega}{\sqrt{n}} \right)^{\nicefrac14} \cdot\|u\|_{H^1}
    \]
    für alle $u\in H_0^1(\Omega,\R)$ und damit auch

    \begin{cor}\label{004}
    Die Einbettung
    \[
H_0^1(\Omega,\R) \hookrightarrow L^4(\Omega,\R)
    \]
    ist kompakt.
    \end{cor}

    Zurück zu Navier-Stokes:

    Es sei 
    \[
    X=\{v\in C^2(\ol\Omega,\R^3) \, | \, \div \, v=0, v\big|_{\partial\Omega}=0\}
    \]
    und es sei
    \[
    \ms H:=\ol{X}^H_1 = \{v\in H_0^1(\Omega,\R^3) \, | \, \div \, v =0\}
    \]
    Wir betrachten $\ms H$ als Hilbertraum mit dem Skalarprodukt
    \[
    (u,v)_{\ms H}= \int_\Omega\nabla u \cdot \nabla v \d x
\]
Dieses ist mittels \textit{Poincaré} äquivalent zum üblichen Skalarprodukt auf $H^1$:

Wir multiplizieren nun die Navier-Stokes-Gleichung \ref{12} mit $w\in X$ und erhalten

\begin{align*}
    \int_\Omega (\eta\Delta v - v\nabla v + K)w &= \int_\Omega \nabla \vp  \cdot w =0 \\
    \LRa \, \int_\Omega (\eta \underbrace{\nabla v \cdot \nabla w}_{=\sum_{k,j=1}^3 
    \partial_jv_k\partial_k w_j} -\underbrace{v\cdot v \cdot \nabla w}_{=\sum_{k,j=1}^3
    v_kv_j(\partial_kw_j)}- K\cdot w)&=0 
\end{align*}
mit
\[
    a(u,v,w):= \int_\Omega u\cdot(v\cdot \nabla w)
\]
folgt $u$ Lösung von Navier-Stokes
\begin{align}\label{13}
    \Ra \, \eta\cdot (v,w)_{\ms H} - a(v,v,w)- \int_\Omega Kw =0 \qq \forall x \in \ms H
\end{align}

Lösungen von (\ref{13}) heißen schwache Lösungen der Navier-Stokes-Gleichung.

\begin{remark}
    Es gilt
    \begin{align*}
        a(v,v,v)&=\int_\Omega v(v\cdot \nabla v)=\frac12 \int_\Omega\sum_{k,j} v_k\partial_k(v_j,v_j)
        =-\frac12 \int_\Omega\sum(v_j,v_j)\partial_kv_k=0 \qq \text{für } v\in \ms H 
    \end{align*}
\end{remark}
Für $K\in L^2 (\Omega,\R^3)$ ist $\int_\Omega K \cdot$ ein stetiges Funktional auf $\ms H$, somit
existiert nach \textit{Riesz} ein $\tilde K \in \ms H$, so dass
\[
    \int K\cdot w= (\tilde K, w)  \qq \forall w\in \ms H
\]
Dies gilt auch für $a(u,v,\cdot)$ mit $u,v\in \ms H$, somit existiert $B(u,v) \in \ms H$ mit
\[
    a(u,v,w)=(B(u,v),w)_{\ms H}.
\]
Es folgt, dass
\begin{align*}
    (\ref{13}) \, &\LRa \, (\eta v - B(v,v)- \tilde K, w)_{\ms H}=0 \qq \forall w\in \ms H\\
        &\LRa \, \eta v - B(u,v)=\tilde K.
\end{align*}
Es sei nun $Y=L^4(\Omega,\R^3)$. Dank \textit{Ladyzhenshaya-Ungleichung} ist die Einbettung
\[
    \ms H \hookrightarrow Y
\]
kompakt und es gilt mit zweimaliger Anwendung der \textit{Cauchy-Schwarz-Ungleichung}:
\[
    |a(u,v,w)|\leq\|u\|_{L^4}\|v\|_{L^4}\|w\|_{H^1}
\]


\begin{theorem}[Existenz schwacher Lösungen]\label{005}
Sei $\ms H$ ein Hilbertraum, $Y$ ein Banachraum und es sei die Einbettung $\ms H \hookrightarrow Y $
kompakt. Insbesondere sei
\[
    \|u\|_Y\leq \beta \|u\|_{\ms H} \qq \forall u \in \ms H
\]
Es sei $a: \ms H^3 \ra \R$ eine Multilinearform, so dass
\begin{align*}\label{14}
    |a(u,v,w)|\leq \|u\|_Y\|v\|_Y\|w\|_{\ms H}
\end{align*}
und $a(v,v,v)=0$ für alle $v\in \ms H$. Es sei $\tilde K\in \ms H$, $\eta >0$. Dann existiert
$v\in \ms H$ mit
\[
    \eta(v,w)- a(v,v,w)= (\tilde K, w)_{\ms H} \forall w \in \ms H
\]
\end{theorem}

\begin{proof}
    o.E. sei $\eta=1$. Wir suchen eine Lösung zu $v-B(v,v)+\tilde K=0$. Es gilt
    \begin{align*}
        \|B(v,v)\|_{\ms H}& \leq \alpha \|u\|_Y\|v\|_Y\leq \alpha \beta^2 \|u\|_{\ms H}\|v\|_{\ms H}
    \end{align*}
    Es sei $F(v)=B(v,v)$. $F$ ist lokal Lipschitz, denn für $\|u\|_Y< \rho,$ $\|v\|_Y<\rho$ gilt
    \begin{align*}
        \|F(u)-F(v)\|_{\ms H} &= \|B(u-v,u)-B(v,u-v)\|_{\ms H}\leq 2\alpha \rho \|u-v\|_Y\\
            &\leq 2\alpha\beta^2 \rho \|u-v\|_{\ms H}.
    \end{align*}
    Sei $(v_n)_{n\in \N}$ beschränkt in $\ms H$. Nach Wahl einer Teilfolge ist $v_n$ eine Cauchyfolge
    in $Y$. $\Ra$ $F(v_n)$ ist eine Cauchyfolge in $\ms H$, da gilt
    \begin{align*}
        \|F(u)-F(v)\|_{\ms H}\leq 2\alpha \rho\|u-v\|_Y
    \end{align*}
    Damit ist $F\in \ms C(\ms H,\ms H)$. 

    Nehmen wir nun an, dass $v$ eine Lösung ist von
    \[
        v=tF(v)-t\tilde K \qq \text{für } t\in [0,1]
    \]
    Damit gilt
    \[
        (v,v)_{\ms H} = t\underbrace{a(v,v,v)}_{=0} + t(\tilde K, v)_{\ms H} \, \Ra \,
        \|v\|_{\ms H}\leq \|\tilde K\|_{\ms H}.
    \]

    Die beiden Voraussetzungen des \textit{Leray-Schauder-Prinzips} sind damit erfüllt,
    somit existiert ein Fixpunkt $v$, der $ v=F(v)-\tilde K$ erfüllt. \[ \]
\end{proof}

\section{Monotone Operatoren}

Man möchte folgendes Prinzip auf Banachräume verallgemeinern $F:\R\ra\R$ genüge
\begin{description}
    \item{(a)}
    monoton
    \item{(b)}
    stetig
    \item{(c)}
    $F$ ist koerziv, d.h. $\lim_{v\ra \pm \infty} F(v)=\pm \infty$
\end{description}
Dann
\[
    \forall b\in \R \, \exists ! v\in \R: \, F(v)=b
\]

\begin{defi}\label{4.1}
    Sei $X$ ein reflexiver Banachraum und
    \[
        A:X\ra X^\star
    \]
    ein Operator. A heißt
    \begin{description}
        \item{(i)}
        monoton $\Ra$ $\forall u,v\in X:$ $(Au-Av,u-v)\geq 0$
        \item{(ii)}
        strikt monoton $\Ra$ $\forall u\neq v\in X:$ $(Au-Av,u-v)>0$
        \item{(iii)}
        stark monoton
        \[
            \Ra \exists c>0 \, \forall u,v\in X:\, (Au-Av,u-v)\geq c\|u-v\|^2
        \]
        \item{(iv)}
        koerziv
        \[
            \Ra \, \lim_{\|u\|\ra \infty}\frac{(Au.u)}{\|u\|}=\infty
        \]
    \end{description}
\end{defi}

\begin{remark}
    $A$ stark monoton $\Ra$ $A$ koerziv.
\end{remark}

\begin{proof}
    \[
        \frac{(Av,v)}{\|v\|}=\frac{ (Au-A(0), v) + (A(0),v) }{\|v\|}\geq c\|v\|-\|A(0)\|
    \]
\end{proof}

\subsubsection*{Beispiele:}

\begin{description}
    \item{1.}
    Sei $f:\R\ra\R$. Das Dualitätsproblem in $\R$ ist die Multiplikation. Im Spezialfall $X=\R=X^\star$
    stimmen die Monotoniebegriffe reeller Funktionen und für Operatoren überein, da
    \[
        (f(u)-f(v),u-v)=(f(v)-f(u))\cdot(u-v)\geq 0
    \]
    $f$ ist außerdem genau dann koerziv, d.h.
    \[
        \lim_{v\ra \pm \infty} \frac{f(u)\cdot u}{|u|}=\infty \, \LRa \, f(v)\ra \pm \infty\qq(v\ra\pm
                \infty)
    \]
    \item{2.}
    $g:\R\ra \R$
    \[
        g(u):=\begin{cases} |u|^{p-2}u, & u\neq 0\\ 0, & v=0 \end{cases}
    \]
    $\Ra$ \begin{description}
        \item{(i)}
        $p>1 \, \Ra \, g$ strikt monoton.
        \item{(ii)}
        $p\geq 2 \, \Ra \, (g(u)-g(v),u-v)\geq c|u-v|^p$
        \item{(iii)}
        $p=2 \, \Ra \, g$ stark monoton
    \end{description}
    \begin{proof}
    Übungsaufgabe. \[ \]
    \end{proof}
\end{description}

\begin{defi}
    Sei $X$ reflexiver Banachraum und $A:X\ra X^\star$ ein Operator $A$ heißt
    \begin{description}
        \item{(i)}
    \end{description}
\end{defi}

\begin{remark}
    stark stetig $\Ra$ stetig $\Ra$ demistetig $\overset{\Ra}{(\star)}$ hemistetig
\end{remark}

\begin{proof}
    Zu $(\star)$. Seien $u,v,w\in X$, $t\ra t_0$ $\Ra$ $u+tv \ra u+tv$
    \[
        \overset{\Ra}{\text{demistetig}} \, A(u+tv) \rightharpoonup A(u+t_0v)
    \]
    \[
        (A(u+tv),w)\ra (A(u+t_0v),w)
    \]
\end{proof}

\begin{lem}\label{4.4}
    $X$ reflexiver Banachraum, $A:X\ra X^\star$. Dann gelten
    \begin{description}
        \item{(i)}
        $A$ stark stetig $\Ra$ $A$ kompakt.
        \item{(ii)}
        $A$ demistetig $\Ra$ $A$ lokal beschränkt.
        \item{(iii)}
        $A$ monoton $\Ra$ $A$ lokal beschränkt.
        \item{iv}
        $A$ monoton und hemistetig $\Ra$ $A$ demistetig.
    \end{description}
\end{lem}

\begin{proof}
    \begin{description}
    \item{(i)}
        Sei $(v_n)_{n\in \N}\subset X$ beschränkt, da $X$ reflexiv folgt:
        \[
            \exists v^{m_k} \rightharpoonup v \in X\overset{\text{stark stetig}}{\Ra} A(v_{m_k})\ra A(v)
        \]
        $\Ra$ $A$ kompakt.
    \item{(ii)}
    Wöre $A$ nicht lokal beschränkt, so existiert $v\in X$, $v_m\ra v$, sodass
    \[
        \|Av_m\|\ra \infty \lightning  \qq Av_m \rightharpoonup Av
    \]
    \item{(iii)}
    Angenommen, $v_m\ra v: \, \|Av_m\|\ra \infty$
    \[
        a_n:=(1+\|Av_n\|\|v_n-v\|)^{-1}
    \]
    $A$ monoton 
    \begin{align*}
        \Ra \, 0&\leq (Au_n-Av,u_n-v)= (Au_n-Av,(u_n-v)+(u-v))\\
        a_n(Au_n,v-u)&\leq a_n((Au_n,u_n-v)-(Av,u_n-v))\\
        &\leq 1+ c(u,v) \qq \forall v\in X
    \end{align*}
    Einsetze $v\leadsto 2u-v$
    \[
        -a_n (Au_n,v-u)\leq 1+c(u,v) \qq \forall v \in X
    \]
    \[
        \Ra \, |(a_n A u_n,v-u)|\leq 1+c(u,v)
    \]
    $\Ra \, (a_nAu_n)_{n\in \N}$ ist punktweise beschränkt.
    \[
        \Ra \, \sup_n\|a_nAu_n\|_{X^\star}\leq c(u)
    \]
    \[
        \Ra \, \|Au_n\|\leq \frac{c(u)}{a_n} = c(u)(1+\|Au_n\|\|u_n-v\|)
    \]
    Für
    \[
        \|u_n-u\|c(u)=\lambda <1: \, \|Au_n\|\leq \frac{c(u)}{1-\lambda}<\infty
    \]
    \item{(iv)}
    Gelte $u_n\ra u$. $A$ monoton $\Ra$ $(Au_n)$ beschränkt.
    $X^\star$ reflexiv
    \[
        Au_{n_k}\rightharpoonup b \in X^\star
    \]
    \[
        Au=b
    \]
    Dies gilt für jede Teilfolge von $u_n, \Ra  Au_n \rightharpoonup b$
    \end{description}
    \[ \]
\end{proof}

\begin{lem}[Minty]\label{4.5}
    Sei $X$ ein refelexiver Banachraum und $A:X\ra X^\star$, hemistetig und monoton. Dann gelten
    \begin{description}
        \item{(i)}
        $A$ maximal monoton, d.h.
        \begin{align*}
            v\in X, \, b\in X^\star, \, (b-Av,u-v) \geq 0 \qq \forall v\in X\\
                \Ra \,b=Au
        \end{align*}
        \item{(ii)}
        $A$ genügt der Bedingung (M), d.h.
        \[
            Au=b \La \begin{cases} v_n \rightharpoonup v \\ Av_n\rightharpoonup b \\ (Av_n,v_n)\ra (b,v)
        \end{cases}
        \]
        \item{(iii)}
        \[
            u_n \rightharpoonup u, \, Au_n\ra b \, \Ra \, Au=b
        \]
        \[
            u_n\ra u \,\wedge \, Au_n \rightharpoonup b \, \Ra \, Au=b
        \]
    \end{description}
\end{lem}

\begin{proof}
    \begin{description}
    \item{(i)}
    Sei $w\in X$ und $v=v-tw, \, t>0$.
    \[
        \Ra \,(b-A(u-tw),w)\geq 0
    \]
    \[
        \Ra\,(b-Au, w)\geq 0 \qq w\in X
    \]
    \[
        \Ra \,(b-Av,w)=0 \forall w
    \]
    \[
        b=Aw
    \]
    \item{(ii)}
    $A$ ist monoton.
    \begin{align*}
    \Ra \, 0&\leq (Au_n-Av,u_n-v) = (Au_n,u_n)-(Av,u_n) - (Au_n,Av,v)\\
    &\ra (b,v)-(Av,u)-(b-Av,v)
    \end{align*}
    \end{description}
\end{proof}

\begin{theorem}[Brouder, Minty]\label{4.6}
    Sei $X$ ein separabeler, reflexiver Banachraum mit Basis $(w_i)_{i\in \N}$. Sei $A:X\ra X^\star$
    monoton, koerziv und hemistetig. Dann existiert für alle $b\in X^\star$ eine Lösung $u\in X$ von
    \[ Au=b \]
    $\{A=b\}$ ist abgeschlossen, beschränkt, und konvex. Falls $A$ strikt monoton ist, ist die Lösung
    eindeutig.
\end{theorem}

\textsc{Beweisidee:} Der Beweis erfolgt per \textit{Galerkin-Approximation}.
\begin{description}
    \item{1.}
    $X$ separabel, d.h.
    \[
        X=\bigcup_{n=1}^\infty X^n, \qq X_n=\mr{span}(w_1,…,w_n)
    \]
    Approximiere „die Gleichung“ $Au=b$ durch „endlichdimensionale“ Probleme der Form
    \[
        (Au,z^n)=(b,z^n)    \forall z^n\in X_n
    \]
    \item{2. \textit{A-priori}-Abschätzung für Lösungen:}
    Wir zeigen, dass die Folge $(v_n)_{n\in \N}$ der Lösungen dieser Probleme beschränkt ist (mit Hilfe
            der Koerzivität).
    \item{3. Schwache Konvergenz}
    $X$ reflexiv $\Ra$ $(v_n)_{n\in \N}$ hat eine schwach konvergente Teilfolge
    \[
        u_{n_k}\rightharpoonup u \in X
    \]
    \item{4.}
    Lemma von Minty $\Ra$ Au=b.
    
\end{description}

\begin{proof}
    \begin{description}
    \item{(i)}
    Der Beweis erfolgt per \textit{Galerkin-Approximation}
    \[
        X_n=\mr{span}(w_1,…,w_n)
    \]
    Betrachte
    \begin{equation}\label{20}
        (Au^n-b,w_k)=0 \qq \forall k=1,…,n
    \end{equation}
    Definiere
    \[
        g:\R^n\ra \R^n; \qq g_k(c^n):= (A(\sum_mc_m^nw_m)-b,w_k)
    \]
    \[
        \ref{20} \, \Ra \, g(c^n)=0 \qq \text{für ein } c^n\in\R^n  
    \]
    O.B.d.A sei $|c^n|:= \|\sum_{k=1}^n c^n_kw_k\|_X$ als norm auf $\R^n$. Aus Lemma \ref{4.4} folgt,
    dass $A$ demistetig. 

    Außerdem gilt
    \[
        \sum_{k=1}^n g_k(c^n)c^n_k=(Au^n,u^n)-(b,u^n), \qq \text{wobei } u^n=\sum_kg_k(c^n)w_k^n
    \]
    $A$ ist koerziv
    \[
        \Ra \, R_0>0 \qq \forall \|u\|\geq R_0>0: \qq (Au,u)\geq (1+\|b\|)\|v\| =0
    \]
    \begin{align*}
        \Ra \, g(c^n)\cdot c^n &= \sum_{k=1}^ng_k(c^n)\cdot c_k^n \geq (1+\|b\|)\|u^n\| - \|b\|\|u^n\|
        =\|u^n\|>0 \qq \forall c^n \text{ mit } \|u^n\|\geq R_0
    \end{align*}
    Mit \textit{Brouwer} folgt (beachte: $|\cdot|_{\R^n}=\|\|_X$)
    \[
        \exists u^n\in X_n: \, g(c^n)=0,
    \]
    d.h. \ref{20} gilt für $u^n$.
    \[
        \Ra \, \|u^n\|\geq R_0
    \]
    \item{(ii)}
    $A$ monoton $\Ra$ $A$ lokal becshränkt
    \[
        \Ra \, \exists r,\delta>0: \, \|u\|\leq r \, \Ra \, \|Au\|\leq \delta
    \]
    und
    \[
        (Au^n-A^v,u^n-v)\geq 0 \qq \forall v\in X
    \]
    Wegen $u^n\in X_n$ und \ref{20} gilt
    \[
        (Au^n,u^n)=(b,u^n)\, \Ra \, |(Au^n,u^n)|=\|b\|R_0
    \]
    \begin{align*}
        \|Au^n\|&= \sup_{v\in X, \, \|v\|=r} \frac1r(Au^n,v) \overset{\text{Monotonie}}{\leq} 
        \sup_{v\in X, \|v\|=5} \frac1r ((Av,v)+(Au^n,u^n)-(Av,u^n)) \leq \frac1r (\delta r+ \|b\|R_0
                +\delta R_0)
    \end{align*}
    $\Ra $ $(Au^n)_{n\in \N}$ ist beschränkt.
    \item{(iii)}
    Konvergenz des Galerkin-Verfahrens; $\|u^n\|\leq R_0$ und $X$ reflexiv.
    \[
        \Ra \, \exists \text{ Teilfolge } \, u^n\rightharpoonup u \,\in X
    \]
    Sei 
    \[
        w\in \bigcup_{m=1}^\infty X_m \, \Ra \, \exists n_0\in\N: \, w\in X_n\qq n\geq n_0
    \]
    \begin{align*}
        \Ra \, \forall n\geq n_0: \, (Au^n,w)&=(b,w)\\
        \Ra \, \lim_{n\ra \infty} (Au^n,w)=(b,w) \qq w\in \bigcup_{m=1}^\infty X_m
    \end{align*}
    $X^\star$ reflexiv und $(Au^n)$ beschränkt
    \begin{align*}
        \exists Au^{n_m}\rightharpoonup c \in X^\star\\
        \Ra \, \forall w\in X: \, (Au^{n_k},w)\ra (c,w)\\
        \Ra \, b=c, \qq \text{weil } \ol{\bigcup_{m=1}^\infty X_m}=X \qq (X \text{ separabel})\\
        \Ra Au^{n_m} \rightharpoonup b
    \end{align*}
    Wiederhole dieses Argument für jede Teilfolge von $Au^n$.
    \[
        \Ra \, Au^n\rightharpoonup b
    \]
    \begin{align*}
        \Ra \, (Au^n,u^n)&=(b,u^n)\ra (b,u)\\
        \lim_{n\ra \infty} (Au^n,u^n)&= (b,u)
    \end{align*}
    Mit Lemma \ref{4.5} (ii) erhalten wir
    \[ 
        Au=b
    \]
    \item{(iv)}
    Eigenschaften der Lösungsmenge $S=\{A=b\}$:
    \begin{description}
    \item{(a) Beschränktheit:}
    \[
        \exists \, R_0>0 \, \forall \|V\|\geq R_0>0: \, (Au,u)\geq (1+\|b\|)\|u\|
    \]
    Sei $u$ eine Lösung mit $\|u\|\geq R$
    \begin{align*}
        \Ra \, 0&=(Au,u) -(b,u)\geq \|u\|>0 \, \lightning\\
        \Ra \, S\subset B_{R_0}(0)
    \end{align*}
    \item{(b) Konvexität:}
    Seien $u_1,u_2\in S$, $w=tu_1+(1-t)u_2$, $v\in X$.
    \begin{align*}
        (b-Av,w-v)&=(b-Av,t(u_1-v))+(b-Av,(1-t)(u_2-v))\\
                 &= t(Au_1-Av,u_1-v)+(1-t)(Au_2-Av,u_2-v)\\
                 &\geq 0 \qq \forall v\in X\\
        \overset{\text{Lem. \ref{4.5}}}\Ra \, b&=Aw \, \Ra \, w\in S 
    \end{align*}
    \end{description}
    \end{description}
    \[ \]
\end{proof}

\begin{remark}
    Der Satz gilt auch, falls $X$ nicht separabel ist (ohne Beweis).
\end{remark}

\begin{cor}\label{4.7}
    Sei $X$ separabel, refl. $\R$-Banachraum und $A:X\ra X^\star$ strikt monoton, koerziv, hemstetig.
    Dann existiert
    \[
        A^{-1}:X^\star\ra X
    \]
    und $A^{-1}$ it trikt monoton und demistetig.
\end{cor}

\begin{proof}
    Übung. \[ \]
\end{proof}

\section{Der Nemyckii-Operator}

\begin{defi}\label{4.8}
    Sei $G\subset \R^n$, $f:G\times \R^n\ra \R^k$. Durch Anwendung auf $u:G\ra \R^n$ definieren
    wir den Operator $F$ als
    \begin{align}
        (Fu)(x)=f(x,u(x)), \qq x\in G
    \end{align}
\end{defi}

\begin{lem}\label{4.9}
    Die Funktion $f$ erfülle
    \begin{description}
        \item{1. Carathéodory-Bedingung:}
        \[
            f(\cdot,\eta): x \ra f(x,\eta) \qq \text{messbar } \forall \eta\in \R^n
        \]
        \[
            f(x,\cdot):\eta\ra f(x,\eta) \qq \text{stetig f.f.a. } x\in G
        \]
        \item{2.Wachstums-Bedingung:}
        \[
            |f^j(x,\eta)|\leq |a(x)|+ b \sum_{i=1}^n|\eta^i|^{\nicefrac{p_i}{q}}
        \]
        mit $b>0$, $a\in L^q(G)$ und $1\leq p_i$, $q<\infty$, $i=1,…,n$. 
    \end{description}
    Dann ist $F$ aus Definition \ref{4.8} ein Operator der Form
    \[
        F:\prod_{i=1}^n L^{p_i}(G) \ra (L^q(G))^k
    \]
    und $F$ ist stetig sowie beschränkt. Es gilt
    \[
        \|Fu\|_{L^q}\leq c\left( \|a\|_{L^q} + \sum_{i=1}^n \|u^i\|_{L^{p_i}}^{\nicefrac{p_i}{q}} \right)
    \]
    $F$ heißt in diesem Fall \textit{Nemyckii-Operator}.
\end{lem}

\begin{proof}
    (nur für $n=1$, $k=1$, $p=p_1$, $u=u^1$)
    \begin{description}
    \item{1. Messbarkeit von $Fu$:}
    Es sei $u\in L^p$, also messbar. Wir approximieren $u$ durch
    $(u_j)_{j=1}^n$, so dass
    \[
        u_j\ra u \text{ auf } G, \, u_j=\sum_{i=1}^{N_j} c_i^j \chi_{G_i^j}
    \]
    Es gilt
    \[
        (Fu)(x)= \lim_{j\ra \infty} f(x,u_j(x)) \text{für fast alle } x
    \]
    wegen Stetigkeit von in der 2. Variablen. Weiter gilt
    \[
        f(x,u_j(x))=\sum_{i=1}^{N_j} f(x,c_i^j) \chi_{G_i^j(x)}
    \]
    Damit ist $f(x,u_j(x))$ messbar für alle $j\in \N$ als Produkt von messbaren Funktionen.
    Somit ist $f(x,u(x))$ auch messbar als punktweiser Limes von messbaren Funktionen.

    \item{2. Beschränktheit von $F$:}
    \begin{align*}
        \|Fu\|_{L^q}^q &= \int_G|f(x,u(x))|^q= \int(|a(x)| + b |u(x)|^{\nicefrac pq})\\
            &\leq C\left( \|a(x)\|_{L^q}^q+\|u\|_{L^p} \right) \qq \text{nach Young}.
    \end{align*}
    \item{3. Stetigkeit von $F$:}
    \[u_j\ra u \qq \text{in } L^p\]
    \[
        \Ra \, \exists \text{ Teilfolge :} \, u_{j_k}\ra u \qq \text{f.ü.}
    \]
    \[
        \Ra \, (Fu_{j_k})(x) \ra (Fu)(x) \qq \text{f.f.a. } x\in G \text{ (wegen Stetigkeit)}
    \]
    Aber es gilt zudem, dass
    \[
        \int_G|f(x,u_{j_k}(x)) - f(x,u(x))|^q 
        \leq C \underbrace{(\|u\|^q_{L^q} + \|u_{j_k}\|_{L^p}^q+\|u\|_{L^p}^q)}_{\text{beschränkt.}}
    \]
    Es gilt
    \begin{align*}
        |f(x,u_{j_k}(x)) - f(x,u(x))|^q &\leq C(|f(x,u_{j_k}(x))|+ |f(x,u(x))|^q)\\
        &\leq C(|a(x)|^q + b^q|u_{j_k}|^q + |f(x,u(x))|^q) =: h_{j_k}(x)
    \end{align*}
    Es folgt
    \[
        \|Fu_{j_k}-Fu\|_{L^q}^q\leq \int_G h_{j_k}
    \]
    mit
    \[
        h_{j_k}\ra h \qq \text{in } G; \qq \int_{G}h_{j_k} \ra \int_G h,
    \]
    denn $\|u_{j_k}\|_{L^p}\ra \|u\|_{L^p}$. Jetzt folgt mit majorisierter Konvergenz, dass
    \[
        F(u_{j_k})(x)\ra (Fu)(x) \text{ in } L^p
    \]
    Die selbe Rechnung mit gleichem Limes folgt allerdings auch unter vorheriger
    Auswahl einer Teilfolge von $u_j$, somit gilt auch für die gesamte Folge, dass
    \[
        \|Fu_j-Fu\|\ra 0 \qq (j\ra \infty)
    \]
    \end{description}\[ \]
\end{proof}

\subsubsection*{Anwendung: $p$-Laplace}

Wir betrachten das Randwertproblem:
\begin{align}\label{23}
    \begin{split}
    -\div\left( |\nabla u|^{p-2} \nabla u \right) + su &=f \qq \text{auf } \Omega\\
    u&=0\qq \text{auf } \partial \Omega
    \end{split}
\end{align}
für $1<p<\infty$, $\Omega\subset\R^n$ offen, beschränkt, $s\geq 0$, $u=0$ auf $\partial\Omega$.

\noindent Schwache Formulierung: Es sei $f\in L^p (\Omega)$

Wir suchen $u\in X= W^{1,p}_0(\Omega)$, so dass
\begin{align}\label{22}
    \int_\Omega |\nabla u|^{p-2} \nabla u \nabla \vp + s\int_{\Omega}u\vp=\int f\vp \qq \forall \vp \in X
\end{align}

Wir definieren einen Operator $A$ durch
\[
    \lal Au,\vp\ral = (Au)(\vp) := \int_\Omega |\nabla u|^{p-2}\nabla u \nabla \vp +\int su\vp \qq 
    \forall u,\vp \in X
\]
und ein Funktional $b$ durch
\[
\lal b,\vp \ral = \int_\Omega f\cdot \vp
\]

\begin{lem}\label{4.10}
    Es sei $p\geq \frac{2n}{n+2}$ $f\in L^{p'}$, dann gilt
    \[
        A:X\ra X'
    \]
    beschränkt und $b\in X'$. Weiter gilt, dass (\ref{22}) äquivalent ist zu
    \[
        Au=b
    \]
\end{lem}

\begin{proof}
    Wir benutzen $\|u\|_X=\|\nabla u \|_{L^p}$ als äquivalente Norm auf $X$ mittels \textit{Poincaré}.
    \begin{description}
        \item{1)}
        Es gilt
        \begin{align*}
            |\lal Au,\vp \ral| &\leq \int_\Omega |\nabla u |^{p-1}|\nabla \vp| + s\int_\Omega |u\vp|\\
            &\overset{\text{Hölder}}{\leq} \left( \int_\Omega |\nabla u|^{(p-1)p'} \right)
            ^{\nicefrac1{p'}}\cdot \left( \int_\Omega |\nabla \vp|^p \right)^{\nicefrac 1p}
            +s \|u\|_{L^2}\|\vp\|_{L^2}\\
            &\overset{p'=\frac{p}{p-1}}{=} \|\nabla u\|_{L^p}^{p-1} \|\nabla \vp\|_{L^p} +
            s \|u\|_{L^2}\|\vp\|_{L^2}= ……
        \end{align*}
        Für $p\geq \frac{2n}{n+2}$ bettet $W^{1,p}$ genau stetig in $L^2$ ein. Es gilt somit, dass
        \[
            ……\leq C (\|\nabla u \|_{L^p}^{p-1}+s\|\nabla u\|_{L^p})\|\nabla\vp\|_{L^p}
        \]
        \[
            \Ra \, Au\in X'
        \]
        und
        \begin{align*}
            \|Au\|_{X'}&\leq \sup_{\|\vp\|_X\leq1} |\lal Au,\vp \ral| \leq C \|\nabla u\|_{L^p}^{p-1}
            =C\|\nabla u\|_X^{p-1}
        \end{align*}
        \item{2)}
        Gleichermaßen gilt
        \[
            \|b\|_{X'} \leq C\|f\|_{L^{p'}}
        \]
        \item{3)}
        Es folgt, dass
        \[
            (\ref{22}) \, \LRa \, \lal Au,\vp\ral= \lal b,\vp\ral \qq \vp \in X
        \]
        Das ist nichts anderes als
        \[
            Au=b
        \]
    \end{description}\[ \]
\end{proof}

\begin{lem}\label{4.11}
    Es seien die Voraussetzungen von Lemma \ref{4.10} gegeben, $A$ definiert nach (\ref{22}). Dann sind
    die Voraussetzungen von Theorem \ref{4.6} gegeben.
\end{lem}

\begin{proof}
    \begin{description}
    \item{1. Monotonie:}
    Es sei $g=(g^1,…,g^n):\R^n\ra \R^n$ und
    \[
        g:\xi \mapsto |\xi|^{p-2} \xi, \qq g(0)=0  
    \]
    Damit folgt, dass
    \[
        \frac{\partial g^i}{\partial \xi^j}(\xi)=|\xi|^{p-2} \delta_i^j + (p-2)|\xi|^{p-4}\xi_i\xi_j
        \qq i,j=1,…,n \qq \xi\neq 0.
    \]
    Somit gilt aber, dass
    \begin{align*}
        \sum_{i,j=1}^n \frac{\partial q^i}{\partial \xi _j}(\xi) \eta^i\eta^j&= |\xi|^{p-2}
        \left( |\eta|^2+ (p-2)\frac{(\xi \eta)^2}{|\xi|^2} \right)
        \geq \min(1,p-1) |\xi|^{p-2} |\eta|^2
    \end{align*}
    Nun sei $u\neq v \in X.$ es folgt, dass
    \begin{align*}
        \lal Au-Av,u-v\ral &= \int_\Omega (g(\nabla u)- g(\nabla v))(\nabla u - \nabla v) + s \int_\Omega
        |u-v|^2\\
            &\geq \int_\Omega \underbrace{\int_0^1 \frac{\d}{\d \tau} g(\nabla v + \tau(\nabla u -
                        \nabla v)) \d\tau\cdot(\nabla u-\nabla v)}_{:= I}  \d x\\
    \end{align*}
    Es gilt:
    \begin{align*}
        I&\geq \int_0^1 \sum_{i,j=1}^n\frac{\partial g^i}{\partial\xi_j}(\nabla v + \tau(\nabla u-\nabla v))
        \cdot (\partial_ju-\partial_jv)(\partial_iu-\partial_iv)\d \tau\\
        &\geq  c |\nabla u - \nabla v|^2\underbrace{\int_0^1 |\nabla v+\tau(\nabla u -\nabla v)|^{p-2}\d\tau}
        _{<\infty \text{ für }p>1} >0,
    \end{align*}
    denn $|\nabla v + \tau (\nabla u - \nabla v)|^{p-2}>0$ außer für max. 1 Punkt $\tau_0(x)$. Somit gilt
    aber
    \[
        \lal Au-Av, u-v \ral >0 \qq \Ra \, A \, \text{strikt monoton.}
    \]
    \item{2. Koerzitivität:}
    Es sei $u\in X$. Es gilt
    \[
        \lal Av,u\ral= \int_\Omega |\nabla u|^p + s|u|^2\geq \int_\Omega |\nabla u|^p
    \]
    \[
        \frac{\lal Au,u \ral}{\|u\|_X} \geq \|\nabla u\|^{p-1}_{L^p} \ra \infty \qq \text{für }\|u\|_X
        \ra \infty \, \text{und } p>1.
    \]
    \item{3. Stetigkeit:}
    Für die Funktion $g$ aus 1. gilt
    \[
        |g^i(\xi)|\leq c |\xi|^{p-1} = c |\xi|^{\nicefrac pq} \qq \text{für } q=\frac{p}{p-1}
    \]
    und $g$ ist stetig. Somit ist 
    \[
        F:(L^p)^n\ra (L^{p'})^n, \qq  F:\nabla u \mapsto g(\nabla u)
    \]
    ein $n$-dimenisonaler \textit{Nemyckii-Operator}. Die Stetigkeit von $A$ ist gegeben durch
    \[
        \lal Au,\vp\ral=\int_\Omega \underbrace{F(\nabla u)}_{\in L^{p'}} \cdot \underbrace{\nabla \vp}
        _{\in L^p} +\underbrace{s\int_\Omega u \vp}_{\text{beschränkt.}}
    \]
    folgt durch Normabschätzung. Damit ist $A$ stetig und somit hemistetig.
    \item{4.}
    $W_0^{1,p}$ ist separabler Banachraum, somit sind alle Voraussetzungen von \ref{4.6} erfüllt
    \end{description}
    \[ \]
\end{proof}


\end{document}
