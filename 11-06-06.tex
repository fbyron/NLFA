\subsection{Der Nemychii-Operator}

\begin{defi}\label{4.8}
    Sei $G\subset \R^n$, $f:G\times \R^n\ra \R^k$. Durch Anwendung auf $u:G\ra \R^n$ definieren
    wir den Operator $F$ als
    \begin{align}
        (Fu)(x)=f(x,u(x)), \qq x\in G
    \end{align}
\end{defi}

\begin{lem}\label{4.9}
    Die Funktion $f$ erfülle
    \begin{description}
        \item{1. Carathéodey-Bedingung:}
        \[
            f(\cdot,\eta): x \ra f(x,\eta) \qq \text{messbar } \forall \eta\in \R^n
        \]
        \[
            f(x,\cdot):\eta\ra f(x,\eta) \qq \text{stetig f.f.a. } x\in G
        \]
        \item{2.Wachstums-Bedingung:}
        \[
            |f^j(x,\eta)|\leq |a(x)|+ b \sum_{i=1}^n|\eta^i|^{\nicefrac{p_i}{q}}
        \]
        mit $b>0$, $a\in L^q(G)$ und $1\leq p_i$, $q<\infty$, $i=1,…,n$. 
    \end{description}
    Dann ist $F$ aus Definition \ref{4.8} ein Operator der Form
    \[
        F:\prod_{i=1}^n L^{p_i}(G) \ra (L^q(G))^k
    \]
    und $F$ ist stetig sowie beschränkt. Es gilt
    \[
        \|Fu\|_{L^q}\leq c\left( \|a\|_{L^q} + \sum_{i=1}^n \|u^i\|_{L^{p_i}}^{\nicefrac{p_i}{q}} \right)
    \]
    $F$ heiqt in diesem Fall \textit{Nemychii-Operator}.
\end{lem}

\begin{proof}
    (nur für $n=1$, $k=1$, $p=p_1$, $u=u^1$)
    \begin{description}
    \item{1. Messbarkeit von $Fu$:}
    Es sei $u\in L^p$, also messbar. Wir approximieren $u$ durch
    $(u_j)_{j=1}^n$, so dass
    \[
        u_j\ra u \text{ für auf } G, \, u_j=\sum_{i=1}^{N_j} c_i^j \chi_{G_i^j}
    \]
    Es gilt
    \[
        (Fu)(x)= \lim_{j\ra \infty} f(x,u_j(x)) \text{für fast alle } x
    \]
    wegen Stetigkeit von in der 2. Variablen. Weiter gilt
    \[
        f(x,u_j(x))=\sum_{i=1}^{N_j} f(x,c_i^j) \chi_{G_i^j(x)}
    \]
    Damit ist $f(x,u_j(x))$ messbar für alle $j\in \N$ als Produkt von messbaren Funktionen.
    Somit ist $f(x,u(x))$ auch messbar als punktweiser Limes von messbaren Funktionen.

    \item{2. Beschränktheit von $F$:}
    \begin{align*}
        \|Fu\|_{L^q}^q &= \int_G|f(x,u(x))|^q= \int(|a(x)| + b |u(x)|^{\nicefrac pq})\\
            &\leq C\left( \|a(x)\|_{L^q}^q+\|u\|_{L^p} \right) \qq \text{nach Young}.
    \end{align*}
    \item{3. Stetigkeit von $F$:}
    \[u_j\ra u \qq \text{in } L^p\]
    \[
        \Ra \, \exists \text{ Teilfolge :} \, u_{j_k}\ra u \qq \text{f.ü.}
    \]
    \[
        \Ra \, (Fu_{j_k})(x) \ra (Fu)(x) \qq \text{f.f.a. } x\in G \text{ (wegen Stetigkeit)}
    \]
    Aber es gilt zudem, dass
    \[
        \int_G|f(x,u_{j_k}(x)) - f(x,u(x))|^q \leq C \underbrace{(\|u\|^q_{L^q} + \|u_{j_k}\|_{L^p}^q+\|u\|_{L^p}^q+)}
        _{\text{beschränkt.}}
    \]
    Mit dem Satz über domenierte Konvergenz folgt
    \[
        \|Fu_{j_k}-Fu\|_{L^p}\ra 0.
    \]
    \begin{align*}
        |f(x,u_{j_k}(x)) - f(x,u(x))|^q &\leq C(|f(x,u_{j_k}(x))|+ |f(x,u(x))|^q)\\
        &\leq C(|a(x)|^q + b^q|u_{j_k}|^q + |f(x,u(x))|^q) =: h_{j_k}(x)
    \end{align*}
    Es folgt
    \[
        \|Fu_{j_k}-Fu\|_{L^q}^q\leq \int_G h_{j_k}
    \]
    mit
    \[
        h_{j_k}\ra h \qq \text{in } G; \qq \int_{G}h_{j_k} \ra \int_G h,
    \]
    denn $\|u_{j_k}\|_{L^p}\ra \|u\|_{L^p}$. Jetzt folgt mit majorisierter Konvergenz, dass
    \[
        F(u_{j_k})(x)\ra (Fu)(x) \text{ in } L^p
    \]
    Die selbe Rechnung mit gleichem Limes folgt allerdings auch unter vorheriger
    Auswahl einer Teilfolge von $u_j$, somit gilt auch für die gesamte Folge, dass
    \[
        \|Fu_j-Fu\|\ra 0 \qq (j\ra \infty)
    \]
    \end{description}\[ \]
\end{proof}

\subsubsection*{Anwendung: $p$-Laplace}

Wir betrachten das Randwertproblem:
\begin{align}\label{23}
    \begin{split}
    -\div\left( |\nabla u|^{p-2} \nabla u \right) + su &=f \qq \text{auf } \Omega\\
    u&=0\qq \text{auf } \partial \Omega
    \end{split}
\end{align}
für $1<p<\infty$, $\Omega\subset\R^n$ offen, beschränkt, $s\geq 0$, $u=0$ auf $\partial\Omega$.

\noindent Schwache Formulierung: Es sei $f\in L^p (\Omega)$

Wir suchen $u\in X= W^{1,p}_0(\Omega)$, so dass
\begin{align}\label{22}
    \int_\Omega |\nabla u|^{p-2} \nabla u \nabla \vp + s\int_{\Omega}u\vp=\int f\vp \qq \forall \vp \in X
\end{align}

Wir definieren einen Operator $A$ durch
\[
    \lal Au,\vp\ral = (Au)(\vp) := \int_\Omega |\nabla u|^{p-2}\nabla u \nabla \vp +\int su\vp \qq 
    \forall u,\vp \in X
\]
und ein Funktional $b$ durch
\[
\lal b,\vp \ral = \int_\Omega f\cdot \vp
\]

\begin{lem}\label{4.10}
    Es sei $p\geq \frac{2n}{n-2}$ $f\in L^{p'}$, dann gilt
    \[
        A:X\ra X'
    \]
    beschränkt und $b\in X'$. Weiter gilt, dass (\ref{22}) äquivalent ist zu
    \[
        Au=b
    \]
\end{lem}

\begin{proof}
    Wir benutzen $\|u\|_X=\|\nabla u \|_{L^p}$ als äquivalente Norm auf $X$ mittels \textit{Poincaré}.
    \begin{description}
        \item{1)}
        Es gilt
        \begin{align*}
            |\lal Au,\vp \ral| &\leq \int_\Omega |\nabla u |^{p-1}|\nabla \vp| + s\int_\Omega |u\vp|\\
            &\overset{\text{Hölder}}{\leq} \left( \int_\Omega |\nabla u|^{(p-1)p'} \right)
            ^{\nicefrac1{p'}}\cdot \left( \int_\Omega |\nabla \vp|^p \right)^{\nicefrac 1p}
            +s \|u\|_{L^2}\|\vp\|_{L^2}\\
            &\overset{p'=\frac{p}{p-1}}{=} \|\nabla u\|_{L^p}^{p-1} \|\nabla \vp\|_{L^p} +
            s \|u\|_{L^2}\|\vp\|_{L^2}= ……
        \end{align*}
        Für $p\geq \frac{2n}{n-2}$ bettet $W^{1,p}$ genau stetig in $L^2$ ein. Es gilt somit, dass
        \[
            ……\leq C (\|\nabla u \|_{L^p}^{p-1}+s\|\nabla u\|_{L^p})\|\nabla\vp\|_{L^p}
        \]
        \[
            \Ra \, Au\in X'
        \]
        und
        \begin{align*}
            \|Au\|_{X'}&\leq \sup_{\|\vp\|_X\leq1} |\lal Au,\vp \ral| \leq C \|\nabla u\|_{L^p}^{p-1}
            =C\|\nabla u\|_X^{p-1}
        \end{align*}
        \item{2)}
        Gleichermaßen gilt
        \[
            \|b\|_X' \leq C\|f\|_{L^{p'}}
        \]
        \item{3)}
        Es folgt, dass
        \[
            (\ref{22}) \, \LRa \, \lal Au,\vp\ral= \lal b,\vp\ral \qq \vp \in X
        \]
        Das ist nichts anderes als
        \[
            Au=b
        \]
    \end{description}\[ \]
\end{proof}

\begin{lem}\label{4.11}
    Es seien die Voraussetzungen von Lemma \ref{4.10} gegeben, $A$ definiert nach (\ref{22}). Dann sind
    die Voraussetzungen von Theorem \ref{4.6} gegeben.
\end{lem}

\begin{proof}
    \begin{description}
    \item{1. Monotonie:}
    Es sei $g=(g^1,…,g^n):\R^n\ra \R^n$ und
    \[
        g:\xi \mapsto |\xi|^{p-2} \xi, \qq g(0)=0  
    \]
    Damit folgt, dass
    \[
        \frac{\partial g^i}{\partial \xi^j}(\xi)=|\xi|^{p-2} \delta_i^j + (p-2)|\xi|^{p-4}\xi_i\xi_j
        \qq i,j=1,…,n \qq \xi\neq 0.
    \]
    Somit gilt aber, dass
    \begin{align*}
        \sum_{i,j=1}^n \frac{\partial q^i}{\partial \xi _j}(\xi) \eta^i\eta^j&= |\xi|^{p-2}
        \left( |\eta|^2+ (p-2)\frac{(\xi \eta)^2}{|\xi|^2} \right)
        \geq \min(1,p-1) |\xi|^{p-2} |\eta|^2
    \end{align*}
    Nun sei $u\neq v \in X.$ es folgt, dass
    \begin{align*}
        \lal Au-Av,u-v\ral &= \int_\Omega (g(\nabla u)- g(\nabla v))(\nabla u - \nabla v) + s \int_\Omega
        |u-v|^2\\
            &\geq \int_\Omega \underbrace{\int_0^1 \frac{\d}{\d \tau} g(\nabla v + \tau(\nabla u +
                        \nabla v)) \d\tau\cdot(\nabla u-\nabla v)}_{:= I}  \d x\\
    \end{align*}
    Es gilt:
    \begin{align*}
        I&\geq \int_0^1 \sum_{i,j=1}^n\frac{\partial g^i}{\partial\xi_j}(\nabla v + \tau(\nabla u-\nabla v))
        \cdot (\partial_ju-\partial_jv)(\partial_iu-\partial_iv)\d \tau\\
        &\geq  c |\nabla u - \nabla v|^2\underbrace{\int_0^1 |\nabla v+\tau(\nabla u -\nabla v)|^{p-2}\d\tau}
        _{<\infty \text{ für }p>1} >0,
    \end{align*}
    denn $|\nabla v + \tau (\nabla u - \nabla v)|^{p-2}>0$ außer für max. 1 Punkt $\tau_0(x)$. Somit gilt
    aber
    \[
        \lal Au-Av, u-v \ral >0 \qq \Ra \, A \, \text{strikt monoton.}
    \]
    \item{2. Koerzitivität:}
    Es sei $u\in X$. Es gilt
    \[
        \lal Av,u\ral= \int_\Omega |\nabla u|^p + s|u|^2\geq \int_\Omega |\nabla u|^p
    \]
    \[
        \frac{\lal Au,u \ral}{\|u\|_X} \geq \|\nabla u\|^{p-1}_{L^p} \ra \infty \qq \text{für }\|u\|_X
        \ra \infty \, \text{und } p>1.
    \]
    \item{3. Stetigkeit:}
    Für die Funktion $g$ aus 1. gilt
    \[
        |g^i(\xi)|\leq c |\xi|^{p-1} = c |\xi|^{\nicefrac pq} \qq \text{für } q=\frac{p}{p-1}
    \]
    und $g$ ist stetig. Somit ist 
    \[
        f:(l^P)^n\ra (L^p)^n, \qq  F:\nabla u \mapsto g(\nabla u)
    \]
    ein $n$-dimenisonaler \textit{Nemychii-Operator}. Die Stetigkeit von $A$ ist gegeben durch
    \[
        \lal Au,\vp\ral=\int_\Omega \underbrace{F(\nabla u)}_{\in L^{p'}} \cdot \underbrace{\nabla \vp}
        _{\in L^p} +\underbrace{s\int_\Omega u \vp}_{\text{beschränkt.}}
    \]
    folgt durch Normabschätzung. Damit ist $A$ stetig und somit hemistetig.
    \item{4.}
    $W_0^{1,p}$ ist seperabler Banachraum, somit sind alle Voraussetzungen von \ref{4.6} erfüllt
    \end{description}
    \[ \]
\end{proof}
