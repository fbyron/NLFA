\subsection{Verallgemeinerung der Determinantenformel}

Wir haben einen Kandidaten für $\deg$ identifiziert; für $f\in D'_Y(\ol\Omega,\R^n)$, $y\in \R^n\cap
\RV(f)$ gilt
\[
    \deg(f,\Omega, g)=\sum_{x_j\in f^{-1}(y)} \sign J_f(x_j).
\]

\noindent \textbf{Problem:} Nur für glatte $f$, und reguläre Punkte zu definieren. Programm zum 
Existenzbeweis: Verallgemeinerung der Determinanten-Formel auf
\begin{description}
    \item{-} kirtische Werte $y$.
    \item{-} nur stetige Funktionen $f$.
\end{description}

\noindent Drei Beispiele zur Illustration:
\begin{enumerate}
    \item $f(x)=x^2$, $U=(-1,1)$
    \item $f(x)=x^2$, $U=(-1,2)$
    \item $f(x)=x+2\sin(x)$, $U=(-10,10)$
\end{enumerate}
Wie aus den Beispielen ersichtlich ist, haben wir noch (kleine) Schwierigkeiten, den Abbildungsgrad
an kritischen Werten von $f$ zu definieren. Stetige Fortsetzungen liegt aber nahe. Das funktioniert
aber nur falls es von diesen nicht „zu viele“ gibt.

\subsubsection*{Schritt 1:} Kritische Werte von $f$:

\begin{lem}[Sard]\label{2.5}

Es sei $f\in C^1(\Omega, \R^n)$, $\Omega\subset\R^n$ sei offen und beschränkt. Dann ist
$\CV(f)$ eine Lebesgue-Nullmenge.
\end{lem}
    
\begin{proof}
    Klar, falls $f$ eine affine Abbildung ist. (Dimension des Bildraumes der (konstanten) Ableitung!)
    Wir linearisieren. 

    \noindent Es sei $\CP(f):=\{x\in \Omega\,|\, J_f(x)=0\}$ die Menge der kritischen Punkte von $f$.
    Sei $\{ Q_j \}_{j\in \N}$ eine abzählbare, offene Überdeckung von $\Omega$ bestehend aus Würfeln,
    so dass
    \[
        \ol \Omega_i\subset \Omega, \, i\in \N.
    \]
    Es gilt 
    \[
        \CV(f)=f(\CP(f))=\bigcup_{j\in \N} f(\CP(f)\cap Q_j)
    \]
    Es reicht also zu zeigen, dass
    \[
    |f(\CP(f)\cap Q_j)|
    \]
    für alle $j\in \N$ verschwindet. Sei nun $Q$ ein solcher Würfel, $\rho$ dessen Kantenlänge.
    Sei $\eps >0$, und sei $Q$ unterteilt in $N^n$ Würfel $Q^i$ der Seitenlänge $\frac{\rho}{N}$, so
    dass
    \begin{align}\label{4}
        |f(x)-f(\tilde x)-f'(\tilde x)(x-\tilde x)|&\leq \int_0^1|f'(\tilde x+t(x-\tilde x))-
        f'(\tilde x)|\cdot|x-\tilde x|\d t\leq \frac{\eps\rho}{N} \qq \text{für alle} 
            \qq x,\tilde x \in Q^i
    \end{align}
    So ein $N$ existiert, nachdem $f'$ auf $Q$ gleichmäßig stetig ist. Nun enthalte $Q^i$ einen
    kirtischen Punkt $\tilde x_i \in \CP(f)$, ohne Einschränkung sei $\tilde x_i=0$, $f(\tilde x_i)=0$,
    und setzen $M=f'(\tilde x _i)$. Wegen $\det M =0$ existiert eine ONB $\{b^j\}_{j=1}^n$ mit
    \[
    b^n \bot \Bild(M)
    \]
    Weiter gilt 
    \begin{align*}
        Q^i&\subset \left\{ \sum_{j=1}^n \lambda _j b^j \, \big| \, \| \lambda \|_2\leq \sqrt{n} 
            \frac{\rho}{N} \right\}\subset\left\{\sum_{j=1}^n\lambda _jb^j \, \big|
            \, |\lambda_j|\leq \sqrt{n}\frac{\rho}{N} \, \forall 1\leq j\leq n\right\}
    \end{align*}
    Damit existiert $C>0$, (unabhängig von $i$), so dass
    \[
        MQ^i\subset \left\{\sum_{j=1}^{n-1}\lambda_jb^j\, \big| \, |\lambda_j|
            \leq C\cdot \frac{\rho}{N} \right\}
    \]
    mit $C=\sqrt{n} \max_{x\in \ol Q}|f'(x)| $. Damit gilt nach (\ref{4}) sogar, dass
    \[
        f(Q^i)\subset\left\{ \sum_{j=1}^n \lambda_jb^j\, \big| \, |\lambda_j|\leq 
        (C+\eps)\frac{\rho}{N} \, \forall 1\leq j\leq n-1, \, |\lambda_n|\leq \frac{\eps\rho}{N}\right\}
    \]
    Es folgt
    \[
        |f(Q^i)|\leq \frac{\tilde C\eps}{N^n},
    \]
    falls in $Q^i$ ein kritischer Punkt liegt. Es gibt maximal $N^n$ Unterwürfel $Q^i$ mit kritischen
    Punkten. Somit gilt
    \[
        |f(Q\cap \CP(f))|\leq C\cdot \eps
    \]
\end{proof}

\noindent Dank Lemma \ref{2.5} ist $\R^n\setminus \CV(f)$ dicht in $\R^n$. Das reicht leider (?)
noch nicht.

\noindent Wir brauchen $d_1=d_2$, um den Abbildungsgrad sinnvoll durch die Determinanten-Formel
definieren zu können. (denn $\deg$ soll konstant sein auf Zusammenhangskomponenten, unabhängig
von krit. Werten.)

\textbf{Idee:} Umschreiben der Determinantenformel als \textit{Integral}.
Es sei im Weiteren $\eta_\eps$ ein Standard-Mollifier (Standard-Dirarcfolge), d. h.
\[
    \eta_\eps \in C^\infty_c(B_\eps(0)\subset \R^n), \qq \int_{\R^n} \eta_\eps=1,\qq \eta_\eps\geq 0
\]

\begin{lem}\label{2.6}
    Sei $f\in D^1_y(\ol \Omega, \R^n)$, $y\nin \CV(F)$. Dann gilt
    \begin{align}\label{5}
        \deg(f,\Omega,y)=\sum_{x_j\in f^{-1}(y)}\sign J_f(x_j)=\int_\Omega \eta_\eps (f(x)-y) J_f(x)\d x
    \end{align}
    für alle $\eps$ hinreichend klein, d. h.
    \[
        \eps_0>\eps>0, \qq \text{mit} \qq \eps_0=\eps_0(f,y)
    \]
    Es gilt $\supp (\eta_\eps(f(\cdot)-y))\subset \Omega$ für $\eps < \dist(y,f(\partial \Omega))$
\end{lem}

\begin{proof}
    \begin{description}
        \item{1)} 
        Falls $f^{-1}(y)=\varnothing$, dann sei $\eps_0=\dist(y,f(\partial \Omega))$
        \item{2)}
        Falls $f^{-1}(y):= \{x^i\}_{1\leq i \leq N}$, sei $\eps_0>0$, so dass
        \[
            f^{-1}(B_{\eps_0}(y)) = \bigcup_{i=1}^N U(x_i)=:\bigcup_{i=1}^N U_i
        \]
        mit $U_i\cap U_j=\varnothing$ für $i\neq j$.
        Aus dem \textit{Satz über die imliziete Funktion} (war ja klar, dass wir den mal brauchen)
        folgt für evtl. noch kleiners $\eps_0$, dass
        \[
            f\big|_{U_i} \qq \text{bijektiv,} \qq J_f(x)\neq 0 \qq \forall x \in U_i.
        \]
        Wieder gilt $\eta _{\eps}(f(x)-y)=0$ für
        \[
            x\in \ol \Omega \setminus \bigcup _{i=1}^N U_i
        \]
        Damit gilt
        \begin{align*}
            \int_\Omega \eta_\eps(f(x)-y) J_f(x) \d x&= \sum_{i=1}^N \sign J_f(x^i)\cdot
            \int_{B_{\eps_0(x)}}\eta_\eps (\tilde x ) \d \tilde x\\
            &=\sum_{x^i\in f^{-1(y)}} \sign  J_f(x^i)
        \end{align*}
        mit $f(x)=\tilde x$.
    \end{description}
    \[ \]
\end{proof}
Die Integraldarstellung ergibt auch für kritische Punkte Sinn. Aber:
Wegen der Abhängigkeit von $\eps_0$ von $f$ und $y$ ist auch der Wert des Integrals nicht
\textit{a priori} stetig in $f,y$, denn $\eps_0$ hängt ab von $f$ und $y$.

