\chapter{Einleitung}

\section{Thema der Vorlesung}

In der linearen Funktionalanalysis haben wir eine Vielzahl von Methoden kennengelernt um
Ergebnisse aus der endlichdimensionalen linearen Algebra auf den unendlichdimensionalen Fall 
zu verallgemeinern. Ein Hauptaufgabe war dabei, die Lösbarkeit von Gleichungen der Form
\[
    Ax=y
\]
für lineare Operatoren $A$ auf $\infty$-dimensionalen Banchräumen zu zeigen.
\begin{description}
    \item{Ein Beispiel:}
    
    Sei $\Omega\subset\R^3$, gefüllt mit einer inkompressiblen, riskosen Flüssigkeit.
    $v_j(x)$ sei die Geschwindigkeit der Flüssigkeit an der Stelle $x\in \Omega$. 
    $p(x)$ ist der Druck an der Stelle $x$.
    \begin{description}
    \item{Randbedingungen:}
    $v_j(x)=0 \qq x\in \partial\Omega$
    \item{Inkompressibilität:}
    $\partial_jv_j(x)=0 \qq x\in \Omega$
    \item{Bewegungsungleichung:}
    Wir betrachten die Kräfte, die auf einen kleinen Würfel, eingeschlossen durch
    $(x_1,x_2,x_3)$, $(x_1+\Delta x_1, x_2+\Delta x_2, x_3+\Delta x_3)$. Druck auf eine Oberfläche
    des Würfels mit Normale $x$, ist $f_j^p=p\cdot \Delta x_2 \Delta x_3 \cdot \delta_{ij} $ auf die
    gegenüberliegende Seite wirkt
    \[
        -(p+\partial _i p \Delta x_i) \Delta x_2 \Delta x_3 \delta_{ij}
    \]
    \end{description}
Zusammen ergibt sich 
\[
    f=(\partial_j p)\Delta V
\]

Kraft durch Viskosität auf eine Oberfläche mit Normale $x_1$ ist 
\[
    f_j^{V,x_i}= -2\eta \Delta x_2\Delta x_3 \partial_1v_j
\]
mit einer Konstante $\eta$. Der gleiche Trick wie oben ergibt für die gegenüberliegende
Oberfläche
\[
    \eta \Delta x_2 \Delta x_3 \partial (v_j+\partial v_j\Delta x_1)
\]
Zusammen ergibt sich
\begin{align*}
    f_j^V&= \eta\Delta V \cdot \partial_i\partial_iv_j\\
    \text{Newton:} \qq \rho \Delta V \frac{\d}{\d t} v_j (t,x(t))
        &= \eta \Delta V  \partial _i\partial_iv_j- \Delta V (\partial _j p) + \Delta 
        V\kappa_j 
\end{align*}
mit einer externen Konstante $\kappa_j$, bspw. Gravitation.
Teilen durch $\Delta V$ und die Kettenregel ergibt
\begin{align*}
    \rho \partial_t v_j&= \eta \partial_i\partial_i v_j - \rho (v_i\partial_i) - \partial_j p + K_j 
        \qq \text{(Newton)}\\
    d_jv_j&=0
\end{align*}
Navier-Stokes. Frage: Existiert eine eindeutige Lösung zur sationären Navier-Stokes-Gleichung:
\begin{align}\label{1}
    \eta \partial_i\partial_iv_j-(v_i\partial_i)v_j+\partial_jp+K_j=0
\end{align}
\end{description}
Wir können die Gleichung etwas umschreiben. Sei $H$ ein Hilbertraum
\[
    H:=\ol{\left\{ u\in C^\infty_c (\Omega,\R^3), \text{ so dass } \partial_jv_j=0 \right\}}
        ^{W^{1,2}(\Omega, \R^3)}
\]
Ein Skalarprodukt auf $H$ ist gegeben durch
\begin{align*}
    (u,v)_H:= \int_{\Omega} \nabla u\cdot \nabla v \d x = \int_\Omega \sum_{i,j=1}^3 
    (\partial_iu_j)(\partial_iv_j)
\end{align*}
$\Omega$ beschränkt $\Ra$ $(\cdot, \cdot)_H$ ist äquivalent zum üblichen Skalarprodukt 
(mittels Poincaré).
Wir multiplizieren (\ref{1}) mit $\omega \in H $, integrieren und erhalten
\begin{align*}
    \int_\Omega (\eta \partial_i\partial_i v_j- (v,\partial_i) v_j + K_j) \cdot \omega_j 
    &= \int_\Omega (\partial _jp)\omega_j=0 \text{, da $\omega$ divergenzfrei.}\\
    (\ref{1}) \Ra \eta (v,\omega)_H-a(v,v,w)-\int_\Omega K\omega&=0
\end{align*}
Ebenso für $a$:
\begin{align*}
    a(u,v,w):= (\underbrace{B(u,v)}_{\text{bilinear.}},w)_H
\end{align*}
Also
\begin{align*}
    (\ref{1}) \Ra \, (\eta v - B(u,v)) - \tilde K , \omega)_H=0 \qq \forall \omega \in H
\end{align*}
somit
\[
    \eta v - B(v,v) = \tilde K
\]
Das ist eine Gleichung der Form
\begin{align}\label{2}
    Fv=\tilde K,\qq\text{mit $F$ einem Nichtlinearen Operator}
\end{align}
Im ersten Teil der Vorlesung beschäftigen wir uns mit der eindeutigen Lösbarkeit von
Gleichungen der Form
\[
    Fx=y, \qq F:X\ra Y, \qq X,Y\qq \text{Banachräume}
\]
und zum Abschluß zeigen wir mit Hilfe des \textit{Schauder'schen Fixpunktsatzes} die Existenz
und finden eine Lösung von (\ref{2}), also der schwachen Form der stationären 
Navier-Stokes-Gleichung.

Im zweiten Teil der Vorlesung beschäftigen wir uns mit Variationsrechnung (d.h. dem Finden
von Minimierern nichtlinearer Funktionalen)
\[
    W:X\ra \R \qq\text{mit} \qq X \qq \text{ein Banachraum}
\]
Finde 
\[
    x_0\in X: \, W(x_0)= \inf_{y\in X}W(y)
\]
Insbesonder treffen wir dort auf Probleme in der Elastizitätstheorie.

\section*{Aufbau der Vorlesung}
\begin{itemize}
    \item Abbildungsgrad $\ra$ Existenz von Lösungen von $Fx=y$
    \item Monotone Operatoren $\ra$ Eindeutigkeit von Lösungen von $Fx=y$; 
    	zeitabhängige Probleme.
    \item Variationsrechnung $\ra$ $\inf_{y\in X} W(y)$
\end{itemize}
