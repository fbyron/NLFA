Nun ist $A$ demistetig (als Operator $A:V\ra V$), da $A$ pseudomonoton und beschränkt (Lemma \ref{4.16}),
$b\in C(\ol I;V')$. Damit ist $f^n:\ol I\times \R^n \ra \R^n$ eine stetige Abbildung. Nun die Lösbarkeit
von (\ref{34})  auf $[0,T]$ zu zeigen, benötigen wir noch eine \textit{a priori} Abschätzung:

\noindent \textbf{Behauptung:}

\begin{align}\label{35}
\left.\begin{array}{lrl}
i)  & \,\|u_n\|^2_{C(\ol I;H)} + \|u_n\|_X^p &\leq c (b,u_0) \\
ii) & \|Au_n\|_{X'}&\leq c (b,u_0)
\end{array}\right\}
\end{align}
unabhängig von $n$.

\begin{proof}
    Sei also $u_n\in C^1(\ol I;V_n)$ eine Lösung von \ref{34}. Wir multiplizieren die $j$-te Gleichung in
    (\ref{34}) mit $C^n_j(t)$, summieren über $j$, und integriere von $0$ bis $t\leq T$. Mit (\ref{33})
    folgt
    \begin{align*}
        \frac12 \|u_n(t)\|_H^2-\cancel{\frac12\|u_0^n\|^2_H}+ c_0\int_0^t\|u_n(s)\|_V^p\d s&\leq
        \int_b^t\|b_n(s)\|_{V'} \|u_n(s)\|_V\d s + \frac12\|u_0\|_H^2\\
        &\overset{\text{Young}}{\leq} \frac12\|u_0^n\|_H^2+c(u_0,p) \int_0^T \|b_n\|_{V'}^p\d s +
        \frac{c_0}2 \int_0^t \|u_n(s)\|_V^p\d s
    \end{align*}
    Es gilt aber $u_0^n\ra u_0$ in $H$, $b_n\ra b_0$ in $X'$, somit folgt (\ref{35} i). Wegen der
    Beschränktheit. Wegen der Beschränktheit von $A$ folgt auch (\ref{35} ii). Damit ist die Behauptung
    bewiesen.\[ \]
\end{proof}
Aus (\ref{35} i) folgt, dass für alle $n\in\N$ eine Konstante $c_1(b,u_0,u)$ existiert mit
\[
    |c^n(t)|\leq c_1 \qq \forall t\in \ol I
\]
Wir setzen
\[
    K:= \sup_{\ol I \times \ol{B_{2c_1}(0)}} |f^n|
\]
und erhalten mit Hilfe des \textit{Satzes von Peano} die Lösbarkeit des Galerkin-Systems auf dem
Intervall $[0,\min(T,\frac{2c_1}K)]$. Diese Lösung lässt sich auf ganz $I$ fortsetzen, da das
Existenzintervall nur von $n$, $b$,$u_0$ abhängt. Damit ist das Galerkin-System für alle $n$ auf ganz
$I$ (modulo „Klebepunkte“ der lokalen Lösung) lösbar, und $u_n$ erfüllt die \textit{a priori}
Abschätzung (\ref{35} i).
\item{2)}
Konvergenz des Galerkin-Verfahrens: Aus (\ref{35} i) folgt für Teilfolgen von $u_n$, dass
\begin{align*}
    u_n\rightharpoonup u &\qq \text{in }X\\
    Au_n\rightharpoonup \xi &\qq \text{in }X'\\
    u_n(T) \rightharpoonup u^* &\qq \text{in }H
\end{align*}
Es sei nun $w\in \bigcup_{k=1}^\infty V_k$, somit $w\in V_{k_0}$. Aus der Galerkin-Gleichung erhalten wir
für alle $n\geq n_0$
\[
    \Lal \frac{d-u_n(t)}{\d t}, w\Ral_V+\Lal Au_n(t),w\Ral = \Lal b_n(t),w\Ral_V
\]
Wir multiplizieren mit $\vp\in C^1(\ol I, \R)$ und integrieren in der Zeit. Es folgt mit (\ref{34}), dass
\[
    -\int_0^T (u_n(t),w)_H\vp'(t)\d t + \int_0^T \lal Au_n,w\ral_V \vp(t)\d t = \int_0^T\lal b_n(t),w
    \ral_V\vp (t)\d t - (u_n(T),w)_H\vp(T) + (u_0^n,w)_H\vp(0).
\]
Wegen schwacher Konvergenz von $u_n$, $Au_n$, $u_n(T)$ und starker Konvergenz von $u_0^n$, $b_n$ folgt
\begin{align}\label{37}
    -\int_0^T (u(t),w)_H \vp'(t)\d t + \int_0^T \lal \xi, w\ral _V \vp (t)\d t = 
    \int_0^T\lal b(t), w \ral _V \vp (t)\d t - (u^*,w)_H \vp (T) + (u_0, w )_H \vp (0).
\end{align}
Wegen Dichtheit von $\bigcup_{k=1}^\infty V_k$ in $V$ gilt (\ref{37}) auch für \textit{alle} $w\in V$ 
und für alle $\vp\in C^1(\ol I. \R)$, also insbesondere auch für $\vp \in C_c^\infty (I,\R)$. Nach der
Definintion der verallgemeinerten Zeitableitung gilt damit
\begin{align}\label{38}
    \frac{\d u}{\d t} = b-\xi \in X'
\end{align}
und somit $u\in W$. Aus (\ref{33}) und (\ref{38}) folgt mit $v(t)=\vp (t) w$, dass
\[
    \int_0^T (u(t),w)_H \vp ' + \lal b-\xi, w\ral_V \vp (t) \d t = (u(T),w)_H\vp(T)-(u(0),w)_H\vp(0)
\]
Wir ziehen diese Gleichung von (\ref{37}) ab und erhalten
\[
    (u(T),w)_H\vp(T)-(u(0),w)_H\vp(0) = (u^*,w)_H\vp(T)-(u_0,w)_H\vp(0),
\]
also folgt $u(T)=u^*$, $u(0)=u_0$. Es bleibt noch zu zeigen, dass $Au=\xi$. Wir multiplizieren nun die
$j$-te Gleichung im Galerkin-System mit $C_j^m$ und summieren. Dann integrieren wir die Gleichung über
$I$. Mit (\ref{33}) folgt
\[
    \int_0^T \lal Au_n,u_n\ral \d t = \int_0^T \lal b_n,u_n \ral \d t - \frac 12 \|u_n (T) \|_H^2
\]
Es gilt $b_n \ra b$ in $X'$, $u_n(T)\rightharpoonup u(T)$ und $u_n(0)\rightharpoonup u_0$. Mit der
schwachen Unterhalbstetigkeit von Normen folgt, dass
\[
    \limsup_{n\ra \infty} \int_0^T \lal Au_n,u_n\ral \d t \leq \int_0^T \lal b,u\ral \d t -
    \frac12 \|u(T)\|_H^2 + \frac12 \|u(0)\|_H^2
\]
Ebenfalls gilt, dass
\[
    -\frac12 \|u(T)\|_H^2 + \frac12 \|u(0)\|_H^2 \overset{(\ref{33})}{=} 
    -\int_0^T \Lal \frac{\d u}{\d t}, u \Ral \d t \overset{(\ref{38})}{=} \int_0^T \lal \xi- b , u\ral
    \d t.
\]
Somit folgt, dass 
\[
    \limsup_{n\ra \infty} \int_0^T \lal Au_n,u_n \ral \d t \leq \int_0^T \lal \xi, u\ral \d t.
\]
Nachdem $A$ pseudomonoton ist und somit der Bedingung (M) genügt, folgt
\[
    Au=\xi.
\]
Damit ist $u$ eine Lösung der Evolutionsgleichung.
\end{description}
\[ \]
\end{proof}

\begin{lem}\label{4.40}
    Es sei $V$ ein separabler Banachraum $A:V\ra V'$ ein demistetiger Operator, welcher der
    Wachstumsbedingung $\|Au\|_{V'}\leq c\|u\|_V^{p-1}$ für $p>1$ genügt. Der induzierte Operator
    $\tilde A$, definiert durch
    \[
        (\tilde Au)(t)=A(u(t))
    \]
    bildet dann den Raum $L^p(I;V)$ in den Dualraum  $L^{p'}(I;V')$ ab und ist beschränkt und demistetig.
\end{lem}

\begin{proof}
    \begin{description}
        \item{1)}
        Messbarkeit von $\tilde Au$. Wir approximieren $u$ durch  f.ü. stark konvergente
        Treppenfunktionen $(u_n)_{n\in\N}$. Damit konvergiert $\tilde Au_n$ wegen Demistetigkeit von
        $A$ schwach gegen $\tilde A u$ f.f.a. $t\in I$. Bochner-Messbarkeit von $\tilde Au$ folgt mit
        Korollar \ref{4.25} (zum \textit{Satz von Pettis}).
        \item{2)}
        $\tilde A: L^p(I;V)\ra L^{p'}(I; V')$ ist beschränkt. Sei $u\in L^p(I;V)$. Die Funktion
        \[
            t\mapsto \|(\tilde A u )(t)\|_{V'}
        \]
        ist Lebesgue-messbar und es gilt wegen der Wachstumsbedingung $\|(\tilde Au)(t)\|_{V'}^{p'}
        \leq c \|u(t)\|_V^p$. Es folgt, dass $\|\tilde A u\|_{L^{p'}(I;V)}\leq c\|u\|^{p-1}_{L^p(I;V)}$.
        \item{3)}
        Die Demistetigkeit: Es sei $(u_n)_{n\in\N}$ eine Folge in $L^p(I;V)$, die stark konvergiert,
        so dass
        \[
            \int_I\|u_n - u\|_V^p \d t \ra 0 \qq (n\ra \infty)
        \]
        Damit existiert eine f.ü. konvergente Teilfolge (wieder mit $n$ indiziert). Für fast alle
        $t\in I$ gilt somit $\tilde Au_n(t)\rightharpoonup \tilde Au(t)$. Sei dann $\vp L^p(I;V)
        =(L^{p'}(I;V'))'$, dann gilt f.f.a. $t\in I$
        \[
            \lal \tilde A u_n (t),\vp (t)\ral_V \ra \lal \tilde Au(t), \vp (t)\ral _V
        \]
        Mit der Wachstumsbedingung und der \textit{Young'schen Ungleichung} folgt, dass
        \[
            |\lal \tilde A u_n  (t), \vp (t)\ral _V|\leq c \left( \|u_n(t)\| + \|\vp(t)\|_V^p\right)
        \]
        wobei die rechte Seite punktweise f.ü. konvergiert und in $L^1(I)$ gegen 
        $c(\|u\|_V^p+\|\vp\|_V^p)$. Mit majorisierter Konvergenz folgt
        \[
            \int_I\lal \tilde Au_k(t),\vp(t)\ral \d t \ra \int_I \lal \tilde A u (t), \vp (t)\ral _V \d t
            \qq \forall \vp \in L^p(I;V)
        \]
        Damit konvergiert $\tilde Au_n$ schwach gegen $\tilde Au$, zunächst nun für eine Teilfolge.
        Die Konvergenz für die Originalfolge folgt mit dem üblichen Teilfolgen von Teilfolgen Argument.
    \end{description}
    \[ \]
\end{proof}
