\subsubsection*{Korrekturen}

Bei Satz \ref{2.15}. Vorsicht vor der unbeschränkten Komponente $K_\infty$ von $\R^n\setminus f(\partial
\Omega)$! Diese taucht jedoch in der Produktformel ausschließend nicht auf, denn
\[
    f(\ol \Omega)\subset B_r(0), \, \text{mit}\, f(\ol \Omega) \text{ kompakt}
\]
\[
    \Ra \, \deg(f,\Omega,K_\infty\cap B_{2r}(0))=0 \qq \text{und} \qq f(\ol \Omega) \cap (\R^n\setminus
    B_{2r}(0))=\varnothing
\]
somit können wir vereinfacht schreiben
\[
    \deg(f,\Omega,K_\infty)=0
\]
Der in der Produktformel auftauchende Term
\[
    \deg(f,\Omega,K_\infty)\cdot \deg(g,K_\infty,y)
\]
ist also unproblematisch, da er verschwindet.\\[0.5cm]

\begin{lem}\label{3.5}
    Sei $K\subset X$ kompakt, $\eps>0$. Dann existiert ein endlichdimensionaler Untervektorraum $X_\eps
    \subset X$ und eine stetige Abbildung
    \[
        P_\eps:K\ra X_\eps,
    \]
    so dass
    \[
        \|P_\eps(x)-x\|\leq \eps \qq \forall x\in K
    \]
\end{lem}

\begin{proof}
    Es sei $\{x_j\}_{j=1}^N\subset K$, so dass $\bigcup _{j=1}^NB_\eps(x_j)$ eine Überdeckung von $K$
    darstellt. Sei $\{\Phi_j\}_{j=1}^N$ eine Zerleung der Eins, welche $\{B_\eps(x_j)\}_{j=1}^N$
    untergeordnet ist. Wir definieren
    \[
        P_\eps(x):= \sum_{j=1}^N \Phi_j(x)x_j
    \]
    Damit gilt
    \[
        \|P_\eps(x)-x\|=\left\|\sum_{j=1}^N\Phi_j(x)x_j-\sum_{j=1}^N \Phi_j(x)x\right\|
        \leq \sum_{j=1}^N \Phi_j(x)\|x-x_j\|\leq \eps
    \]
    Es ist nämlich $\|x-x_j\|< \eps$ für alle $x\in B_\eps(x_j)$, aber für $\|x-x_j\|>\eps$ ist
    $\Phi_j(x)=0$.
    \[ \]
\end{proof}

\begin{theorem}\label{3.6}
    $X,Y$ Banachräume. Sei $\Omega\subset X$ beschränkt. Dann gilt
    \[
        \ol{\ms F(\Omega,Y)}^{C(\Omega,Y)}=\ms C(\Omega,Y)
    \]
    (D.h. der Abschluss von $\ms F(\Omega,Y)$ bzgl. der Norm der gleichmäßigen Konvergenz ist der Raum
     der kompakten Operatoren $\ms C(\Omega,Y)$.)
\end{theorem}

\begin{proof}
    Es sei $(F_n)_{n\in \N}$ eine Folge in $\ms C(\Omega,Y)$, sodass  $F_n\ra F$. Wir nehmen an, dass
    $F\nin \ms C(\Omega,Y)$. Dann existiert eine (beschränkte) Folge $(x_n)_{n\in \N}$, so dass
    \[
        \|F(x_k)-F(x_l)\|\geq \rho>0 \qq\text{für}\qq k\neq l
    \]
    Es sei $n$ so groß, dass
    \[
        \|F_n-F\|_{ C(\Omega,Y)}<\frac{\rho}{4}
    \]
    Damit gilt
    \[
        \|F_n(x_k)-F_n(x_l)\|\geq \frac\rho2.
    \]
    Das ist ein Widerspruch zur Annahme, dass $F_n\in \ms C(\Omega,Y)$. Somit gilt auch
    \[
        \ol{\ms F(\Omega,Y)}^{C(\Omega,Y)}\subset \ms C(\Omega,Y)
    \]
    Sei andererseits $F\in \ms C(\Omega,Y)$,und sei $K=\ol{F(\Omega)}$ kompakt. sei $P_\eps$ gewählt wie
    in Lemma \ref{3.5}. Es gilt
    \[
        F_\eps=P_\eps\circ F\in \ms F(\Omega,Y) \qq \text{und}\qq F_\eps\ra F
    \]
\end{proof}

Zur Definition des Abbildungsgrades auf unendlichdimensionalen Räumen betrachten wir insbesondere
kompakte Störungen der Identität (d.h. Operatoren der Form $\Id+F$, $F\in \ms C$). Wir bezeichnen deshalb
zunächst einige Interessante Eigenschaften von $\Id+F$.

\begin{lem}\label{3.7}
    Sei $X$ ein Banachraum und $\Omega\subset X$ beschränkt und abgeschlossen, $F\in \ms C(\Omega,X)$.
    Dann ist $\Id + F$ eine eigentliche Abbildung (d.h. $(\Id+F)^{-1}(K)$ ist kompakt für $K\subset X$ 
    kompakt), die abgeschlossene Teilmengen von $\Omega$ auf abgeschlossene Teilmengen von $X$ abbildet.
\end{lem}

\begin{proof}
    Sei $A\subset \Omega$ abgeschlossen, und sei
    \[
        y_n=(\Id+F)(x_n), \qq \text{mit } x_n \in A,
    \]
    so dass $y_n \ra y$ in $X$. Zu zeigen ist, dass $y\in (\Id+F)(A)$. Es gilt
    \[
        y_n-x_n=F(x_n),
    \]
    somit folgt $y_n-x_n\ra z$ in $X$, nach Extraktion einer Teilfolge (da $F$ kompakt, $(x_n)$
    beschränkt in $\Omega$). Es folgt $x_n\ra x \in A$, da $A$ abgeschlossen und $y_n$ konvergent
    nach Annahme. Wir haben aber $x=y-z\in A$. Nachdem aber $y=x+F(x)\in (\Id+F)(A)$ folgt die Behauptung
    mit $F(x)=z$

    Sei nun $\Omega$ abgeschlossen und $K\subset X$ kompakt. Sei $(x_n)_{n\in \N}$ eine Folge in
    $(\Id+F)^{-1}(K)$. Wir können wieder eine Teilfolge
    \[
        K\ni y_n=x_n+F(x_n)
    \]
    wählen, so dass $y_n\ra y$. Wie im vorhergehenden Teil folgt, folgt dass $x_n\ra x\in K$, somit ist
    $(\Id+F)^{-1}(K)$ kompakt.
    \[ \]
\end{proof}

\subsection{Der Leray-Schauder Grad}

Sei $\Omega\subset X$, $X$ ein Banachraum. Wir setzten
\[
    \ms D_y(\ol \Omega,X)=\{ F\in \ms C(\ol\Omega,X)\, | \, y\nin(\Id + F)(\partial \Omega) \}
\]
und
\[
    \ms G_y(\ol\Omega,X):=\{F\in \ms F(\ol \Omega,X)\, |\, y\nin(\Id+F)(\partial\Omega)\}
\]
Es gilt für $F\in \ms D_y(\ol \Omega,X)$, dass $\dist(y,(\Id+F)(\partial\Omega))>0$, da $\Id+F$
abgeschlossene Mengen auf abgeschlossene Mengen abbildet.

\begin{defi}\label{3.8}
    Sei $\Omega\subset X$ offen, beschränkt, $y\in X$, $F\in \ms D_y(\ol \Omega,X)$. Sei $\rho:=
    \dist(y,(\Id+F)(\partial \Omega))$ und sei
    \[
        F_1\in \ms F(\ol \Omega, X),
    \]
    sodass
    \[
        \|F-F_1\|<\rho. \qq (\Ra \, F_1\in \ms G_y(\ol \Omega,X))
    \]
    Nun sei $X_1\subset X$ ein endlichdimensionaler Untervektorraum, so dass
    \[
        F_1(\Omega)\subset X_1,\, y\in X_1
    \] 
    und sei $\Omega_1=\Omega\cap X_1$. Damit ist $F_1\in \ms G_y(\ol\Omega_1,X_1)$ und wir definieren
    \[
        \deg(\Id+F,\Omega,y):=\deg(\Id+F_1,\Omega_1,y)
    \]
\end{defi}

\begin{prop}\label{3.9}
    Die obige Definition ist unabhängig von der Wahl von $F_1$ und $X_1$.
\end{prop}

\begin{proof}
    Sei $F_2\in \ms F (\ol \Omega,X)$, $\|F_2-F\|<S$, $X_2$ entsprechend endlichdim UVR von $X$. Es sei
    $X_0=X_1+X_2$, $\Omega_0=\Omega\cap X_0$. Dann gilt
    \[
        F_i\in \ms F_y(\ol\Omega_0,X_0) \qq i=1,2
    \]
    und es folgt wegen der Reduktionseigenschaft des endlichdimensionalen Abbildungsgrades, dass
    \[
        \deg(\Id + F_i,\Omega_0,y)=\deg(\Id+F_i,\Omega_i,y)\qq i=1,2
    \]
    Sei $H(t)=\Id + (1-t)F_1+tF_2.$ Es folgt, dass
    \[
        H(t)\in D_y(\ol\Omega_0,X_0) \qq t\in [0,1],
    \]
    nachdem gilt $\|H(t)-(\Id+F)\|<\rho$ $\forall t\in [0,1]$.
    \[
        \Ra \, \deg(\Id+F_1,\Omega_0,y)=\deg(\Id+F_2,\Omega_0,y)
    \]
\end{proof}

\begin{theorem}\label{3.10}
    Sei $\Omega\subset X$ beschränkt und offen und sei $F\in \ms D_y(\ol\Omega,X)$, $y\in X$. Dann gilt
    \begin{description}
        \item{i)}
        $\deg(\Id+F,\Omega,y)=\deg(\Id+F-y,\Omega,0)$
        \item{ii)}
        $\deg(\Id, \Omega,y)=1$, falls $y\in \Omega.$
        \item{iii)}
        Falls $\Omega_1,\Omega_2$ offene, disjunkte Teilmengen von $X$ sind, so dass
        $y\nin (\Id+F)(\ol\Omega\setminus (\Omega_1\cup\Omega_2)),$ dann gilt
        \[
            \deg(\Id+F,\Omega,y)=\deg(\Id+F,\Omega_1,y)+\deg(\Id+F,\Omega_2,y).
        \]
        \item{iv)}
        Falls $H:[0,1]\times\ol\Omega\ra X$ stetig, so dass $\forall t_0$, $\forall\eps>0$ ein
        $\delta>0$ gibt mit
        \[
            \|H(t_0)-H(t)\|\leq \eps, \, \text{falls } |t-t_0|<\delta, \qq y[0,1]\ra X \text{ stetig,}
        \]
        und sei $H(t)\in \ms D _{y(t)}(\ol\Omega,X)$ $\forall t\in[0,1]$. Dann gilt
        \[
            \deg(\Id+H(0),\Omega,y(0))=\deg(\Id+H(1),\Omega,y(1))
        \]
    \end{description}
\end{theorem}

\begin{proof}
    \textit{i)-iii)} folgen sofort aus der Definition und den entsprechenden Eigenschaften des
    endlichdimensionalen Abbildungsgrades.

    \noindent\textit{iv)} 
        O.E. sei $y(t)=0$ ($\tilde H(t)(x)=H(t)(x)-y(t)$).\\
        \textbf{Behauptung:} $H\in \ms C([0,T]\times \ol\Omega,X)$
        \begin{proof}
            Sei $\eps >0,\, \forall t\in [0,1]$ sei $H_\eps^t$ eine endlichdimensionale Approximation
            von $H(t)$, so dass
            \[
                \|H_\eps^t-H(t)\|\leq\frac\eps4 \qq \forall t \qq \text{nach Satz \ref{3.6}}
            \]
            Nun sei $\{t_j\}_{j=1}^N$, so dass $\bigcup _{j=1}^N B_{j_j}(t_j)\supset [0,1]$ mit $\delta_j
            $, so dass $\|H(t_j)-H(t)\|\leq \frac\eps4$ für $t\in B_{f_j}(t_j)$. Nun sei $F_\eps$ die
            stückweise affine Interpolation von $H_\eps^{t_j}$. Damit ist $F_\eps$ endlichdimensional und
            $\|F_\eps(t)-H(t)\|<\eps$. Somit ist nach Satz \ref{3.6} $H\in \ms C([0,1]\times 
            \ol \Omega, X)$.
            \[ \]
        \end{proof}
        Mit dem selben Argument, das in Satz \ref{3.7} benutzt wird gilt, dass
        \[
            A:=\bigcup_{t\in [0,1]} (\Id+H(t))(\partial \Omega)
        \]
        ist abgeschlossen als Bild einer abgeschlossenen Menge, somit gilt
        \[
            \dist(A,y)=\rho>0
        \]
        Jetzt benutzen wir einfach die bereits vorher konstruierte endlichdimensionale Approximation 
        $F_{\frac\rho2}$ von $H$ und wenden die Homotopieinvarianz im endlichdimensionalen Fall an.
        \[ \]
\end{proof}

Auch die anderen Aussagen über den endlichdimensionalen Abbildungsgrad übertragen sich.
Die Beweise bleiben gleich.

\begin{theorem}\label{3.11}
    Seien $F,G\in \ms D_y(\ol\Omega,X)$. Es gilt
    \begin{description}
        \item{i)}
        $\deg(\Id+F,\varnothing,y)=0$. Weiter gilt, falls $\Omega_i$, $1\leq i\leq N$ offen, disjunkt,
        so dass
        \[
            y\in (\Id+F)(\ol\Omega\setminus\bigcup_{i=1}^N\Omega_i),
        \]
        folgt dass $\deg(\Id+F,\Omega,y)=\sum_{i=1}^N \deg(\Id+F,\Omega_i,y)$
        \item{ii)}
        Falls $y\nin (\Id+F)(\Omega)$, dann gilt
        \[
            \deg(\Id+F,\Omega,y)=0
        \]
        (die Umkehrung gilt nicht notwendigerweise!)

        Äquivalent dazu gilt
        \[
            \deg(\Id+F,\Omega,y)\neq 0 \, \Ra \, y\in (\Id+F)(\Omega)
        \]
        \item{iii)}
        Falls $\|F(x)-G(x)\|<\dist(y,(\Id+F)(\partial\Omega))$ $\forall x\in \partial \Omega$, so gilt
        \[
            \deg(\Id+F,\Omega,y)=\deg(\Id+G,\Omega,y).
        \]
        Insbesondere gilt das auch für den Fall, dass
        \[
            F=G\qq \text{auf } \partial \Omega
        \]
        \item{iv)}
        $\deg(\Id+\cdot,\Omega,y)$ ist konstant auf Zusammenhangskomponenten von $D_y(\ol\Omega,X)$.
        \item{v)}
        $\deg(\Id+F,\Omega,\cdot)$ ist konstant auf Zusammenhangskomponenten von $X\setminus 
        (\Id+F)(\partial \Omega)$.
    \end{description}
\end{theorem}
