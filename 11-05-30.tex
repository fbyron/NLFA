\newpage

\chapter{Monotone Operatoren}

Man möchte folgendes Prinzip auf Banachräume verallgemeinern: $F:\R\ra\R$ genüge
\begin{description}
    \item{(a)}
    monoton
    \item{(b)}
    stetig
    \item{(c)}
    $F$ ist koerziv, d.h. $\lim_{v\ra \pm \infty} F(v)=\pm \infty$
\end{description}
Dann
\[
    \forall b\in \R \, \exists ! v\in \R: \, F(v)=b
\]

\begin{defi}\label{4.1}
    Sei $X$ ein reflexiver Banachraum und
    \[
        A:X\ra X^\star
    \]
    ein Operator. A heißt
    \begin{description}
        \item{(i)}
        monoton, falls für alle $u,v\in X$ gilt
        \[
            \lal Au-Av,u-v\ral\geq 0
        \]
        \item{(ii)}
        strikt monoton, falls für alle $ u, v\in X$ $u\neq v$ gilt
        \[
            \lal Au-Av,u-v\ral>0
        \]
        \item{(iii)}
        stark monoton, falls es ein $c >0$ gibt, sodass für alle $u,v\in X$ gilt
        \[
            \lal Au-Av,u-v\ral\geq c\|u-v\|^2
        \]
        \item{(iv)}
        koerziv, falls
        \[
            \lim_{\|u\|\ra \infty}\frac{ \lal Au,u\ral }{\|u\|}=\infty
        \]
    \end{description}
\end{defi}

\begin{remark}
    $A$ stark monoton $\Ra$ $A$ koerziv.
\end{remark}

\begin{proof}
    \[
        \frac{\lal Av,v\ral}{\|v\|}=\frac{ \lal Au-A(0), v\ral + \lal A(0),v\ral }{\|v\|}
        \geq \frac{c\|v\|^2 - \|A(0)\|\|v\|}{\|v\|} = c\|v\| - \|A(0)\|\ra \infty \qq (\|v\|\ra \infty)
    \]
\end{proof}

\subsubsection*{Beispiele:}

\begin{description}
    \item{1.}
    Sei $f:\R\ra\R$. Das Dualitätsproblem in $\R$ ist die Multiplikation. Im Spezialfall $X=\R=X^\star$
    stimmen die Monotoniebegriffe reeller Funktionen und für Operatoren überein, da
    \[
        \lal f(u)-f(v),u-v\ral=(f(v)-f(u))\cdot(u-v)\geq 0
    \]
    $f$ ist außerdem genau dann koerziv, d.h.
    \[
        \lim_{v\ra \pm \infty} \frac{f(u)\cdot u}{|u|}=\pm\infty \, \LRa \, f(v)\ra \pm \infty\qq(v\ra\pm
                \infty)
    \]
    \item{2.}
    $g:\R\ra \R$
    \[
        g(u):=\begin{cases} |u|^{p-2}u, & u\neq 0\\ 0, & u=0 \end{cases}
    \]
    $\Ra$ \begin{description}
        \item{(i)}
        $p>1 \, \Ra \, g$ strikt monoton.
        \item{(ii)}
        $p\geq 2 \, \Ra \, \lal g(u)-g(v),u-v\ral \geq c|u-v|^p$
        \item{(iii)}
        $p=2 \, \Ra \, g$ stark monoton
    \end{description}
    \begin{proof}
    Übungsaufgabe. \[ \]
    \end{proof}
\end{description}

\begin{defi}\label{4.2}
    Sei $X$ reflexiver Banachraum und $A:X\ra X^\star$ ein Operator. $A$ heißt
    \begin{description}
    \item{(i)}
    \textit{demistetig} $\LRa$ $\left( u_n\ra u\, \Ra \, Au_n\rightharpoonup Au \right)$
    \item{(ii)}
    \textit{hemistetig} $\LRa \, \forall u,v,w \in X: \, t\mapsto \lal A(u+tv),w\ral$ ist stetig.
    \item{(iii)}
    \textit{stark stetig} $\LRa \, \left( u_n \rightharpoonup u\, \Ra \, Au_n\ra Au \right)$
    \item{(iv)}
    \textit{beschränkt} $\LRa \, A(V)$ beschränkt, falls $V$ beschränkt ist. 
    \end{description}
\end{defi}

\begin{remark}
    stark stetig $\Ra$ stetig $\Ra$ demistetig $\overset{(\star)}{\Ra}$ hemistetig
\end{remark}

\begin{proof}
    Zu $(\star)$. Seien $u,v,w\in X$, $t\ra t_0$ 
    \[
        \Ra\, u+tv \ra u+t_0v
    \]
    \[
        \overset{\text{demistetig}}{\Ra} \, A(u+tv) \rightharpoonup A(u+t_0v)
    \]
    \[
        \LRa \lal A(u+tv),w\ral\ra \lal A(u+t_0v),w\ral
    \]
\end{proof}

\begin{lem}\label{4.4}
    $X$ reflexiver Banachraum, $A:X\ra X^\star$. Dann gelten
    \begin{description}
        \item{(i)}
        $A$ stark stetig $\Ra$ $A$ kompakt.
        \item{(ii)}
        $A$ demistetig $\Ra$ $A$ lokal beschränkt.
        \item{(iii)}
        $A$ monoton $\Ra$ $A$ lokal beschränkt.
        \item{(iv)}
        $A$ monoton und hemistetig $\Ra$ $A$ demistetig.
    \end{description}
\end{lem}

\begin{proof}
    \begin{description}
    \item{(i)}
        Sei $(v_n)_{n\in \N}\subset X$ beschränkt, da $X$ reflexiv folgt:
        \[
            \exists u_{m_k} \rightharpoonup u \in X\overset{\text{stark stetig}}{\Ra} A(u_{m_k})\ra A(u)
        \]
        $\Ra$ $A$ kompakt.
    \item{(ii)}
    Wäre $A$ nicht lokal beschränkt, so existiert $v\in X$, $v_m\ra v$, sodass
    \[
        \|Av_m\|\ra \infty\, \lightning  \qq Av_m \rightharpoonup Av
    \]
    \item{(iii)}
    Angenommen $u_m\ra u: \, \|Au_m\|\ra \infty$
    \[
        a_n:=(1+\|Au_n\|\|u_n-u\|)^{-1} \qq ((a_n)_{n\in \N} \text{ beschränkt duch $1$})
    \]
    $A$ monoton
    \begin{align*}
        \Ra \, 0&\leq \lal Au_n-Av,u_n-v\ral= \lal Au_n-Av,(u_n-u)+(u-v)\ral\\
        \Ra\, a_n\lal Au_n,v-u\ral &\leq a_n(\lal Au_n,u_n-v\ral-\lal Av,u_n-v\ral)\\
        &\leq 1+ c( u,v) \qq \forall v\in X
    \end{align*}
    Einsetze $v\leadsto 2u-v$
    \begin{align*}
        -a_n \lal Au_n,v-u\ral&\leq 1+\tilde c( u,v) \qq \forall v \in X\\
        \Ra \, |\lal a_n A u_n,v-u\ral|&\leq 1+c( u,v)
    \end{align*}
    $\Ra \, (a_nAu_n)_{n\in \N}$ ist punktweise beschränkt. Mit \textit{Banach-Steinhaus}
    \[
        \sup_n\|a_nAu_n\|_{X^\star}\leq c(u)
    \]
    \[
        \Ra \, \|Au_n\|\leq \frac{c(u)}{a_n} = c(u)(1+\|Au_n\|\|u_n-v\|)
    \]
    Für
    \[
        \|u_n-u\|c(u)=\lambda <1: \, \|Au_n\|\leq \frac{c(u)}{1-\lambda}<\infty
    \]
    Ein Widerspruch zu $\|Au_n\|\ra \infty$.
    \item{(iv)}
    Gelte $u_n\ra u$. $A$ monoton $\Ra$ $(Au_n)$ beschränkt.
    $X^\star$ reflexiv
    \[
        \Ra \, Au_{n_k}\rightharpoonup b \in X^\star
    \]
    \[
        \Ra \, Au=b \qq \text{(siehe nachfolgendes Lemma)}
    \]
    Dies gilt für jede Teilfolge von $u_n, \Ra  Au_n \rightharpoonup b$.
    \end{description}
    \[ \]
\end{proof}

\begin{lem}[Minty]\label{4.5}
    Sei $X$ ein refelexiver Banachraum und $A:X\ra X^\star$, hemistetig und monoton. Dann gelten
    \begin{description}
        \item{(i)}
        $A$ maximal monoton, d.h.
        \begin{align*}
            u\in X, \, b\in X^\star, \text{ und } \lal b-Av,u-v\ral\geq 0 \qq \forall v\in X\, \Ra \,b=Au
        \end{align*}
        \item{(ii)}
        $A$ genügt der Bedingung (M), d.h.
        \begin{align}\label{15}
            \left.\begin{array}{lr}
            v_n \rightharpoonup v& \\
            Av_n\rightharpoonup b& \\
            \lal Av_n,v_n\ral \ra \lal b,v\ral&
            \end{array}\right\}
            \,\Ra \, Av=b            
        \end{align}
        \item{(iii)}
        \[
            u_n \rightharpoonup u, \, Au_n\ra b \, \Ra \, Au=b
        \]
        \[
            u_n\ra u \,\wedge \, Au_n \rightharpoonup b \, \Ra \, Au=b
        \]
    \end{description}
\end{lem}

\begin{proof}
    \begin{description}
    \item{(i)}
    Sei $w\in X$ und $v=v-tw, \, t>0$.
    \[
        \Ra \,\lal b-A(u-tw),w\ral\geq 0
    \]
    \[
        \Ra\,\lal b-Au, w\ral\geq 0 \qq w\in X
    \]
    \[
        \Ra \,\lal b-Av,w\ral=0 \forall w
    \]
    \[
        b=Aw
    \]
    \item{(ii)}
    $A$ ist monoton.
    \begin{align*}
    \Ra \, 0&\leq \lal Au_n-Av,u_n-v\ral = \lal Au_n,u_n\ral-\lal Av,u_n\ral - \lal Au_n,Av,v\ral\\
    &\ra \lal b,v\ral-\lal Av,u\ral-\lal b-Av,v\ral
    \end{align*}
    \end{description}
\end{proof}

\begin{theorem}[Brouder, Minty]\label{4.6}
    Sei $X$ ein separabeler, reflexiver Banachraum mit Basis $(w_i)_{i\in \N}$. Sei $A:X\ra X^\star$
    monoton, koerziv und hemistetig. Dann existiert für alle $b\in X^\star$ eine Lösung $u\in X$ von
    \[ Au=b \]
    $\{A=b\}$ ist abgeschlossen, beschränkt, und konvex. Falls $A$ strikt monoton ist, ist die Lösung
    eindeutig.
\end{theorem}

\begin{idea} Der Beweis erfolgt per \textit{Galerkin-Approximation}.
\begin{description}
    \item{1.}
    $X$ separabel, d.h.
    \[
        X=\bigcup_{n=1}^\infty X^n, \qq X_n=\mr{span}(w_1,…,w_n)
    \]
    Approximiere „die Gleichung“ $Au=b$ durch „endlichdimensionale“ Probleme der Form
    \[
        (Au,z^n)=(b,z^n)    \forall z^n\in X_n
    \]
    \item{2. \textit{A-priori}-Abschätzung für Lösungen:}
    Wir zeigen, dass die Folge $(v_n)_{n\in \N}$ der Lösungen dieser Probleme beschränkt ist (mit Hilfe
            der Koerzivität).
    \item{3. Schwache Konvergenz}
    $X$ reflexiv $\Ra$ $(v_n)_{n\in \N}$ hat eine schwach konvergente Teilfolge
    \[
        u_{n_k}\rightharpoonup u \in X
    \]
    \item{4.}
    Lemma von Minty $\Ra$ Au=b.
\end{description}
\[ \]
\end{idea}
