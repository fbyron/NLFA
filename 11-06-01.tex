\begin{proof}
    \begin{description}
    \item{(i)}
    Der Beweis erfolgt per \textit{Galerkin-Approximation}
    \[
        X_n=\mr{span}(w_1,…,w_n)
    \]
    Betrachte
    \begin{equation}\label{20}
        (Au^n-b,w_k)=0 \qq \forall k=1,…,n
    \end{equation}
    Definiere
    \[
        g:\R^n\ra \R^n; \qq g_k(c^n):= (A(\sum_mc_m^nw_m)-b,w_k)
    \]
    \[
        \ref{20} \, \Ra \, g(c^n)=0 \qq \text{für ein } c^n\in\R^n  
    \]
    O.B.d.A sei $|c^n|:= \|\sum_{k=1}^n c^n_kw_k\|_X$ als norm auf $\R^n$. Aus Lemma \ref{4.4} folgt,
    dass $A$ demistetig. 

    Außerdem gilt
    \[
        \sum_{k=1}^n g_k(c^n)c^n_k=(Au^n,u^n)-(b,u^n), \qq \text{wobei } u^n=\sum_kg_k(c^n)w_k^n
    \]
    $A$ ist koerziv
    \[
        \Ra \, R_0>0 \qq \forall \|u\|\geq R_0>0: \qq (Au,u)\geq (1+\|b\|)\|v\| =0
    \]
    \begin{align*}
        \Ra \, g(c^n)\cdot c^n &= \sum_{k=1}^ng_k(c^n)\cdot c_k^n \geq (1+\|b\|)\|u^n\| - \|b\|\|u^n\|
        =\|u^n\|>0 \qq \forall c^n \text{ mit } \|u^n\|\geq R_0
    \end{align*}
    Mit \textit{Brouwer} folgt (beachte: $|\cdot|_{\R^n}=\|\|_X$)
    \[
        \exists u^n\in X_n: \, g(c^n)=0,
    \]
    d.h. \ref{20} gilt für $u^n$.
    \[
        \Ra \, \|u^n\|\geq R_0
    \]
    \item{(ii)}
    $A$ monoton $\Ra$ $A$ lokal becshränkt
    \[
        \Ra \, \exists r,\delta>0: \, \|u\|\leq r \, \Ra \, \|Au\|\leq \delta
    \]
    und
    \[
        (Au^n-A^v,u^n-v)\geq 0 \qq \forall v\in X
    \]
    Wegen $u^n\in X_n$ und \ref{20} gilt
    \[
        (Au^n,u^n)=(b,u^n)\, \Ra \, |(Au^n,u^n)|=\|b\|R_0
    \]
    \begin{align*}
        \|Au^n\|&= \sup_{v\in X, \, \|v\|=r} \frac1r(Au^n,v) \overset{\text{Monotonie}}{\leq} 
        \sup_{v\in X, \|v\|=5} \frac1r ((Av,v)+(au^n,u^n)-(Av,u^n)) \leq \frac1r (\delta r+ \|b\|R_0
                +\delta R_0)
    \end{align*}
    $\Ra $ $(au^n)_{n\in \N}$ ist beschränkt.
    \item{(iii)}
    Konvergenz des Galerkin-Verfahrens; $\|u^n\|\leq R_0$ und $X$ reflexiv.
    \[
        \Ra \, \exists \text{ Teilfolge } \, u^n\rightharpoonup u \,\in X
    \]
    Sei 
    \[
        w\in \bigcup_{m=1}^\infty X_m \, \Ra \, \exists n_0\in\N: \, w\in X_n\qq n\geq n_0
    \]
    \begin{align*}
        \Ra \, \forall n\geq n_0: \, (Au^n,w)&=(b,w)\\
        \Ra \, \lim_{n\ra \infty} (Au^n,w)=(b,w) \qq w\in \bigcup_{m=1}^\infty X_m
    \end{align*}
    $X^\star$ reflexiv und $(Au^n)$ beschränkt
    \begin{align*}
        \exists Au^{n_m}\rightharpoonup c \in X^\star\\
        \Ra \, \forall w\in X: \, (Au^{n_k},w)\ra (c,w)\\
        \Ra \, b=c, \qq \text{weil } \ol{\bigcup_{m=1}^\infty X_m}=X \qq (X \text{ separabel})\\
        \Ra Au^{n_m} \rightharpoonup b
    \end{align*}
    Wiederhole dieses Argument für jede Teilfolge von $Au^n$.
    \[
        \Ra \, Au^n\rightharpoonup b
    \]
    \begin{align*}
        \Ra \, (Au^n,u^n)&=(b,u^n)\ra (b,u)\\
        \lim_{n\ra \infty} (Au^n,u^n)&= (b,u)
    \end{align*}
    Mit Lemma \ref{4.5} (ii) erhalten wir
    \[ 
        Au=b
    \]
    \item{(iv)}
    Eigenschaften der Lösungsmenge $S=\{A=b\}$:
    \begin{description}
    \item{(a) Beschränktheit:}
    \[
        \exists \, R_0>0 \, \forall \|V\|\geq R_0>0: \, (Au,u)\geq (1+\|b\|)\|u\|
    \]
    Sei $u$ eine Lösung mit $\|u\|\geq R$
    \begin{align*}
        \Ra \, 0&=(Au,u) -(b,u)\geq \|u\|>0 \, \lightning\\
        \Ra \, S\subset B_{R_0}(0)
    \end{align*}
    \item{(b) Konvexität:}
    Seien $u_1,u_2\in S$, $w=tu_1+(1-t)u_2$, $v\in X$.
    \begin{align*}
        (b-Av,w-v)&=(b-Av,t(u_1-v))+(b-Av,(1-t)(u_2-v))\\
                 &= t(Au_1-Av,u_1-v)+(1-t)(Au_2-Av,u_2-v)\\
                 &\geq 0 \qq \forall v\in X\\
        \overset{\text{Lem. \ref{4.5}}}\Ra \, b&=Aw \, \Ra \, w\in S 
    \end{align*}
    \end{description}
    \end{description}
    \[ \]
\end{proof}

\begin{remark}
    Der Satz gilt auch, falls $X$ nicht separabel ist (ohne Beweis).
\end{remark}

\begin{cor}\label{4.7}
    Sei $X$ separabel, refl. $\R$-Banachraum und $A:X\ra X^\star$ strikt monoton, koerziv, hemstetig.
    Dann existiert
    \[
        A^{-1}:X^\star\ra X
    \]
    und $A^{-1}$ it trikt monoton und demistetig.
\end{cor}

\begin{proof}
    Übung. \[ \]
\end{proof}
