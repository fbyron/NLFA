Wir haben gezeigt, dass $A$ alle Voraussetzung von Satz \ref{4.6} erfüllet, somit folgt sofort:

\begin{theorem}\label{4.12}
    Sei $\Omega\subset\R^n$ offen, beschränkt, $s\geq 0$. Es sei $p\geq  \frac{2n}{n+2}$, $p>1$. Dann
    existiert zu jedem $f\in L^{p'}$ genau eine schwache Lösung des $p$-Laplace-Randwertproblems
    (\ref{23}).
\end{theorem}

\begin{proof}
    Lemma \ref{4.10}, \ref{4.11}, Satz \ref{4.6}.
    \[ \]
\end{proof}

\begin{remark}
    \begin{description}
        \item{1)}
        Der Satz \ref{4.12} gilt auch für Gleichungen der Form
        \begin{align*}
            -\div(A,(x,\nabla u))&=f \qq \text{in } \Omega\\
                        u&=0 \qq \text{auf }\partial\Omega
        \end{align*}
        falls $A$ die folgenden Bedingungen erfüllt
        \begin{itemize}
            \item \textit{Crathéodory} (Stetigkeit im 2. Argument, messbar im 1. Argument)
            \item  $|A(x,\eta)|\leq C\cdot (g(x)+ |\eta|^{p-1})$ $g\in L^{p'}(\Omega)$
                (Wachstumsbedingung)
            \item $(A(x,\eta)- A(x,\xi))\cdot (\eta-\xi)>0$ f.f.a. $x$, $\eta\neq \xi$
            \item $A(x,\eta)\cdot\eta\geq c|\eta|^p-h(x)$ $h\in L^1(\Omega)$ (Koerzitivität)
        \end{itemize}
        \item{2)}
        Man kann $f$ auch beliebig aus $(W_0^{1,p}(\Omega))'$ wählen. Das ist eine etwas größere Menge
        als $L^{p'}$.
        \item{3)} Aber was ist mit $s<0$?
    \end{description}
\end{remark}

\subsection{Pseudomonotone Operatoren}

In diesem Abschnitt wird die Theorie monotoner Operatoren und die Theorie kompakter Operatoren vereint.
Ein typischer pseudomonotoner Operator ist von der Form
\[
    A=A_1+A_2,
\] 
\[
    A_1: X\ra X' \qq \text{monoton, hemistetig.}
\]
\[
    A_2:X\ra X' \qq \text{strak stetig (d.h. kompakt in separablen, reflexiven Räumen $X$).}
\]

\subsubsection*{Beispiel}
\[
    \lal A_1u,\vp\ral = \int_\Omega |\nabla u|^{p-2} \nabla u \cdot \nabla \vp\qq
    \text{monoton, hemistetig.}
\]
\[
    \lal A_2u , \vp \ral =\int_\Omega su\cdot \vp\qq
    \text{kompakt als Abbildung von $X=W_0^{1,p}$ nach $X'$ für vernünftige $p$, auch für $s<0$.}
\]

\begin{defi} \label{4.13}
Wir sagen $A$ genüge Bedingung (M), falls gilt
\begin{align}\label{24}
    \left.\begin{array}{rl}
    u_n\rightharpoonup u & \text{in $X$} \\
    Au_n\rightharpoonup b & \text{in $X$} \\
    \limsup_{n\ra \infty} \lal Au_n,u_n\ral &\leq\, \lal b,u\ral
    \end{array}\right\}
    \,\Ra \, Au=b
\end{align}
\end{defi}

\begin{lem}\label{4.14}
    Es sei $X$ ein reflexiver, reeller Banachraum,
    \[
        A:X\ra X', \qq B: X\ra X'
    \]
    seien Operatoren.
    Es gilt
    \begin{description}
        \item{1)}
        $A$ monoton, hemistetig $\Ra$ $A$ genügt (M) (\ref{24})
        \item{2)}
        $A$ genügt (M) (\ref{24}), $B$ stark stetig $\Ra$ $A+B$ genügt (M) (\ref{24}).
    \end{description}
\end{lem}

\begin{proof}
    \begin{description}
        \item{1) (siehe \textit{Minty-Lemma} \ref{4.5})} Es  gilt $\forall v \in X$, dass
        \begin{align*}
            0&\leq \lal Au_n-Av, u_n-v \ral = \underbrace{\lal Au_n,u_n\ral}_{\leq \lal b,u\ral-
            \lal Au_n-Av,v\ral} - \lal Av,u_n\ral\\
            \Ra \, 0&\leq \lal b,u\ral- \lal Av,u\ral - \lal b-Av,v\ral= \lal b-Av,u-v\ral
        \end{align*}
        Sei $w \in X$, $v=u-tw$.
        \begin{align*}
            \Ra \, 0\leq \lal b-Av,u-v\ral &= \lal b-A(u-tw),tw\ral \qq t\ra 0 \qq
            \text{(Hemistetigkeit)}\\
            \Ra \, \lal b-Av, w \ral &\geq 0 \qq \forall w\\
            \lal b-Av, w \ral &\leq 0 \qq \text{(durch Einsetzen von $w$, $tw-w$ oben)}\\
            \Ra \, Av&= b .
        \end{align*}
        \item{2)}
        Es gelte $u_n \rightharpoonup u $ in $X$.
        \[
            Au_n+Bu_n \rightharpoonup b \qq \text{in }X'
        \]
        \[
            \limsup_{n\ra \infty} \lal Au_n+ Bu_n,u \ral\leq \lal b,u\ral
        \]
        Es gilt wegen starker Stetigkeit von $B$ , dass
        \[
            Bu_n\ra Bu \qq \text{in }X'
        \]
        somit folgt dass
        \begin{align*}
            \limsup _{n\ra \infty} \lal Au_n,u_n\ral & \leq \lal b-Bu,u\ral\\
            \Ra \, Au&= b - Bu\\
            \Ra \, Au + Bu &= b.
        \end{align*}
        $A$ erfüllt (M) \ref{24}.
    \end{description}
    \[ \]
\end{proof}

\begin{defi}[Pseudomonotonie] \label{4.15}
    Es sei $X$ ein reflexiver, reeller Banachraum
    \[
        A: X\ra X' \qq \text{ein Operator}
    \]
    $A$ heißt \textit{pseudomonoton}, falls gilt
    \begin{align*}
        \left.
        \begin{array}{ll}
            u_n \rightharpoonup u \qq \text{in } X\\ \limsup\lal Au_n,u_n-u\ral \leq 0
        \end{array}
        \right\}
        \, \Ra \lal Au, u-w\ral \leq \liminf \lal Au_n, u_n - w \ral \qq w\in X
    \end{align*}
\end{defi}

\begin{lem}\label{4.16}
    Sei $X$ ein reflexiver, reeller Banachraum, $A,B: X\ra X'$ zwei Operatoren
    \begin{description}
        \item{(i)}
        $A$ monoton, hemistetig $\Ra$ $A$ pseudomonoton.
        \item{(ii)}
        $A$ stark stetig $\Ra$ $A+ B$ pseudomonoton.
        \item{(iii)}
        $A, B$ beide pseudomonoton $\Ra$ $A+ B$ pseudomonoton.
        \item{(iv)}
        $A$ pseudomonoton $\Ra$ $A$ erfüllt (M) (\ref{24})
        \item{(v)}
        $A$ pseudomonoton, lokal beschränkt $\Ra$ $A$ demistetig.
    \end{description}
\end{lem}

\begin{proof}
    \begin{description}
        \item{1)}
        Sei
        \[
            u_n\rightharpoonup u \qq\text{in } X
        \]
        \[
            \limsup \lal Au_n , u_n - u \ral \leq 0.
        \]
        $A$ monoton
        \begin{align*}
            \lal Au_n- Au, u_n -u \ral & \geq 0 \\
            \Ra \, \liminf\lal Au_n , u_n - u \ral &\geq \liminf \lal Au, u_n - u \ral = 0,
        \end{align*}
        wegen schwacher Konvergenz von $u_n$.
        \[
            \lim \lal Au_n , u_n - u\ral =0
        \]
        Es sei $w\in X$, $z=u+t(w-u)$, $t>0$.
        \begin{align*}
            \Ra \, \lal Au_n - Az, u_n - z \ral &= \lal Au_n - Az , u_n -(u+t(w-u)) \ral \geq 0\\
            \Ra \, t\lal Au_n , u-w \ral &\geq \underbrace{-\lal Au_n , u_n - u\ral + \lal Az, u_n - u
                \ral}_{\ra 0} + t\lal Az, u-w\ral\\
            \Ra \, \liminf \lal Au_n , u_n - w \ral &\geq \liminf \lal Au_n, u-w \ral \geq \lal Au, v-w
            \ral
        \end{align*}
        und die Behauptung folgt.
        \item{2)}
        \begin{align*}
            u_n&\rightharpoonup u \qq \text{in } X\\
            \Ra \, Au_n&\ra Au \qq \text{$A$ stark stetig}\\
            \Ra \, \lal Au,u-w\ral &= \lim \lal Au_n ,u_n - w \ral,
        \end{align*}
        womit die Behauptung folgt.
        \item{3)}
        Es sei $u_n \rightharpoonup u$ Folge in $X$ mit
        \[
            \limsup \lal Au_n + Bu_n, u_n -u\ral \leq 0
        \]
        Wir zeigen, dass dann gilt
        \[
            \limsup\lal Au_n , u_n -u\ral \leq 0
        \]
        und
        \[
            \lal Bu_n, u_n - u \ral \leq 0
        \]
        Daraus folgt die Behauptung durch Addition der beiden Ungleichungen der Pseudomonotonität für
        $A$ und $B$ separat.
        
        \noindent Angenommen (zum Widerspruchsbeweis), dass
        \begin{align*}
            \limsup \lal Au_n , u_n - u \ral &> 0\\
            \Ra \, \text{Teilfolge} \, u_{n_k}:\, \lim \lal Au_{n_k}, u_{n_k}-u\ral &=a\\
            \Ra \, \limsup \lal Bu_{n_k},u_{n_k-u}&= \limsup \lal (B+A)u_{n_k} - A u_{n_k}, u_{n_k}-u\ral
            \\
            &\leq \underbrace{\limsup \lal (B+A) u_{n_k}, u_{n_k}-u\ral}_{\leq 0} +\underbrace{
            \lim \lal -Au_{n_k}, u_{n_k} - u\ral}_{=-a} \leq -a\\
            \Ra \, 0= \lal Bu,u-u\ral &\leq \liminf \lal Bu_{n_k} , u_{n_k}-u\ral\\
            &\leq \limsup \lal Bu_{n_k}, u_{n_k}-u\ral \leq -a <0 \qq \lightning,
        \end{align*}
        womit die Behauptung folgt.
        \item{4)}
        Es sei $u_n\rightharpoonup u$ Folge in $X$, so dass
        \[
            Au_n \rightharpoonup b
        \]
        und
        \begin{align*}
            \limsup \lal Au_n,u_n\ral &\leq \lal b,u\ral\\
            \Ra \,  \limsup \lal Au_n, u_n - u\ral &\leq \limsup \lal Au_n, u_n\ral - \underbrace{
            \lim\lal Au_n, u\ral }_{=\lal b,u\ral }\\
            & \leq \lal b,u\ral - \lal b,u\ral = 0.
        \end{align*}
        Damit gilt wegen der Pseudomonotonietät für alle $w\in X$:
        \begin{align*}
            \lal Au, u-w\ral &\leq \liminf \lal Au_n, u_n - w\ral \leq \lal b,u \ral - \lal b,w\ral
            = \lal b, u-w\ral
        \end{align*}
        mit $w'=2u-w$ folgt die umgekehrte Ungleichung, somit gilt
        \[
            \lal Au, u-w \ral = \lal b , u-w \ral \qq w\in X \, \Ra \, Au=b 
        \]
        und damit die Behauptung.
        \item{5)}
        Es gelte $u_n \ra u$ in $X$.
        \[
            \Ra \, (Au_n)_{n\in \N} \text{ beschränkt in $X'$, da $A$ lokal bechränkt.}
        \]
        \[
            \Ra \, \exists \text{ Teilfolge:} \, Au_{n_k}\rightharpoonup b
        \]
        \[
            \Ra \, \lim \lal Au_{n_k},u_{n_k}-u\ral =0
        \]
        Aus der Pseudomonotonität folgt
        \begin{align*}
            \lal Au, u-w \ral &\leq \liminf \lal Au_{n_k} , u_{n_k} - w\ral \qq \forall w \in X\\
            &= \lal b ,  u-w \ral \qq \text{$u_n$ stark-schwach konvergent.}\\
            \overset{\text{wie in 4)}}{\Ra}  Au&= b.\\
            \Ra \, Au_{n_k}&\rightharpoonup Au.
        \end{align*}
        Dieses Argument gilt auch bei vorheriger Auswahl einer weiteren Teilfolge, somit folgt die
        Behauptung, da $A$ demistetig.
    \end{description}
    \[ \]
\end{proof}
