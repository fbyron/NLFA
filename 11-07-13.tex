\subsection*{Konvexe Funktionale}

\begin{theorem}[Unterhalbstetigkeit konvexer Funktionale]\label{5.2}
    Es sei $L:\R^n\times \R\times \ol\Omega\ra \R$ eine glatte, von unten beschränkte Funktion. Weiter
    sei die Abbidung
    \[
        p\mapsto L(p,z,x)
    \]
    konvex $\forall z \in \R$, $x\in \Omega$. Dann ist
    \[
        I(u)=\int_\Omega L(\nabla u, u, x ) \d x
    \]
    im $W^{1,q}(\Omega)$ schwach folgenunterhalbstetig für $1<q<\infty$, d.h.
    \[
        (u_j\rightharpoonup u \qq \text{in $W^{1,p}$}) \, \Ra \, \liminf I(u_j)\geq I(u)
    \]
\end{theorem}

\begin{proof}
(Skizze, siehe auch lineare Funktionalanalysis) Wir nehmen an, dass $L\geq 0$. Wir setzen
\[
    l:= \liminf I(u_j)=\lim I(u_{j_k}) \qq\text{nach übergang zu einer Teilfolge}
\]
Es bleibt zu zeigen, dass $I(u)\leq l$. Dazu wählen wir $u_j\ra u$ in $L^1$ (kompakte Einbettung),
$u_j\ra u$ fast überall. Mit Egoroff existiert für alle $\eps>0$ ein $F_\eps\subset \Omega$, 
$|\Omega\setminus F_\eps|\leq \eps$. Wähle
\[
    G_\eps := \{x\in\Omega:\, |u|+|\nabla u|<\frac1\eps\}
\]
und $E_\eps=G_\eps\cap F_\eps$, es gilt, dass $|\Omega\subset F_\eps|\ra 0$ für
$\eps \ra 0$. Auf der Menge $E_\eps$ können wir in der Variablen $L$ zum Limes übergehen
\[
    \lim_{i\ra \infty} \int_{G_\eps} L(\nabla u, u_h,x) \d x = \int_{G_\eps} L(\nabla u, u ,x)
    \ra 0 \qq (\text{schawache Konvergenz})
\]
und wegen Konvexität gilt
\[
    I(u_n)\geq \int_{G_\eps} L(\nabla u_h,u_h,x)\geq (*)
\]
Mit monotoner Konvergenz und $\eps \ra 0$ folgt die Behauptung.\[ \]
\end{proof}

\begin{remark}
\begin{description}
    \item{(1)} Den ausfühlichen Beweis findet man im Skript zur linearen Funktionalanalysis.
    \item{(2)} Für $L=f(\nabla u) + g(u,x)$ folgt der Beweis einfach durch schwache Unterhalbstetigkeit
    von $\int f$ und starker Unterhalbstetigkeit von $\int g$.
\end{description}
\end{remark}
\begin{theorem}[Existenz von Minimierern] \label{5.3}
Es sei $L$ wie im Satz \ref{5.2} und zusätzlich gelte
\[
    L(p,z,x) \ra |p|^q-\beta
\]
für ein $q>1$, $\beta\in \R$ und für alle $x\in \Omega.$
Dann existiert ein Minimierer $u$ von $L$ in $W_0^{1,q}(\Omega)$

\end{theorem}

\begin{proof}
    Es sei $u_j$ eine Folge, so dass 
    \[
        I(u_j)\ra \inf_{v\in W_0^{1,q}(\Omega)} I(v) \qq (\text{Minimalfolge})
    \]
    Wegen der Koerzitivität von $I$ gilt 
    \[
        \|u_j\|_{W^{1,q}(\Omega)}\overset{\text{\textit{Poincaré}}}{\leq} \int_\Omega |\nabla u|^q
        \overset{\text{koerziv}}{\leq} C
    \]
    Somit existiert eine Teilfolge mit $u_j\rightharpoonup u$ in $W^{1,q}(\Omega)$ und mit Satz
    \ref{5.2} folgt die Behauptung. \[ \]
\end{proof}

\begin{theorem}[Eindeutigkeit von Minimierern]\label{5.4}
    Es gelte
    \begin{description}
    \item{i)}
    $L=L(p,x)$ ist unabhängig von $z$.
    \item{ii)}
    $\exists \Theta > 0$, so dass
    \[
        \sum_{i,j=1}^n\partial_{p_j}\partial_{p_i} L(p,x) \cdot \xi_i\xi_j\geq \Theta |\xi|^2
        \qq \forall p,\, \xi \in \R^n, \,\forall x\in \Omega.
    \]
    (oder eine äquivalente Tensorformulierung im Fall von vektorwertigen $u$.)
    Dann ist der Minimierer $u\in W_0^{1,q}(\Omega)$ eindeutig.
    \end{description}
\end{theorem}

\begin{proof}
    Angenommen $u,\tilde u \in W_0^{1,q}(\Omega)$ minimieren beide das Funktional $I$, dann ist
    \[
        v:=\frac{u+\tilde u}2 \in W_0^{1,q}(\Omega)
    \]
    \textbf{Behauptung:} $I(v)\in \frac{I(u)+ I(\tilde u)}{2}$, mit strikter Ungleichung, falls nicht gilt,
    dass $u=\tilde u$ f.ü. in $\Omega$.
    \begin{proof}
    Es gilt wegen der gleichmäßigen Konvexität von $L$, dass
    \[
        L(p,x)\geq L(q,x) + \partial _p L(q,x)|p-q|+ \frac\Theta2 |p-q|^2.
    \]
    Wir setzen $q=\frac{\nabla u + \nabla \tilde u}2$, $p=\nabla u$. Es folgt, dass
    \[
        I(v) + \int_\Omega \partial _pL\left(\frac{\nabla u + \nabla \tilde u}2, x\right)\cdot \left( 
        \frac{\nabla u-\nabla \tilde u}2\right) \d x + \frac\Theta8 \int_\Omega |\nabla u - \nabla \tilde u|^2
        \d x\leq I(u).
    \]
    Ähnlich folgt aber, dass
    \[
        I(v)+\int_\Omega \partial _p L\left( \frac{\nabla u + \nabla \tilde u}2, x \right) \left( \frac{\nabla \tilde u
        -\nabla u}2\right) \d x +\frac\Theta8\int_\Omega|\nabla u - \nabla \tilde u|^2 \d x \leq I(u).
    \]
    Somit gilt, dass
    \begin{align*}
        I(v)&\leq \frac{I(u)+I(\tilde u)}2 -\frac\Theta8 \int_\Omega |\nabla u-\nabla \tilde u|^2 \d x\\
        \Ra \, \nabla u - \nabla \tilde u &= 0 \qq \text{f.ü. auf }\Omega\\
        \Ra \, u&=\tilde u \qq \text{f.ü. auf $\Omega$, da $u,\tilde u\in W_0^{1,q}(\Omega)$}.
    \end{align*}
    Und die Behauptung folgt. \[ \]
    \end{proof}
    \[ \]
\end{proof}
Wir können für $\Omega$ mit hinreichend glatten Rand Minimierer auch in der Klasse der Funktionen
\[
    \ms A:= \{u\in W^{1,q}:\, u=g \, \text{auf } \partial \Omega\}
\]
suchen, wobei die Randwerte im Spursinne gegeben sind

\noindent $\ra$ siehe \textit{Evans, Partial Differential Equations, Chapter 5}

\noindent Alternativ sei $g\in C^1(\ol \Omega)$, dann können wir statt $L$ den Lagrangen
$\tilde L:= L(\nabla u-\nabla g,u-g,x)$
……
Die Konvexitäts- und Koerzitivitätseigenschaften übertragen sich von $L$ auf $\tilde L$, Sätze \ref{5.1}
bis \ref{5.4} sind aber auch auf $\tilde L$ anwendbar.
Die Klasse von Randwerten, die sich so behandeln lassen sind aber \textit{kleiner} als die der Randwerte
im Spursinne.

\begin{beispiel}
    \[
        L(p,z,x)= \frac{|p|^q}{q} + \frac{sz^2}2 -f(x) z
    \]
    Die schwache Euler-Lagrange-Gleichung zu $L$ lautet:
    \[
        -\int_\Omega |\nabla u|^{q-2} \nabla u \cdot \nabla v + suv\d x = \int_\Omega fv\d x \qq \forall
        v\in W_0^{1,q}(\Omega)
    \]
    $\leadsto$ $q$-Laplace.
\end{beispiel}

\begin{remark}
    Es sei $V$ ein Banachraum, $I:V\ra \R$ ein (strikt-) konvexes Funktional, das $\forall u\in V$
    Fréchet-differenzierbar ist. Es gilt
    \begin{align*}
        I':V\ra \ms L(V,\R)= V'
    \end{align*}
    und $A:= I'$ ist (strikt-) monoton.
\end{remark}

\begin{proof}
    Sei $u,v\in V$, es gilt
    \begin{align*}
        I(tu+(1-t)v) &\leq tI(u) + (1-t)I(v)\\
        \Ra \, I(v)&\geq \underbrace{\Lal Au,v-u \Ral}_{=I'} + I(u)\\
        I(u)&\geq \Lal Av, u-v\Ral + I(v)\\
        \Ra \, 0&\leq \Lal Au-Av,u-v \Ral \, \Ra \, A \, \text{monoton},
    \end{align*}
    strikte Monotonität ebenso. \[ \]
\end{proof}

Für allgemeine (unterhalbstetige) Funktionale $I:V\ra\R$ existiert das sogennante Subdifferential
$\partial I$, definiert als
\[
    \left\{A\in V':\, I(v)\leq \Lal A,v-u\Ral + I(u)\, \forall u\in V\right\}
\]
und das Subdifferential ist ein maximal monotoner Operator. (\textit{Satz von Rockefellar})

\subsection*{Nichtkonvexe Funktionale, ein Beispiel aus der nichtlinearen Elastizitätstheorie}

\subsubsection*{Motivation eines nichtkonvexen Funktionals}

Wir betrachten einen Festkörper in einem Gebiet $\Omega\subset\R^3.$ Dieser besteht aus Atomen mit
Gittervektoren $e_1$, $e_2$, $e_3$.\\
\noindent Energiefunktion(Lagrangian):
\[
    L=L(e_1,e_2,e_3).
\]
\textbf{Annahme (Cauchy-Born-Hypothese)} Affine Randbedingung für Atome $\Ra$ Im Inneren affine
Anordnung. Verformung des Körpers durch eine glatte Funktion
\[
    y:\Omega\ra \R^3
\]
gegeben ist. Aus \textit{Cauchy-Born-Hypothese} folgt, dass
\[
    (e_1,e_2,e_3)= (\nabla y)(\ol e_1, \ol e_2, \ol e_3) \qq(\text{undeformiert})
\]
$\leadsto$ $L=L(\nabla y)$
