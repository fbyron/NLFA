\item % 3)
Wir suchen $v\in L^{p'}(S;X')$, so dass
\[
    g(t)=\int_{t_0}^t v(\tau) \d \tau
\]
Wir können $S$ bis auf eine Nullmenge als Vereinigung abgeschlossener Intervalle darstellen, somit genügt
es, die Aussage für $S=[t_0,t_1]$ zu beweisen. Es sei $T=t_1-t_0$ und wir setzen
\[
    g_{n,i} = g(\underbrace{t_0+2^{-n}T}_{(**)}i) \qq\text{für $i=0,1,…,2^n$ $n\in\N$},
\]
und
\[
    u_n(t)=\frac1{2^{-n}T} (g_{n,i+1}-g_{n,i})\qq \text{für $2^{-n}Ti\leq t-t_0<2^{-n}T(i+1)$}
\]
Es gilt wegen (\ref{28}), dass $2^n-1$
\begin{align}\label{29}
    \|v_n\|_{L^{p'}(S;X')} = \left( \sum_{i=0}^{2^n-1}\left\| \frac1{2^{-n}T}(g_{n,i+1}-g_{n,i}) \right\|
    ^{p'}_{X'}\cdot 2^{-n}T\right)^{\nicefrac1{p'}} \overset{(\ref{28})}{\leq} \|\vp\|_{(L^p(S,X))'}
\end{align}
Wir zeigen nun, dass f.f.a. $t\in S$ die Folge
\[
    (\|u_n(t)\|_{X'})_{n\in\N}
\]
beschränkt bleibt. Es sei $N_0:=\{t\in S: \, \sup_{n}\|u_n(t)\|_{X'}=\infty \}$ und
\[
    S_{r,K}:= \{t\in S: \, \sup_{n\leq r} \|u_n(t)\|_{X'}\geq K\} \qq r,K\in \N.
\]
Es gilt $S_{r,K}\subset S_{p,K}$ für $p\geq r$ und
\[
    N_0 = \bigcap_{K=1}^\infty \bigcup_{r=1}^\infty S_{r,K}
\]
Nachdem $u_n$ sthw. konstant ist, besteht $S_{r,k}$ aus disjunkten Intervallen der For $[t_i,t_i+n_i)$ 
$i=1, … , m$. Es gilt
\[
    K^{p'}|S_{r,K}|\leq \sum_{i=1}^m \left( \left\| \frac{g(t_i+h_i)-g(t_i)}{h_i}\right\|^{p'}_{X} h_i
            \right) \overset{(\ref{28})}{\leq} \|\vp\|
\]
Nachdem $S_{r,K}$ in $r$ wächst folgt
\begin{align*}
    K^{p'} \left|\bigcup_rS_{r,K}\right| &\leq \|\vp\|_{L^p(S;X)'}\\
    \Ra \, |N_0|&=\left| \bigcap_K\bigcup_r S_{r,K} \right| \leq \inf_{K}\left|\bigcup _r S_{r,K} \right|
    =0
\end{align*}
Nun sei $\{z_j\}_{j\in\N}$ dicht in $X$ und wir definieren
\[
    a_j(t)=\lal g(t), z_j\ral \qq t\in S
\]
\[
    b_{j,u}(t) =\lal u_n(t), z_j \ral \qq t\in S 
\]
Es gilt $b_{j,n}\in L^{p'}(S;\R)$ mit gleichmäßiger beschränkter Norm.
\[
    \|b_j^n\|_{L^{p'}(S;\R)}^{p'}=\int_S|\lal u_n(\tau), z_j\ral|^{p'}\leq \int_S \|u_n\|_X^{p'} \cdot
    \|z_j\|_X^{p'} \leq \|z_j\|_X^{p'} \|u_n\|_{L^{p'}(S;X')}
\]
Wir können somit sukzessive eine Teilfolge auswählen, so dass für alle $j\in \N$ gilt
\[
    b_j^n\rightharpoonup b_j \in L^{p'}(S;\R).
\]
\textbf{Behauptung 1:} $\forall t\in S$ gilt
\[
    \int_{t_0}^t b_j(\tau) \d \tau= a_j (t).
\]
\begin{proof}
    Es sei $\{\tilde S_n\}_{n\in\N}= \{t_{n,i}\}_{n\in\N}$, $i\in \{0, …, 2^n\}$, die Menge aller Punkte,
    welche durch die Intevallteilung ($**$) erreicht werden. Offensichtlich liegt $\{\tilde S_n\}
    _{n\in \N}$ dicht in $S$. Nun sei $\tilde t\in S_n$, damit gilt $\tilde t\in S_m$ $\forall m\geq n.$
    Es gilt für $m\geq n$
    \begin{align*}
        \int_{t_0}^{\tilde t} b_{m,j}(\tau) \d \tau &= \int_{t_0}^{\tilde t} \lal u_m,z_j\ral 
        = \sum_{i=1}^{2^m\frac{\tilde t}T} \frac1{2^{-m}T}\lal g_{m,i+1} - g_{m,i}, z_j\ral \cdot 2^{-m}T
        \\
        &= \lal g_{m, 2^m\frac{\tilde t}T}, z_j\ral = \lal g(t_0+ \tilde t), z_j \ral = a_j (\tilde t)
    \end{align*}
    Aber wegen $b_n^j\rightharpoonup b_j$ weil $\chi_{(0,\tilde t)}$ eine Testfunktion in $L^p(S;\R)$
    ist, gilt
    \[
        \int_{t_0}^{\tilde t} b_j(\tau) \d \tau= a_j(\tau)
    \]
    Mit Stetigkeit von $a_j(t)$ und von $\int_{t_0}^t b_j(\tau)\d\tau$ folgt die Behauptung.\[ \] 
\end{proof}
\noindent\textbf{Behauptung 2:} Es gilt in diesem Fall (d.h. $a_j(t)=\int_{t_0}^tb_j(\tau)\d\tau$ f.f.a
$t\in S$, und $b_j\in L^1([t_0,t_1);\R)$) dass $a_j$ fast überall differenzierbar ist und
\[
    b_j(t)=a'_j(t) \qq \forall t\in S\setminus N_j, \, |N_j|=0.
\]
\begin{proof}
    Übungsaufgabe. \[ \]
\end{proof}

Nun sei $t\in S\setminus \bigcup_{j=0}^\infty N_j$ und $i_n\in \N$ sei für $n\in \N$ so gewählt,
dass gilt $2^{-n}Ti_n\leq t-t_0<2^{-n}T(i_n+1)$. Dann folgt:
\[
    \lim_{n\ra \infty} b_{j,n}= \lim_{n\ra \infty} \lal u_n(t),z_j\ral = \lim_{n\ra \infty}
    \lal \frac1{2^{-m}T} (g_{n,i_n+1}-g_{n,i_n}), z_j \ral = a'_j(t).
\]
Damit folgt aber, dass $\forall t\in S\setminus \bigcup_{j=0}^\infty N_j$ gilt
\[
    u_n(t) \rightharpoonup v(t)  \qq  \text{in } X'. 
\]
Für $t\in \underbrace{\bigcup_{i=0}^\infty N_j}_{\text{Nullmenge}}$ setzen wir $v(t)=0$. Die so
definierte Funktion $v:S\ra X'$ ist Bochner-messbar (als Limes von Treppenfunktionen) und es gilt für
$t\in S$, dass
\[
    \|u(t)\|_{W'}\leq \liminf_{n\ra \infty} \|v_n(t)\|_{X'}
\]
Daraus und aus (\ref{29}) folgt mit Fatou, dass
\begin{align}\label{30}
    \|u\|_{L^{p'}(S;X')}^{p'} \leq \liminf\int_S\|u_n(t)\|_{X'}^{p'} \leq \|\vp\|_{(L^p(S;X))'}
\end{align}
Also ist $u\in L^{p'}(S;X')$. Wegen Behauptung 1 gilt für alle $j\in \N$, dass
\[
    \lal g(t),z_j\ral = a_j(t) = \int_{t_0}^t a_j'(\tau)\d\tau=\int_{t_0}^{t} \lim_{n\ra \infty}
    \lal u_n(\tau) , z_j \ral \d\tau= \int_{t_0}^t \lal u(\tau), z_j\ral \d\tau
    = \lal \int_{t_0}^t u_(\tau) \d\tau, z_j\ral.
\]
Also ist
\[
    g(t)= \int_{t_0}^tu(\tau) \d\tau
\]
\item % 4)
Es sei nun $u$ eine Treppenfunktion aus $L^p(S;X)$ auf einer endlichen Menge von Intevallen, also
\[
    u(t)=\begin{cases} x_i & \text{für } s_i<t\leq t_i, \, i=1,…,m \\
         0& \text{sonst}\end{cases}
\]
Damit folgt sofort
\begin{align*}
    \vp(u)&=\vp\left(\sum_{i=1}^m (u_{t_i,x_i}- u_{s_i,x_i} )\right)
    = \sum_{i=1}^m \lal g(t_i)-g(s_i),x_i\ral\\
    &= \sum_{i=1}^m \lal \int_{s_i}^{t_i} u(t)\d t, x_i\ral = \sum_{i=1}^m \int_{s_i}^{t_i}
    \lal v(t) , u(t)
\end{align*}
Nachdem diese Treppenfunktionen dicht liegen in $L^p(S;X)$ und dank \textit{Hölder'scher Ungleichung}
die Abbildung
\[
    u\mapsto \int_S\lal v(t),u(t)\ral \d t
\]
stetig ist bzgl. $u\in L^p(S;X)$ gilt
\[
    \vp (u)=\int_S \lal v(t),u(t)\ral \d t \qq \forall u\in L^p(S;X)
\]
Wegen \textit{Hölder} gilt ebenfalls, dass
\[
    \|\vp\|_{(L^p(S;X))'} \leq \|v\|_{L^{p'}(S;X')}.
\]
Mit (\ref{30}) folgt also
\[
    \|\vp\|_{(L^p(S;X))'} = \|v\|_{L^{p'}(S;X')}.
\]
Eindeutigkeit der Darstellung folgt dann daraus, dass wegen Normgleichheit das Nullfunktional $\vp=0$
nur durch die Nullfunktion $v=0$ dargestellt werden kann.
\end{enumerate}
\[ \]
\end{proof}

\begin{remark}
    Die Zuordnung $T:\vp \mapsto v$.
    \[
        T:(L^p(S;X))'\ra L^{p'}(S;X')
    \]
    ist eine lineare, surjektive Isometrie.
\end{remark}

\begin{remark}
    Mit einigen kleinen Änderungen funktioniert der Beweis auch für $X$ nur separabel. Mit weiteren
    kleinen Änderungen funktioniert der Beweis auch für
    \[
        \vp \in (L^1(S;X))'.
    \]
    Die Funktion $v$ ist dann in $L^\infty(S;X)$.
\end{remark}

\begin{remark}
    Der Satz gilt auch für $X$  nur reflexiv. Der Beweis ist dann aufwendiger. Im Allgemeinen gilt der
    Satz, wenn $X$ die sog. \textit{Radon-Nilkodyn Eigenschaft} besitzt.
\end{remark}
