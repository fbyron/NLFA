\subsection{Der Fixpunktsatz von Kakutani und eine Anwendung in der Spieltheorie}

Im Folgenden beweisen wir die sogenannten \textit{Nash-Gleichgewichte} in n-Personen.
Als Vorbereitung ist es notwendig, den \textit{Brouwer'schen Fixpunktsatz} auf mengenwertigen Funktionen
zu verallgemeinern.

Für $K\subset \R^n$ konvex, kompakt, bezeichnen wir mit $\CS(K)$ die Menge der konvexen (Teil-)Mengen
von $K$.

\begin{theorem}[Kakutani]\label{2.16}
    Sei $K\subset \R^n$ kompakt, konvex $f:K\ra \CS(K)$. Falls die Menge
    \[
        \Gamma := \{(x,y)\in K\times K\, |\, y\in f(x)\}
    \]
    in $K\times K$ abgeschlossen ist, dann existiert $x\in K$ mit $x\in f(x)$.
\end{theorem}

\begin{proof}
    In einer Dimension: $K=[v_1,v_2]$, wir wählen $y_j\in f(v_j)$ und definieren
    \[
        f^1(x)=\sum_{j=1}^{2} \lambda_j(x)y_j,
    \]
    wobei $\lambda _j(x)$ die baryzentrischen Koordinaten von $x$ bezeichne, d.h.
    \[
        \lambda_j(x)\geq 0, \qq \sum_{i=1}^2 \lambda_1(x)=1,\qq x=\sum_{j=1}^2 \lambda_j(x)v_j
    \]
    Nach Konstruktion ist $f^1:K\ra K$ stetig, besitzt mit Brouwer also einen Fixpunkt. Das hilf
    noch nicht viel. Wir betrachten die $k-1$-te baryzentrische Unterteilung der Menge $K$ gegeben
    durch die Vertizes 
    \[
        \{v_i\}_{i=1}^{2^{k-1}+1}
    \]
    Wir wählen wieder $y_j\in f(v_j)$ und definieren $f^k(v_j)=g_j$ interpolieren auf den unterteilten
    Intervallen wie gehabt durch baryzentrische Koordintaten. Es gilt
    \[
        f^k: K\ra K
    \] 
    ist stetig, also mit Fixpunkt
    \begin{align}\label{7}
        x^k=\sum_{i=1}^2 \lambda_i^kv_i^k= \sum_{i=1}^2 \lambda_i^ky_i^k \qq \text{mit} \qq 
        y_i^k=f^k(v_i^k)
    \end{align}
    in einem Teilintervall  (in einem Simplex der $k$-fachen baryzentrischen Koordinaten). Es gilt:
    \[
        \left(x^k,\lambda_1^k,\lambda_2^k,y_1^k,y_2^2\right)\in K\times[0,1]^2\times K^2
    \]
    Somit konvergiert eine Teilfolge in $K$ gegen
    \[
        \left(x^0,\lambda_1^0,\lambda_2^0,y_1^0,y_2^0\right)
    \]
    Nachdem die Teilintervalle zu Punkten degenerieren gilt:
    \[
        v_i^k\ra x^0 \qq i=1,2
    \]
    Wir haben 
    \[
        \left(v_i^k,y_i^k\right)\in\Gamma\ra \left(v_i^0,y_i^0\right)\in \Gamma \qq 
        \text{wegen Abgeschlossenheit von }\Gamma
    \]
    Somit gilt $y_i^0\in f(x^0)$, $i=1,2$. Mit (\ref{7}) folgt
    \[
        x_0 =\lim_{k\ra\infty} \sum_{i=1}^2\lambda_i^ky_i^k=\sum_{i=1}^2 \lambda^0_iy_i^0 
        \in f\left(x^0\right),
    \]
    $f\left(x^0\right)$ ist konvex. 

    In $n$ Dimensionen nehmen wir zunächst an, dass $K$ ein Simplex (Konvexe Hülle von $n+1$
    nichtgenerierten Vertizes) ist. Wir wenden wiederholt die Baryzentrische Unterteilung an,
    die der Durchmesser der Simplizes schrumpft mit einem Faktor $\frac{n}{n+1}$.
    Zur $k$-ten baryzentrischon Unterteilung mit Vertizes $\{v_j\}_{j=1}^{N_k}$. Wählen wir wieder
    $y_j\in f(v_j)$ mit $f^k(x)$ durch Interpolation mittels baryzentrischen Koordinaten in jdn.
    Unter-Simplex. Wir erhalten wieder Fixpunkte $x^k$, und eine Teilfolge von
    \[
        (x^k,\lambda_1^k,…,\lambda_{n+1}^k,y_1^k,…,y_{n+1}^k)
    \]
    konvergiert im „richtigen“ Unter-Simplex. Der Rest des Beweises folgt exakt wie im eindimensionalen
    Fall. Ist $K$ kein Simplex, so können wir $f^k$ jeweils auf einen Simplex $\tilde K\supset K$ stetig
    fortsetzen und verfahren wie im Beweis des \textit{Brouwer'schen Fixpunktsatzes}.
    \[ \]
\end{proof}

\subsubsection*{Spieltheorie:}

Ein $n$-Personen Spiel besteht aus $n$ Spielern, wobei der $t$-te Spieler $m_i$ mögliche Aktionen
ausführen kann. Die Menge der möglichen Aktionen des $i$-ten Spielers bezeichnen wir mit
$\Phi_i=\{ 1,…,m_i \}$. Nachdem jeder Spieler eine Aktion ausgeführt hat, wird abgerechnet. Der $i$-te
Spiele bekommt den Payoff
\[
    R_i(\vp), \qq \vp=(\vp_1,…,\vp_n)\in \Phi=\prod_{j=1}^n \Phi_j
\]
Wir betrachten den Fall, dass das Spiel häufig ausgeführt wird, und jdn. Spieler seine Aktion nach einer Wahrscheinlichkeitsverteilung (einer Strategie) auf $\Phi_i $ wählt. Diese bezeichenen wir mit
\[
    S_i=\{s_i^1,…,s_i^{m_i}\},\qq s_i^k\geq 0, \qq \sum_{k=1}^{m_i}s_i^k=1 \qq \forall i
\]
Die Menge aller möglichen Strategien für den $i$-ten Spieler nennen wir $S_i$. $S_i^k$ ist die
Wahrscheinlichkeit, dass der $i$-te Spieler die Aktion $k$ wählt. Die Wahrscheinlichkeit, dass jeder
Spieler eine bestimmte Aktion $s_i^k$ wählt ist
\[
    S(\vp)=\prod_{i=1}^n S_i(\vp), \qq S_i(\vp)=s_i^{k_i}
\]
mit $\vp=(k_1,…,k_n)\in \Phi$. Wir nehmen hier an, dass die Spieler ihre Aktionen unabhängig voneinander
wählen. Der mittlere Payoff für Spieler $i$ ist dan der Erwartungswert für $R_i(\vp)$. Wir nennen diesen
$R_i(s)$ und erhalten
\[
    R_i(s)=\sum_{\vp\in\Phi}s(\vp) K_i(\vp)
\]
Nach Konstruktion ist
\[
    R_i(s)
\]
stetig in $s$. Was ist nun die optimale Strategie für Spiel $i$? Falls alle anderen Spiele nach einer
bekannten  Strategie handeln, is es optimal $\ol{S}_i$ so zu wählen, dass
\begin{align}\label{8}
    R_i(s\setminus \ol{s_i})=\max_{\tilde s_i \in S_i} R_i(s\setminus \tilde s_i)
\end{align}
$s\setminus \tilde s_i$: Strategiekomination, die sich ergibt, wenn man $s_i$ durch  $\tilde s_i$
ersetzt. Wir bezeichnen mit $B_i(s)$ die Menge aller Strategien $\ol s_i\in S_i$, die $\ref{8}$ erfüllen.
Insbesondere gilt $\ol s_i\in B_i(s)$ genau dann wenn $\ol s_i^k=0$ falls
\[
    R_i(s\setminus (0,…,1,…)) < \max_{1<l<m_i}R_i(s\setminus(0,…,1,…)) \qq \text{mit der 1 als $l$-ter
    Eintrag}
\]
Insbesondere existiert immer eine reine (immer die selbe Aktion wird gewählt) optimale Strategie
zu gegebenem $s$. Weiter sieht man, dass $B_i(s)$ konvex ist und nicht leer. Es seien nun $s,\ol s\in S$.
Wir nennen $\ol s$ die optimale Antwort gegen $s$, falls $\ol s_i\in B_1(s)$ $i=1,…,n$.
Es sei $B(s)$ die Menge aller optimalen Antworten gegen $S$, wir haben
\[
    B(s)=\prod_{i=1}^n B_i(s).
\]

\begin{defi}[Nash-Gleichgewicht]
    Eine Strategiekombination $\ol s\in S$ heißt Nash-Gleichgewicht, falls gilt
    \[
        \ol s \in B(\ol s)
    \]
    d.h. $\ol s$ ist eine optimale Antwort gegen sich selbst. Anders gesagt, $\ol s$ ist ein 
    Nash-Gleichgewicht, falls kein Spieler durch eigene Änderung seiner Strategie seinen Payoff erhöhen
    kann.
\end{defi}

\subsubsection*{Beispiele:}

%\begin{description}
    \textbf{1. Das Gefangenendilemma:}
    Zwei Gefangene können entweder schweigen, oder mit der Polizei kooperieren, das heißt den jeweils
    anderen belasten.
    \[
        \begin{array}{c|cc||c|cc} R_1& k_2& s_2& R_2&k_2&s_2\\ \hline k_1&0&2&k_1&0&-1\\
            s_1&-1&1&s_1&2&1\end{array}
    \]
%\end{description}
\begin{theorem}\label{2.18}
    Jedes $n$-Personen Spiel besitzt ein Nash-Gleichgewicht.
\end{theorem}

\begin{proof}
    Wir betrachten $s\mapsto B(s)$, $S$ ist kompakt, konvex, $B(s)\subset S$ ist nicht leer, konvex für
    alle $s$.
    \[
        s^n\in S, \qq \ol s^m\in B(s^m), \qq s^m\ra s, \qq \ol s^k\ra \ol s
    \]
    Wir haben $R_i(s^m\setminus \tilde s_i)\leq R_i(s^m\setminus \ol s_m)$ mit Stetigkeit von $R_i$ folgt
    sofort
    \[
        R_i(s\setminus \tilde s_i) \leq R_i(s\setminus \ol s) \qq \forall s_i\in S_i
    \]
    Somit ist $\ol s \in B(s)$ und $\Gamma= \{ (s,B(s))\subset S^2 \}$ ist abgeschlossen.
    Der \textit{Fixpunktsatz von Kakutani} garantiert Existenz von
    \[
        \ol s\in S \qq \text{mit} \qq \ol s \in B(\ol s).
    \]
\end{proof}
