\newpage
\chapter{Der Brouwer'sche Abbildungsgrad}

\section*{Motivation}
\begin{description}
    \item[Ziel:]
    $f(x)=0$ zu lösen für $f:U\subset X \ra X$, $X$ Banachraum.
    \item[Frage:]
    Existenz/Anzahl der Lösungen
    \item[Rückblick auf Funktionentheorie:]
    Sei $z_0\in \C$
    \[
        n(\gamma,z_0) = \frac{1}{2\pi i} \int_\gamma \frac1{z-z_0} \d z
    \]
\end{description}
$\leadsto$ Verallg.: $f\in \mc H(\C)$. $0\nin f(\gamma)$
\[
    n(f(\gamma),0)=\frac1{2\pi i} \int_\gamma \frac{f'}{f} \d z = \sum_k n(\gamma,z_k) \alpha_k
\]
wobei $f(z_n)=0$, $\alpha_k$ Vielfachheiten.

\noindent Ziel: Verallg. des Begriffs „Umlaufzahl“ für Abb. $f:U\subset \R^n\ra \R^n$

\section{Die Determinantenformel}

\subsection*{Notation}

$U\Subset \R^n$ offen, $f\in C^1(U,\R^n)$
\begin{align*}
    J_f(x)&:=\det \d f(x)\\
    \RV(f)&:=\{ y\in \R^n \,| \, \forall \, x\in f^{-1}(y), \, J_f(x)\neq 0 \}\\
    \CV(f)&:= \R^n \setminus \RV(f)\\
    D_y^k(\ol U, \R^n)&:=\{f\in C^k(\ol U, \R^n)\,|\,y\nin f(\partial U)\}\\
    D_y(\ol U, \R^n)&:=D_y^0(\ol U, \R^n)
\end{align*}
$\tau(\R^n)$ bezeichne die Topologie auf $\R^n$.
\begin{defi}
    Eine Abbildung
    \[
        \deg: \bigcup_{U\in \tau(\R^n),\, y\in \R^n} (D_y(\ol U, \R^n)\times \{U\}\times \{y\}\ra \R,
    \]
    d.h.
    \[
        \deg=\deg(f,U,y) 
    \]
    heißt Gradabbildung, falls
    \begin{description}
        \item[(D1)]
        Translationsinvarianz: 
        \[
            \deg(f,U,y)=\deg(f-y,U,0)
        \]
        \item[(D2)]
        Normalisation:
        \[
            \deg(\Id,U,y)=1 \qq \forall y\in U
        \]
        \item[(D3)]
        Additivität: Seien $U_1,U_2\subset U$ offen und disjunkt, sodass
        \[
            y\nin f(\ol U\setminus(U_1\cup U_2)),
        \]
        dann gelte
        \[
            \deg(f,U,y)=\deg(f,U_1,y)+\deg(f,U_2,y)
        \]
        \item[(D4)]
        Homotopieinvarianz: 
        \[
            H(t)=(1-t)f+tg \in D_y(\ol U,\R^n) \qq \forall \, t\in [0,1]\, \Ra \, \deg(f,U,y)=\deg(g,U,y)
        \]
\end{description}
\end{defi}

\begin{theorem}\label{2.1}
    Sei $\deg$ eine Gradabbildung. Dann gilt
    \begin{description}
        \item{(i)}
        $\deg(f,\varnothing,y)=0$ und
        \[
            \deg(f,U,y)=\sum_{i=1}^N \deg(f,U_i,y)
        \]
        falls 
        \[
            y\nin f(\ol U\setminus\bigcup_{i=1}^NU_i),
        \]
        wobei $U_i\subset U$ offen und disjunkt für $1\leq i \leq N$.
        \item{(ii)}
        $y\nin f(U) \, \Ra \, \deg(f,U,y)=0$
        \item{(iii)}
        $|f(x)-g(x)|<\dist(y,f(\partial U)) \, \forall \, x\in \partial U \, \Ra \, \deg(f,U,y)
        =\deg(g,U,y)$
    \end{description}
\end{theorem}

\begin{proof}
    \begin{description}
    \item{(i)}
    Sei $U_1=U$, $U_2=\varnothing$, einsetzen in \textbf{(D3)}
    \[
        \Ra \, \deg(f,\varnothing,y)=0
    \]
    $i=1$: $U_2=\varnothing$
    \[
        \Ra \, \deg(f,U,y)=\deg(f,U_1,y)
    \]
    $i>1:$ Induktion mittels \textbf{(D3)}
    \item{(ii)}
    \begin{align*}
        y\nin f(U) \Ra y &\nin f(\ol U)\, \Ra \, y\nin f(\ol U\setminus \varnothing)\\
        \overset{(i)}{\Ra } \, \deg (f,U,y)&=0 \qq (i=1, U_1=\varnothing)
    \end{align*}
    \item{(iii)}
    Sei $H(t,x):=(1-t)f(x)+tg(x)$ und sei $x\in \partial U$
    \begin{align*}
        \Ra \, |H(t,x)-y|&= |f(x)-y+t(g(x)-f(x))|\\
            &\geq |f(x)-y| - |g(x)-f(x)|\\
            &\geq \dist(y,f(\partial U))- |g(x)-f(x)| >0\\
        \Ra \, y&\nin H(t,\partial U) \qq \forall \, t\,\Ra \,H(t)\in D_y(\ol U,\R^n)\\
        &\overset{(\textbf{D4})}{\Ra} \text{ Behauptung.}
    \end{align*}
    \end{description}
    \[ \]
\end{proof}

\begin{theorem}\label{2.2}
    \begin{description}
        \item{i)} $\deg(\cdot,U,y)$ ist lokal konstant in $D_y(\ol U,\R^n)$
        \item{ii)} $\deg(f,U,\cdot)$ ist lokal konstant in $\R^n\setminus f(\partial U)$
        \item{iii)} Seien $H:[0,1]\times \ol U\ra \R^n$ und $y:[0,1]\ra \R^n$ stetig (d.h. $H$ ist eine
            Homotopie zwischen $H(0)=H(0,\cdot)$ und $H(1)$), so gilt
        \[
            \deg(H(0),U,y(0))=\deg(H(1),U,y(1)),
        \]
        falls $H(t)\in D_{y(t)}(\ol U, \R^n)$ $\forall t \in [0,1]$
    \end{description}
\end{theorem}

\begin{proof}
    Beachte: $D_y(\ol U, \R^n)$ ist offen in $C^0(\ol U, \R^n)$
    \begin{description}
    \item{i)}
        $\|f-g\|_{C^0,\ol U} <\eps \, \Ra \, |f(x)-g(x)|<\eps \, \forall \, x\in\partial U$ mit
        \[
            \eps:= \dist(y,f(\partial U)) \, \Ra \, \deg (f,U,y)=\deg (g,U,y)
        \]
    \item{ii)}
        Sei $y_0\nin f(\partial U)$ und $y\in B_{\dist(y_0,f(\partial U))}(y_0)\subset \R^n\setminus
                f(\partial U)$
        \begin{align*}
            \Ra \, \|(f-y)-f\|&<\dist(y_0,f(\partial U))\\
                \overset{i)}{\Ra} \, \deg(f-y,U,y_0) &= \deg(f,U,y_0)\\
            \overset{\textbf{(D1)}}{\Ra } \, \deg (f,U,y_0+y)&= \deg(f,U,y_0)
        \end{align*}
    \item{iii)}
        $H$ ist gleichmäßig stetig
        \[
            \Ra \, H:[0,1]\ra C^0(\ol U,\R^n)\qq t\mapsto H(t,\cdot)
        \]
        ist auch stetig. $H$ ist ein stetiger Weg in $D_y(\ol U,\R^n)$. Sei $y$ fest
        \[
            \Ra \, \deg(H(t,\cdot),U,y)=\Const,
        \]
        weil $\deg(\cdot,U,y)$ konst. auf Zusammenhangskomponenten ist. Für $y=y(t):$
        \begin{align*}
            \deg(H(0),U,y(0))&=\deg(H(0)-y(0),U,0)=\deg(H(t)-y(t), U, 0) \qq \forall t\in [0,1]\\
            &= \deg(H(1),U,y(1)).
        \end{align*}
    \end{description}
    \[ \]
\end{proof}

\begin{lem}\label{1.3.5}
    Zwei nichtsinguläre Matrizen $M_1,M_2\in \Gl(n)$ sind genau dann homotop in $\Gl(n)$, falls
    \[
        \sign \det M_1 = \sign\det M_2
    \]
\end{lem}

\begin{proof}
    \begin{description}
    \item{„$\Ra$“} Sei $M\in \Gl(n)$. Wegen der Linearität von $\det $ in Zeilen können elementare
    Zeilenumformungen mit Hilfe stetiger Deformationen in $\Gl(n)$ erzeugt werden.
    \[
        M \, \leadsto\diag(m_1,…,m_m), \qq \text{mit} \qq |m_i|=1
    \]
    weil $\sign \det M_1=\sign \det M_2$.
    \[
        H(t):= \begin{pmatrix} \pm \cos (\pi t)& \mp \sin(\pi t)\\ \sin(\pi t) & \cos(\pi t)\end{pmatrix}
    \]
    ist eine Homotopie in $\Gl(n)$ von $\diag(\pm 1,1)$ nach $\diag(\mp 1,-1)$. $i=n$ transformiere
    $\diag(\mp 1,-1)$.

    \noindent $i=n$ transformiere $\diag(m_i,…,m_i)$ nach $\diag(\pm1,1)$

    \[
        \leadsto \begin{pmatrix} \sign\det M& 0& \cdots&0\\ 0& 1 &\cdots&0\\\vdots&&\ddots&0\\ 0&\cdots&
        \cdots& 1\end{pmatrix}
    \]

    \end{description}
\end{proof}

\begin{theorem}
    Sei $f\in D_y^1(\ol U, \R^n)$, $y\nin \CV(f)$ und $\deg$ eine Gradabbildung. Dann gilt
    \[
        \deg(f,U,y)=\sum_{x\in f^{-1}(y)} \sign J_f(x),
    \]
    wobei die Summe endlich ist.
\end{theorem}

\begin{proof}
    O.B.d.A. $y=0$ (nach \textbf{D1}). Alle $x\in f^{-1} (0)$ sind isolierte Punkte in $U$
    (Homöomorphiesatz).
    $f^{-1}(0)$ hat höchstens am Rand einen Häufungspunkt, aber $0\nin f(\partial U)$.
    \[
        \Ra \, f^{-1}(0)= \{ x^i \}_{i=1}^N
    \]
    Wähle $\delta >0$ so klein, dass $B_\delta(x^i)$ paarw. disjunkt.
    \[
        \overset{\text{\ref{2.1} i)}}{\Ra} \, \deg(f,U,0)=\sum_{i=1}^N \deg(f,B_\delta(x^i),0).
    \]
    Beachte $0\nin f(\ol U \setminus \bigcup_{i=1}^N B_\delta(x^i))$.
    \begin{align*}
        f\in C^1 \, \Ra \, f(x)&= \d f(x)(x-x^i) + |x-x^i| r(x-x^i) \qq \text{mit} \qq r\in C^0(B_\delta(x^i),\R^n), 
        \qq r(0)=0.\\
           H(t,x) &:= \d f (x^i) (x-x^i)+ (1-t) (x-x^i)+(x-x^i)
    \end{align*}
    Zeige $0\nin H(t,\partial B_\delta(x^i))$, für alle $t\in [0,1]$.
    \[
        J_f(x^i)\neq 0 \, \Ra \, \exists\, \lambda >0:\, |\d f(x^i)(x-x^i)|\geq \lambda|x-x^i|
    \]
    O.B.d.A. sei $\delta$ so klein, dass $|r(x-x^i)|<\lambda$ in $B_\delta (x^i)$.
    \begin{align*}
        \Ra\, | H(t,x) |&>|\d f(x^i)(x-x^i)|-(1-t)(x-x^i)r(x-x^i)\geq \lambda \delta - \delta|r|>0
        \qq \forall \, x\in \partial B_\delta (x^i)\\
        \overset{(\textbf{D4})}{\Ra} \, \deg(f,U,0)&=\sum_{i=1}^N\deg (\d f (x^i)(\cdot-x^i),
                B_\delta(x^i),0)
    \end{align*}
    Lemma \ref{1.3.5}
    \[
        \Ra \, \deg(\d f (x^i)(\cdot- x^i),B_\delta (x^i),0)= \deg(\diag(\sign J_f(x^i),1,…,1),
            B_\delta(x^i),0)
    \]
    Falls $J_f(x^i)>0\, \overset{\textbf{(D2)}}{\Ra}\,\deg(I(\cdot-x^i),B_\delta (x^i),0)=1$. Es genüngt
    also, $\deg(M(\cdot-x^i),B_1(x^i),0)$. Zu berechnen, wobei $M=\diag(-1,1,…,1)$ ist.
    \begin{align*}
        U_1&:= B_1 (x^i) = \{ \max_{1\leq k\leq n} |x_k-x_k^i|<1 \}\\
        U_2&:= U_1+ (2,0,\cdot,0)\\
        g(r)&=2-|r-1|, \, h(r)=1-r^2\\
        f_1(x)&:= (1-g(x_1-x_1^i)h(x_2-x_2^(i))…h(x_n-x_n^{(i)}),…, 1)\\
        f_2(x)&:= (1,x_2-x_2^{(i)}, … , x_n-x_n^{(i)})\\
        f_1^{-1}(0)&= \{ y,z \} \qq y=x^i, z = x^i+(2,0,…,0)\\
        f_1|_{\partial U}&= f_2|_{\partial U} \qq \Ra \qq \deg(f_1,U_2,0)=0\\
        \Ra \, \deg (f_1,U,0)&= \deg(f_1,U_1,0)+ \deg(f_2,U_2,0) \qq (\star)\\
        \Ra \, \deg(M,B_1(x^i),0)&=\deg(\partial f_1(y),B_1(x^i), 0)\\
        &=\deg(f_1, U_1,0) \overset{(\star)}{=} -\deg (f_1,U_2,0)\\
        &= -\deg(\partial f_1(z),U_2,0)=\deg(\Id, U_2,0)=-1
    \end{align*}
    \[ \]
\end{proof}
