\section{Das Leray-Schauder-Prinzip und der Schauder'sche Fixpunktsatz}

Fixpunktsätze in schneller Folge!

\begin{theorem}[Leray-Schauder-Prinzip]\label{3.12}
    Es sei $F\in \ms C(X,X)$ und es existiere $M>0$, so dass jede Lösung $x$ von
    \[
        x=tF(x) \qq t\in [0,1]
    \]
    die \textit{a priori} Abschätzung
    \[
        \|x\|\leq M
    \]
    erfüllt. Dann besitzt $F$ einen Fixpunkt.
\end{theorem}

\begin{proof}
    Sei $\rho>M$. Es gilt
    \[
        \deg(\Id-F,B_\rho(0),0)=\deg(\Id,B_\rho(0),0)=1
    \]
    Aus der \textit{a priori} Abschätzung folgt
    \[
        H(t)\in \ms D_0(B_\rho(0),X) \qq \forall t\in [0,1]
    \]
    Somit existiert eine Lösung von $x-F(x)=0$.
    \[ \]
\end{proof}

\begin{theorem}[Schauder'sche Fixpunktsatz]\label{3.13}
    Sei $K$ ein abgeschlossene, konvexe, beschränkte Teilmenge von $X$, ein Banachraum. Sei
    $F\in \ms C(K,K)$. Dann besitzt $F$ einen Fixpunkt.
\end{theorem}

\begin{proof}
    $K$ beschränkt, also sei $\rho>0$ mit $K\subset B_\rho(0)$. Es sei $R$ die stetige Fortsetzung von
    $\Id:K\ra K$ auf $X$ nach Satz \ref{2.12}. Es gilt $R:X\ra K$, $R(x)=x$ für $x\in K$.

    \noindent\textit{Anmerkung}: So eine Abbildung heißt „Retraktion“, $K$ ist dann ein Retrakt von $X$.

    \noindent Wir betrachten $\tilde F=F\circ R\in \ms C(\ol{B_\rho(0)},\ol{B_\rho(0)})$. Mit $H(t)
    -t\tilde F$ folgt
    \[
        \deg(\Id-\tilde F, B_\rho(0),0)=\deg(\Id,B_\rho(0),0)=1
    \]
    Somit existiert $x_0=\tilde F(x_0)\in K \, \Ra \, \tilde F(x_0)=F(x_0)=x_0$. 
    \[ \]
\end{proof}

\begin{remark}
    Existenz eines Fixpunktes gilt natürlich auch für $F:G\ra G$, wobei $G$ nur homöomorph zu einer
    konvexen, abgeschlossen, beschränkten Teilmenge eines Banachraumes $X$ ist.
\end{remark}

\begin{theorem}\label{3.14}
    Sei $\Omega\subset X$ offen und beschränkt, $F\in \ms C(\ol\Omega,X)$. Angenommen, es existiert ein
    $x_0\in \Omega$, so dass
    \[
        F(x)-x_0\neq \alpha(x-x_0) \qq \forall x\in \partial \Omega, \, \forall \alpha\in(1,\infty)
    \]
    Dann besitzt $F$ einen Fixpunkt.
\end{theorem}

\begin{proof}
    Sei $H(t)(x)=x-x_0-t(F(x)-x_0)$. Dann gilt aber
    \[
        H(t)\neq 0 \qq \forall x \in \partial\Omega, \, t\in [0,1)
    \]
    Falls $H(1)(x)=0$ für ein $x\in \partial \Omega$, dann ist $x$ ein Fixpunkt von $F$ und wir
    sind fertig. Ansonsten gilt allerdings
    \[
        \deg(\Id-F,\Omega,0)=\deg(\Id-x_0,\Omega,0)=\deg(\Id,\Omega,x_0)=1
    \]
    und es existiert wieder ein Fixpunkt von $F$. 
    \[ \]
\end{proof}

\begin{cor}\label{3.15}
    Sei $\Omega\subset X$ offen und beschränkt, $F\in \ms C (\ol \Omega, X)$. Dann besitzt $F$ einen
    Fixpunkt falls \textit{eine} der folgenden Bedingungen erfüllt ist:
    \begin{description}
        \item{i)}
        $\Omega=B_\rho(0)$, $F(\partial \Omega)\subset \ol\Omega$ (Rohte)
        \item{ii)}
        $\Omega=B_\rho(0)$, $\|F(x)-x\|^2\geq \|F(x)\|^2-\|x\|^2$ $\forall x\in \partial \Omega$
        (Altmann)
        \item{iii)}
        $X$ ist ein Hilbertraum, $\Omega=B_\rho(0)$, $(F(x),x)\leq \|x\|^2 \qq \forall x 
        \in \partial \Omega$ (Krasnosel'shii).
    \end{description}
\end{cor}

\begin{proof}
    Übungsaufgabe. \[ \]
\end{proof}

\noindent \textbf{Beispiele:}

\begin{description}
    \item[1. nichtlineares 2-Punkt Randwertsproblem:]
    Wir betrachten auf $[0,T]$ das Randwertsproblem
    \begin{align}\label{11}
        \left.
        \begin{array}{rcll}
        g(t,u(t),u'(t))&=&u''(t)& 0\leq t \leq T \\
        u(0)=u(1)&=&0&
        \end{array}\right\}
    \end{align}
    Es sei $C_0^2=\{u\in C^2\, | \, u(0)=u(T)=0\}$

    \textbf{Behauptung:} $L$ ist bistetig.
    \begin{proof}
        \begin{description}
            \item{Injektivität:} klar.
            \item{Surjektivität:} Beispielsweise mittels \textit{Green'scher Funktion}:
            \begin{align*}
                K(s,t)&=2T\cdot\sum_{n=1}^\infty \frac1{ \pi^2n^2 } \sin \frac{n\pi s}{T}
                \sin\frac{n\pi t}{T}\\
                u(s)&=L^{-1} (f)(s)=-\int_0^TK(s,t)f(t)\d t
            \end{align*}
            \item{Stetigkeit:} klar.
            \item{Stetigkeit:} \textit{Satz über die offene Abbildung}.
        \end{description}
        \[ \]
    \end{proof}
    \begin{theorem}
        Sei $g:[0,T]\times \R \times \R\ra \R$ eine stetige beschränkte Funktion. Dann bestizt
        (\ref{11}) eine Lösung $u\in C_0^2$.
    \end{theorem}
    \begin{proof}
        Es sei $G:C^1([0,1])\ra C([0,1])$ definiert durch $G(u)(t)=g(t,u(t),u'(t))$ und es sei
        \[
            J: C_0^2 \ra C^1([0,T])
        \]
        die Einbettung. Dann ist (\ref{11}) äquivalent zu
        \[
            L(u)=GJ(u) \, \LRa \, u = GJL^{-1}(u).
        \]
        Es gilt $JL^{-1}$ ist kompakt, das folgt aus \textit{Arzelá-Ascoli}.
        $G$ ist ein beschränkter Operator, denn $g$ ist stetig und beschränkt, 
        somit ist $GJL^{-1}$ kompakt und es existiert $\rho>0$ mit $GJL^{-1}(B_\rho(0))\subset 
        B_\rho(0)$. Damit besitzt $GJL^{-1}$ einen Fixpunket.
        \[ \]
    \end{proof}
